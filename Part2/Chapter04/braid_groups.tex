%\chapter[Propriedades de \texorpdfstring{$B_n$}{Bn} e o grupo de tranças puras]{Propriedades de \texorpdfstring{$B_n$}{Bn} e o grupo de tranças puras}
%\label{cap-4}
\chaptermark{}
%
\hfill%
\begin{minipage}{10cm}
\begin{flushright}
\rightskip=0.5cm
\textit{``The essence of math is not to make simple things complicated, but to make complicated things simple.''}
\\[0.1cm]
\rightskip=0.5cm
--- Stan Gudder
\end{flushright}
\end{minipage}

\section{Propriedades de \texorpdfstring{$B_n$}{Bn} e o grupo de tranças puras}

    Vamos demonstrar algumas propriedades das tranças. Para evitar confusões, por vezes será usada a notação $1_n$ para indicar a identidade de $B_n$.
	%
	%\begin{prop}
	%	\label{centro de B_n}
	%	Para $1\leq i\leq n-1$, $(\sigma_1\sigma_2\cdots\sigma_{n-1})^n\sigma_i = \sigma_i(\sigma_1\sigma_2\cdots\sigma_{n-1})^n$, ou seja, $(\sigma_1\sigma_2\cdots\sigma_{n-1})^n$ pertence ao centro de $B_n$.
	%\end{prop}
	
	%\begin{proof}
	
	%\end{proof}
	
	%\begin{remark}
	%	É possível mostrar que $(\sigma_1\sigma_2\cdots\sigma_{n-1})^n$ gera o centro de $B_n$. Além disso, como $l((\sigma_1\sigma_2\cdots\sigma_{n-1})^n)\neq 0 = l(1_n)$ (sendo $l$ a função definida no Lema \eqref{homomorfismo de comprimento}), então $(\sigma_1\sigma_2\cdots\sigma_{n-1})^n\neq 1_n$ e, consequentemente, o centro de $B_n$ não é trivial.
	%\end{remark}
	%
	\begin{prop}
		\label{B_m subgrupo de B_n}
		Para $1\leq m\leq n$, a função $\psi: \underset{\sigma_i\mapsto\sigma_i}{B_m\to B_n}$ 
		para $1\leq i\leq m-1$ é um homomorfismo injetor de $B_m$ em $B_n$ e então podemos 
		considerar $B_m$ como um subgrupo de $B_n$. 
	\end{prop}
	%
	\begin{proof}
		Pelo Teorema \ref{apresentacao de B_n}, temos
		%
		\begin{align*}
		B_m = \langle \sigma_1, \sigma_2, \dots, \sigma_{m-1} | &\sigma_i\sigma_j = \sigma_j\sigma_i, 
		\text{ para }|i-j|>1 \\ 
		&\sigma_i\sigma_{i+1}\sigma_i = \sigma_{i+1}\sigma_i\sigma_{i+1}, 1\leq i\leq m-2\rangle, \\
		B_n = \langle \sigma_1, \sigma_2, \dots, \sigma_{m-1}, \dots, \sigma_{n-1} | \sigma_i\sigma_j 
		= &\sigma_j\sigma_i, \text{ para }|i-j|>1 \\ 
		&\sigma_i\sigma_{i+1}\sigma_i = \sigma_{i+1}\sigma_i\sigma_{i+1}, 1\leq i\leq n-2\rangle.
		\end{align*} 
		%
		Daí, como $B_n$ satisfaz as relações de $B_m$, então pelo Teorema \ref{teorema de Dyck}, 
		$\psi$ é homomorfismo.
		
		\par\vspace{0.3cm} Agora, suponha que existe $\beta\in B_n$ tal que $\psi(\beta) = 1_n$, 
		ou seja, existe uma trança $\beta$ de $m$ cordas que, quando adicionamos $n-m$ cordas retas, 
		se torna a trança trivial em $B_n$. Mas então devemos ter $\beta = 1_m$, logo $\psi$ é injetora.
	\end{proof}
	%
	\begin{remark}
		Como consequência da Proposição \ref{B_m subgrupo de B_n}, se $\beta\in B_m$ é 
		não trivial, então $\beta\in B_n$, $n\geq m$, também é não trivial (sendo $\beta\in B_n$ 
		a trança $\beta$ de $m$ cordas adicionada de $n-m$ cordas retas);
	\end{remark}
	%
	\begin{remark}
	    Em particular, $B_m\cong B_n$ se, e só se, $m=n$.
	\end{remark}
	%
	\begin{prop}
		\label{geradores de B_n tem ordem infinita}
		$B_n$ é \textbf{livre de torção}, i.e., todo gerador de $B_n$ tem ordem infinita. 
		Equivalentemente, $\sigma_i^k \neq 1, \forall k\in\mathbb{Z}^{\ast}$.
	\end{prop}
	%
	\begin{proof}
		Vamos demonstrar a Proposição \ref{geradores de B_n tem ordem infinita} por indução em $i$, 
		o índice dos geradores.	Como $B_2 = \langle \sigma_1 | - \rangle$, então $|\sigma_1| = \infty$. 
		Logo, em $B_n$, $|\sigma_1| = \infty$. Suponha, então, que $\sigma_2\in B_n$ tem ordem finita. 
		Logo, por definição, $\exists l\in\mathbb{Z}$ tal que $\sigma_2^l = 1_n$. Daí, 
		%
		\begin{equation*}
		    (\sigma_1\sigma_2\sigma_1)\sigma_2^l(\sigma_1\sigma_2\sigma_1)^{-1} = 1_n.
		\end{equation*}
		%
		Note que como $\sigma_i\sigma_{i+1}\sigma_i = \sigma_{i+1}\sigma_i\sigma_{i+1}$, 
		então $\sigma_i\sigma_{i+1}\sigma_i^{-1} = \sigma_{i+1}^{-1}\sigma_i\sigma_{i+1}$ e, 
		consequentemente, $\sigma_i\sigma_{i+1}^l\sigma_i^{-1} = \sigma_{i+1}^{-1}\sigma_i^l\sigma_{i+1}$.
		Substituindo na equação acima, temos
		%
		\begin{align*}
    		1_n &= \sigma_1\sigma_2\sigma_1\sigma_2^l\sigma_1^{-1}\sigma_2^{-1}\sigma_1^{-1} \\
    		&= \sigma_1\sigma_2\sigma_2^{-1}\sigma_1^l\sigma_2\sigma_2^{-1}\sigma_1^{-1} \\
    		&= \sigma_1^l
		\end{align*}
		%
		o que é absurdo, logo $|\sigma_2| = \infty$.
		
		\par\vspace{0.3cm} Então, suponha indutivamente que $|\sigma_j| = \infty, 1\leq j\leq i-1$. 
		Suponha que $|\sigma_i| = l, l\in\mathbb{Z}$. Então, 
		%
		\begin{align*}
    		1_n 
    		&= \sigma_{i-1}\sigma_i\sigma_{i-1}\sigma_i^l\sigma_{i-1}^{-1}\sigma_i^{-1}\sigma_{i-1}^{-1} \\
    		&= \sigma_{i-1}\sigma_i\sigma_i^{-1}\sigma_{i-1}^l\sigma_i\sigma_i^{-1}\sigma_{i-1}^{-1} \\
    		&= \sigma_{i-1}^l
		\end{align*}
		%
		o que é absurdo também. Então, por indução, $|\sigma_i| = \infty, i\geq 1$. 
	\end{proof}
	%
	\begin{remark}
		Como todo elemento não trivial de $B_n$ tem ordem infinita, é possível mostrar que para 
		toda trança $\beta$ não trivial de $B_n$, $\beta^k\neq 1_n$ se $k\neq 0$.
	\end{remark}
	%
	\begin{prop}
	\label{sigma1 e alfa geram B_n}
		O grupo $B_n$, $n\geq 1$, é gerado pelos elementos $\sigma_1$ e 
		$\alpha = \sigma_1\sigma_2\cdots\sigma_{n-1}$.
	\end{prop}
	
	\begin{proof}
		Queremos obter os $\sigma_i$, $i\geq 2$, em termos de $\sigma_1$ e $\alpha$. Primeiro, 
		vamos mostrar que 
		%
		\begin{equation}
		\label{conjugacao}
		    \sigma_{i+1} = \alpha\sigma_i\alpha^{-1}.
		\end{equation} 
		%
		Para isso, note que 
		%
		\begin{align*}
    		\alpha\sigma_i 
    		&= \sigma_1\sigma_2\cdots\sigma_{i-1}\sigma_i\sigma_{i+1}\sigma_{i+2}
    		\cdots\sigma_{n-1}\sigma_i \\
    		&= \sigma_1\sigma_2\cdots\sigma_{i-1}(\sigma_{i}\sigma_{i+1}\sigma_i)\sigma_{i+2}
    		\cdots\sigma_{n-1}  \\
    		&= \sigma_1\sigma_2\cdots\sigma_{i-1}(\sigma_{i+1}\sigma_i\sigma_{i+1})\sigma_{i+2}
    		\cdots\sigma_{n-1} \\
    		&= \sigma_{i+1}\sigma_1\sigma_2\cdots\sigma_{i-1}\sigma_i\sigma_{i+1}\cdots\sigma_{n-1} \\
    		&= \sigma_{i+1}\alpha,
		\end{align*}
		%
		em que usamos as duas relações do Teorema \ref{apresentacao de B_n}. Mas 
		$\alpha\sigma_i = \sigma_{i+1}\alpha$ é equivalente à equação \eqref{conjugacao}, como 
		queríamos demonstrar.
		
		\par\vspace{0.3cm} Agora, vamos mostrar, por indução, que 
		$\sigma_i = \alpha^{i-1}\sigma_1\alpha^{1-i}$. 
		
		\par\vspace{0.3cm} Para $i=1$, temos $\sigma_1 = \alpha^0\sigma_1\alpha^0 = \sigma_1$. 
		Suponha que $\sigma_{i-1} = \alpha^{(i-1)-1}\sigma_1\alpha^{1-(i-1)}$. Daí, temos
		%
		\begin{align*}
    		\alpha^{i-1}\sigma_1\alpha^{1-i} 
    		&= \alpha\Big( \alpha^{(i-1)-1}\sigma_1\alpha^{1- (i-1)} \Big) \alpha^{-1} \\
    		&= \alpha\sigma_{i-1}\alpha^{-1} \\
    		&= \sigma_i,
		\end{align*}
		%
		em que usamos a equação \eqref{conjugacao} na última igualdade. 
		
		\par\vspace{0.3cm} Ora! Então, como $\sigma_i = \alpha^{i-1}\sigma_1\alpha^{1-i}$, 
		podemos representar todo gerador de $B_n$ em termos de $\sigma_1$ e $\alpha$. 
		Consequentemente, $\sigma_1$ e $\alpha$ geram $B_n$.
	\end{proof}
	%
	\begin{definition}[Grupo de Tranças puras]
	\label{def P_n}
	\index{Grupo de Tranças Puras}
		O núcleo do homomorfismo $\pi$ definido no Lema \ref{B_n iso S_n} é chamado grupo de tranças 
		puras e denotado por $P_n$. Uma trança geométrica de $n$ cordas representa um elemento de 
		$P_n$ se, e só se, para todo $i=1,2\dots,n$, a corda dessa trança ligada ao ponto $(i,0,0)$ 
		tem o segundo ponto fixo em $(i,0,1)$. Em símbolos, $P_n = \Ker(\pi: B_n\to S_n )$.
	\end{definition}
	%
	Em outras palavras, o grupo de tranças puras $P_n$ nada mais é que o grupo formado pelas 
	tranças cujas cordas chegam na posição original, correspondendo à permutação identidade. 
	Abaixo segue um exemplo de trança pura.
	%
	\begin{center}
		\begin{tikzpicture}
		\braid[braid colour=black,strands=4,braid start={(0,0)}]	{\sigma_3\sigma_2\sigma_1\sigma_1\sigma_2^{-1}\sigma_3^{-1}}
		%\node[font=\Huge] at (5.5,-5.5) {\(=\)};
		%\braid[strands=3,braid start={(5,0)}]
		%{\sigma_1 \sigma_2 \sigma_3\sigma_4\sigma_1\sigma_2\sigma_3\sigma_1\sigma_2\sigma_1\sigma_3}
		\end{tikzpicture}
		%\begin{tikzpicture}
		%\braid[braid colour=black,strands=5,braid start={(0,0)}]	{\sigma_4\sigma_3\sigma_2\sigma_2\sigma_3^{-1}\sigma_4^{-1}}
		%	\node[font=\Huge] at (5.5,-5.5) {\(=\)};
		%	\braid[strands=5,braid start={(5,0)}]
		%	{\sigma_1 \sigma_2 \sigma_3\sigma_4\sigma_1\sigma_2\sigma_3\sigma_1\sigma_2\sigma_1\sigma_3}
		%\end{tikzpicture}	
	\end{center}
	%
	Note que todas as cordas saem e chegam à posição original.
	%
	\begin{prop}
	\label{P_n normal em B_n}
		Temos que $P_n\vartriangleleft B_n$. Além disso, $B_n/P_n \cong S_n$ e 
		$[B_n:P_n] = |B_n/P_n| = |S_n| = n!$.
	\end{prop}
	%
	\begin{proof}
		Suponha 
		$\alpha = \begin{bmatrix}
		1 & 2 & \cdots & n \\
		i_1 & i_2 & \cdots & i_n
		\end{bmatrix}$ uma permutação de $S_n$. Queremos construir uma trança de $n$ cordas baseados 
		em $\alpha$. Para isso, tome $n$ pontos $A_1, A_2, \dots, A_n$ na reta $l_1$ e $n$ pontos 
		$B_1, B_2, \dots, B_n$ na reta $l_2$ paralela e ``abaixo'' de $l_1$. Tome os segmentos $d_j$ 
		que ligam o ponto $A_j$ ao ponto $B_{i_j}$ e escolha para todo cruzamento dos $d_j$, 
		escolha arbitrariamente se ele é por cima ou por baixo. Desse modo, formamos uma trança, 
		$\gamma$ que, por construção, é tal que $\pi(\gamma) = \alpha$. Portanto, o homomorfismo $\pi$ 
		do Lema \ref{B_n iso S_n} é sobrejetor. 
		
		\par\vspace{0.3cm} Por exemplo, para a permutação $(1\text{ }2\text{ }3\text{ }4\text{ }5)$ 
		em $S_5$, podemos construir a seguinte trança:
		%
		\begin{center}
			\begin{tikzpicture}
			\braid[braid colour=black,strands=5,braid start={(0,0)}]	{\sigma_4\sigma_3\sigma_2\sigma_1}
			%\node[font=\Huge] at (5.5,-5.5) {\(=\)};
			%\braid[strands=3,braid start={(5,0)}]
			%{\sigma_1 \sigma_2 \sigma_3\sigma_4\sigma_1\sigma_2\sigma_3\sigma_1\sigma_2\sigma_1\sigma_3}
			\end{tikzpicture}
		\end{center}
		%
		Pela Definição \ref{def nucleo}, $\Ker\pi = \left\{ \gamma\in B_n | \pi(\gamma) = (1) \right\} = P_n$.
		Daí, pelo Teorema \ref{subgrupos normais e nucleos}, temos $P_n\triangleleft B_n$ e, pelo 
		Teorema \ref{primeiro teorema de isomorfismo}, temos $B_n/P_n\cong S_n$. Consequentemente, 
		$[B_n:P_n] = |B_n/P_n| = |S_n| = n!$.
	\end{proof}
	%
	Para $P_n$, encontrar um conjunto de geradores não é tão simples quanto foi com $B_n$, pois nem 
	sempre podemos ``fatiar'' uma trança pura em pedações óbvios que também são puros. Michael Artin 
	mostrou que, para $1\leq i<j\leq n$, se definirmos
	%
	\begin{equation*}
	\tag{Geradores de Artin}
	\label{geradores de Artin}
	\index{Geradores de Artin}
	    A_{i,j} = 
	    \sigma_{j-1}\sigma_{j-2}
	    \cdots\sigma_{i+1}\sigma_i^2\sigma_{i+1}^{-1}
	    \cdots\sigma_{j-2}^{-1}\sigma_{j-1}^{-1},
	\end{equation*}
	%
	então o conjunto $\{ A_{i,j} \}_{i,j}$ gera $P_n$. Em particular note que, usando um 
	argumento combinatório, a cardinalidade desse conjunto é
	%
	\begin{align*}
	    (n-1) + (n-2) + \cdots + 2 + 1 = \frac{n(n-1)}{2} = \binom{n}{2}.
	\end{align*}
	%
	Abaixo está o diagrama de $A_{i,j}$, sendo $i$ a corda mais à esquerda e $j$ a corda mais à direita.
	%
	\begin{center}
		\begin{tikzpicture}
		\braid[braid colour=black,strands=6,braid start={(0,0)}]	{\sigma_5\sigma_4\sigma_3\sigma_2\sigma_1\sigma_1\sigma_2^{-1}\sigma_3^{-1}\sigma_4^{-1}\sigma_5^{-1}}
		%\node[font=\Huge] at (5.5,-5.5) {\(=\)};
		%\braid[strands=3,braid start={(5,0)}]
		%{\sigma_1 \sigma_2 \sigma_3\sigma_4\sigma_1\sigma_2\sigma_3\sigma_1\sigma_2\sigma_1\sigma_3}
		\end{tikzpicture}
	\end{center}
	%
	Algo interessante de se notar é que as tranças do conjunto $\{A_{i,j}\}_{i,j}$ são conjugadas umas às outras em $B_n$. De fato, seja
	%
	\[\alpha_{i,j} = \sigma_{j-1}\sigma_{j-2}\cdots\sigma_i
	\]
	%
	para $1\leq i<j\leq n$. Através de diagramas, podemos verificar que valem as relações abaixo, dados quaisquer $1\leq i<j<k\leq n$.
	%
	\begin{equation*}
	    \alpha_{j,k}A_{i,j}\alpha_{j,k}^{-1} = A_{i,k} 
	    \quad\text{ e }\quad 
	    \alpha_{i,k}A_{i,j}\alpha_{i,k}^{-1} = A_{j,k}.
	\end{equation*}
	%
	Abaixo mostramos a primeira relação para $i=1, j=5, k=7=n$.
	%
	\begin{center}
		\begin{tikzpicture}
		\braid[braid colour=black,strands=7,braid start={(0,0)}]	{\sigma_6\sigma_5\sigma_4\sigma_3\sigma_2\sigma_1\sigma_1\sigma_2^{-1}\sigma_3^{-1}\sigma_4^{-1}\sigma_5^{-1}\sigma_6^{-1}}
		%\node[font=\Huge] at (5.5,-5.5) {\(=\)};
		%\braid[strands=3,braid start={(5,0)}]
		%{\sigma_1 \sigma_2 \sigma_3\sigma_4\sigma_1\sigma_2\sigma_3\sigma_1\sigma_2\sigma_1\sigma_3}
		%\braid[braid colour=black,strands=7,braid %start={(7,0)}]{\sigma_6\sigma_5\sigma_5\sigma_6^{-1}}
		\end{tikzpicture}
	\end{center}
	%
	\section{Centro de \texorpdfstring{$B_n$}{Bn}}\label{secao centro de B_n}
	Para estudar o centro de $B_n$ (e também de $P_n$), a trança abaixo é muito útil, chamada 
	\textit{volta completa} e denotada por $\Delta_n^2$. 
	%
	\begin{figure}[H]
		\begin{center}
			\includegraphics[width=6.5cm]{Images/volta_completa.png}
		\end{center}\caption{A volta completa, $\Delta_n^2$.}
		\label{full twist}
	\end{figure}
	%
	Da Figura \ref{full twist}, podemos ver que $\Delta_n^2 = (\sigma_1\sigma_2\cdots\sigma_{n-1})^n$. 
	Além disso, podemos também denotar por $\Delta_n$ a meia volta, e podemos escrevê-la em termos 
	dos geradores $\sigma_i$ de $B_n$ da seguinte forma: 
	$\Delta_n 
	= (\sigma_1\sigma_2\cdots\sigma_{n-1})
	(\sigma_1\sigma_2\cdots\sigma_{n-2})
	\cdots(\sigma_2\sigma_1)\sigma_1$. Daí, como a notação sugere, $\Delta_n^2$ realmente é o quadrado 
	de uma trança (e também a raiz $n$-ésima de uma trança, como podemos ver da Figura \ref{full twist}).
	
	\par\vspace{0.3cm} Em particular, note que a meia volta não comuta, em geral, com os geradores 
	de $B_n$, uma vez que $\Delta_n\sigma_2\neq\sigma_2\Delta_n$, por exemplo. De fato, o que 
	ocorre é que $\sigma_i\Delta_n = \Delta_n\sigma_{n-i}$, ou seja, o cruzamento $\sigma_i$ 
	``desliza'' (é claro que se $n$ é par e $i=n/2$, então $\Delta_n$ comuta com $\sigma_i$).
	%
	\begin{figure}[H]
		\label{delta5}
		\begin{center}	
			\includegraphics[width=6cm]{Images/delta5.png}
		\end{center}\caption{Diagrama de $\Delta_5$. Podemos ver que o cruzamento $\sigma_{i}$ ``desliza'', se tornando $\sigma_{n-i}$.}
	\end{figure}
	%
	%\begin{center}
	%	\begin{tikzpicture}
	%	\label{delta}
	%	\braid[braid colour=black,strands=5,braid start={(0,0)}]	{\sigma_2\sigma_1 \sigma_2 \sigma_3\sigma_4\sigma_1\sigma_2\sigma_3\sigma_1\sigma_2\sigma_1}
	%	\node[font=\Huge] at (5.5,-5.5) {\(=\)};
	%	\braid[strands=5,braid start={(5,0)}]
	%	{\sigma_1 \sigma_2 \sigma_3\sigma_4\sigma_1\sigma_2\sigma_3\sigma_1\sigma_2\sigma_1\sigma_3}
	%	\end{tikzpicture}
	%\end{center}
	%
	Contudo, $\Delta_n^2$ comuta com toda trança de $B_n$, como é mostrado abaixo.
	%
	\begin{proof}
		%Queremos mostrar que $\forall\beta\in B_n$, $\Delta_n^2\beta = \beta\Delta_n^2$. Isso equivale a mostrar que $\Delta_n^2\sigma_i = \sigma_i\Delta_n^2$. Por outro lado, como mostrado na Proposição \eqref{sigma1 e alfa geram B_n}, todo gerador de $B_n$ pode ser escrito em termos de $\sigma_1$ e $\alpha = \sigma_1\sigma_2\cdots\sigma_{n-1}$. Como $\alpha$ comuta com $\alpha^n = \Delta_n^2$, basta mostrarmos que $\sigma_1$ comuta com $\alpha^n$.
		Primeiro, note que $\sigma_i\Delta_n = \Delta_n\sigma_{n-i}$, como ilustram os diagramas acima. 
		Daí, temos
		%
		\begin{equation*}
		    \sigma_i\Delta_n^2 = \sigma_i\Delta_n\Delta_n = \Delta_n\sigma_{n-i}\Delta_n = \Delta_n^2\sigma_i,
		\end{equation*}
		%
		ou seja, $\Delta_n^2$ comuta com todo gerador de $B_n$ e, consequentemente, com toda trança de $B_n$.
	\end{proof}
	%
	Ora! Então $\Delta_n^2$ pertence ao centro de $B_n$. Na verdade, é possível mostrar que $\Delta_n^2$ 
	gera o centro de $B_n$ para $n\geq 3$, i.e.
	%
	\begin{theorem}
	\label{centro de B_n}
		Se $n\geq 3$, então $Z(B_n) = Z(P_n) = \langle \Delta_n^2 \rangle 
		= \langle (\sigma_1\sigma_2\cdots\sigma_{n-1})^n \rangle$.
	\end{theorem}
	%
	Em outras palavras, toda trança do centro de $B_n$ é uma potência de $\Delta_n^2$.
	%, isto é, toda trança que comuta com todas as outras é uma potência da volta completa.
	Demonstramos o Teorema \ref{centro de B_n} para $n=3$.
	%
	\begin{prop}
		$Z(B_3) = \langle (\sigma_1\sigma_2)^3 \rangle$.
	\end{prop}
	%
	\begin{proof}
		Primeiro, note que 
		$(\sigma_1\sigma_2)^3\sigma_1 = \sigma_1\sigma_2\sigma_1\sigma_2\sigma_1\sigma_2\sigma_1 
		= \sigma_1(\sigma_1\sigma_2\sigma_1\sigma_2\sigma_1\sigma_2) 
		= \sigma_1(\sigma_1\sigma_2)^3$ e também que 
		$(\sigma_1\sigma_2)^3\sigma_2 
		= \sigma_1\sigma_2\sigma_1\sigma_2\sigma_1\sigma_2\sigma_2 
		= \sigma_2(\sigma_1\sigma_2\sigma_1\sigma_2\sigma_1\sigma_2) 
		= \sigma_2(\sigma_1\sigma_2)^3$. Portanto, 
		$\langle (\sigma_1\sigma_2)^3 \rangle\subseteq Z(B_3)$.
		
		\par\vspace{0.3cm} Agora, seja $g\in Z(B_3)$. Então, devemos ter, necessariamente, 
		$g\sigma_1 = \sigma_1g$ e $g\sigma_2 = \sigma_2g$. Suponha, então, $g = \sigma_1^i\sigma_2^j$. Daí, temos
		%
		\begin{align*}
    		\begin{cases}
        		\sigma_1^i\sigma_2^j\sigma_1 = \sigma_1^{i+1}\sigma_2^j \\
        		\sigma_1^i\sigma_2^{j+1} = \sigma_2\sigma_1^{i}\sigma_2^j
    		\end{cases} 
    		\Rightarrow 
    		\begin{cases}
        		\sigma_2^j\sigma_1 = \sigma_1\sigma_2^j \\
        		\sigma_1^i\sigma_2 = \sigma_2\sigma_1^i
    		\end{cases} 
    		\Rightarrow i = j = 0.
		\end{align*} 
		%
		Suponha, agora, $g = (\sigma_1\sigma_2)^i$, $i\neq 0$. Então, temos
		%
		\begin{align*}
    		\begin{cases}
        		(\sigma_1\sigma_2)^i\sigma_1 = \sigma_1(\sigma_1\sigma_2)^i \\
        		(\sigma_1\sigma_2)^i\sigma_2 = \sigma_2(\sigma_1\sigma_2)^i
    		\end{cases} 
    		\Rightarrow i = 3k, k\in\mathbb{Z}.
		\end{align*}
		%
		A última implicação se deve ao seguinte raciocínio. 
		A quantidade de termos no produto $(\sigma_1\sigma_2)^i\sigma_1$ é $2i+1$. 
		Para transformar $(\sigma_1\sigma_2)^i\sigma_1$ em $\sigma_1(\sigma_1\sigma_2)^i$, 
		devemos manipular os $2i$ termos de $\sigma_2(\sigma_1\sigma_2)^{i-1}\sigma_1$ 
		usando a relação $\sigma_1\sigma_2\sigma_1 = \sigma_2\sigma_1\sigma_2$ 
		para chegar em $(\sigma_1\sigma_2)^i$. Ora! Mas para utilizar a relação, 
		precisamos de blocos de $3$ geradores, ou seja, devemos ter $i\in\mathbb{Z}$ 
		tal que $\displaystyle{\frac{2i}{3}\in\mathbb{Z}}$. Como $2$ e $3$ são relativamente primos, 
		devemos ter $i$ múltiplo de $3$.
		Por exemplo, 
		%
		\[
		\sigma_2(\sigma_1\sigma_2)^{4-1}\sigma_1 
		= \sigma_2\sigma_1\sigma_2\sigma_1\sigma_2\sigma_1\sigma_2\sigma_1 
		= \sigma_1\sigma_2\sigma_1\sigma_2\sigma_1\sigma_2\sigma_2\sigma_1 
		= (\sigma_1\sigma_2)^3\sigma_2\sigma_1\neq(\sigma_1\sigma_2)^4.
		\]
		%
		Daí, concluímos que todo elemento do centro de $B_3$ é uma potência de $(\sigma_1\sigma_2)^3$, 
		ou seja, $Z(B_3)\subseteq\langle (\sigma_1\sigma_2)^3 \rangle$. Portanto, 
		$Z(B_3) = \langle (\sigma_1\sigma_2)^3 \rangle$.
	\end{proof}
	%
	\begin{prop}
		Temos $l(\Delta_n^2) = n(n-1)$, sendo $l$ a função homomorfismo de comprimento definida 
		no Lema \ref{homomorfismo de comprimento} e $\Delta_n^2$ a volta completa em $B_n$.
	\end{prop}
	%
	\begin{proof}
		Escrevendo $\Delta_n^2 = (\sigma_1\sigma_2\cdots\sigma_{n-1})^n$, segue da definição 
		de $l$ que $l(\Delta_n^2) = n(n-1)$. 
		
		\par\vspace{0.3cm} Também podemos escrever $l(\Delta_n^2) = 2l(\Delta_n)$ e, como
		%
		\[
		\Delta_n 
		= (\sigma_1\sigma_2\cdots\sigma_{n-1})(\sigma_1\sigma_2\cdots\sigma_{n-2})
		\cdots(\sigma_1\sigma_2)\sigma_1,
		\]
		% 
		então
		%
		\[
		l(\Delta_n^2) 
		= 2\Big( (n-1) + (n-2) + \cdots + 2 + 1 \Big) = 2\Big( \displaystyle{\frac{n(n-1)}{2}} \Big) = n(n-1).
		\]
		%
	\end{proof}
	%
	%\begin{remark}
	%	No Teorema \eqref{centro de B_n}, usamos o fato de que $Z(B_n) = Z(P_n)$. 
	%Isso se deve ao fato de que $P_n$ é homomorfo a $B_n$, logo (como um homomorfismo preserva a operação) 
	%o centro de $B_n$ é levado no centro de $P_n$.
	%\end{remark}
	%
	%\begin{theorem}
	%	$B_n$ e todos seus subgrupos são residualmente finitos.
	%\end{theorem}
	%
	Podemos ainda definir $f_n: P_n\to P_{n-1}$ como sendo a função que 
	pega uma trança em $P_n$, retira a $n$-ésima 
	corda e a mapeia à trança resultante em $P_{n-1}$. Essa função é um homomorfismo sobrejetor, chamada
	\textit{homomorfismo esquecido}.
	
	\par\vspace{0.3cm} Para $n\geq 2$, definimos $U_n\vcentcolon=\Ker(f_n: P_n\to P_{n-1})$. 
	Do diagrama do gerador de Artin, $A_{i,j}$, fica claro que $A_{i,n}\in U_n$, com $1\leq i\leq n-1$. 
	%
	\begin{theorem}
	\label{U_n livre nos geradores de Artin}
		Para todo $n\geq 2$, $U_n$ é livre nos $n-1$ geradores $\{ A_{i,n} \}_{i=1,2,\dots,n-1}$.
	\end{theorem}
	%
	Aceitaremos o Teorema \ref{U_n livre nos geradores de Artin} sem demonstração. 
	Contudo, um corolário interessante é o seguinte.
	%
	\begin{corollary}
	\label{B_n residualmente finito}
		$B_n$ e seus subgrupos são residualmente finitos.
	\end{corollary}
	%
	\begin{proof}
		Um grupo $G$ é dito residualmente finito se para todo $\beta\in G-\{1\}$ existe 
		um homomorfismo $f$ de $G$ em um grupo finito tal que $f(\beta)\neq 1$.
		
		\par\vspace{0.3cm} Sabemos que grupos livres são residualmente finitos e que o produto 
		semidireto de dois grupos finitamente gerados e residualmente finitos é também residualmente finito.
		
		\par\vspace{0.3cm} Também sabemos, do Teorema \ref{subgrupos normais e nucleos}, que
		$U_n\vartriangleleft P_n$. Da Definição \ref{produto semidireto}, temos 
		$P_n\cong U_n\rtimes P_{n-1}$. Tanto $U_n$ quanto $P_{n-1}$ são finitamente gerados e do 
		Teorema \ref{U_n livre nos geradores de Artin}, $U_n$ é residualmente finito. 
		Daí, por indução em $n$, o Teorema \ref{U_n livre nos geradores de Artin} implica 
		$P_n$ residualmente finito.
		
		\par\vspace{0.3cm} Note que qualquer extensão de um grupo residualmente finito $P$ por 
		um grupo finito é residualmente finita. Como $B_n$ é uma extensão de $P_n$ por $S_n$ e $P_n$ é
		residualmente finito, concluímos que $B_n$ também é.
		
		\par\vspace{0.3cm} Por fim, observe que todos os subgrupos de um grupo residualmente 
		finito são também residualmente finitos.	
	\end{proof}
	%
	\section{Tranças como espaços de configuração}\label{secao trancas como espacos de configuracao}
	Há varias conexões entre grupos de tranças e topologia, e uma delas envolve 
	\textit{espaços de configuração}, que nada mais são do que espaços que contêm 
	todos os estados possíveis de um sistema. Por vezes, espaços de configuração são 
	chamados de \textit{espaços de estado} ou \textit{espaços de parâmetros}
	\index{Espaços de configuração}.
	
	\par\vspace{0.3cm} Por exemplo, podemos usar um espaço de configuração para modelar o 
	movimento coletivo de vários objetos, como carros nas ruas de uma cidade ou moléculas em 
	uma solução ou ainda robôs em uma fábrica.
	
	\par\vspace{0.3cm} Para um exemplo mais concreto, considere o seu braço. Nele, há três juntas: 
	uma no ombro, uma no cotovelo e uma no pulso. Tanto o seu ombro quanto o seu pulso têm dois 
	graus de liberdade (i.e., podem girar em dois sentidos diferentes), enquanto que o seu cotovelo 
	tem apenas um grau de liberdade, totalizando cinco dimensões de configuração.
	
	\par\vspace{0.3cm} Um braço robótico modelado a partir do seu braço tem de navegar por um 
	espaço de configuração cinco-dimensional para poder aproveitar toda a flexibilidade existente 
	nesse sistema de juntas. Para ilustrar, imagine que ao invés de uma mão, você tem uma plataforma 
	rígida conectada ao seu pulso e suponha que você queira levantar um copo d'água de abaixo da sua 
	cintura até acima do seu ombro. Isso não será possível de fazer com a plataforma rígida no lugar da 
	mão, porque apenas a rotação do ombro não é suficiente para fazer o movimento desejado (isso é 
	o que os bebês fazem, e eles sempre derramam a água).
	
	\par\vspace{0.3cm} Mas qual a conexão entre espaços de configuração e grupos de trança? 
	Bom, até agora, falamos de tranças individuais como objetos topológicos, como indicam as 
	Definições \ref{def geometrica tranca} e \ref{outra def geometrica tranca}. Contudo, os grupos 
	de trança em si também são objetos topológicos, no sentido de que cada grupo de trança descreve, 
	de forma natural, os diferentes tipos de \textit{loops} que existem em um certo espaço de configuração. 
	O espaço em questão modela o movimento coletivo de $n$ partículas distintas no plano que não 
	podem colidir.
	
	\par\vspace{0.3cm} Então, imagine $n$ partículas no plano e estenda esse plano para formar 
	a parede esquerda de uma trança. Agora, deslize o plano da esquerda para a direita, e imagine 
	que cada partícula deixa um rastro à medida que o plano se move. O que acontece?
	
	\par\vspace{0.3cm} Bom, se as partículas ficam paradas (no plano), então quando o plano parar, 
	você terá $n$ trilhas paralelas: a trança identidade!
	%
	\begin{figure}[H]
		\begin{center} 
			\includegraphics[width=9cm]{Images/tranca_identidade_conf.png}
		\end{center}\caption{A trança identidade em um espaço de configuração}\label{tranca identidade espaco configuracao}
	\end{figure}
	%
	Entretanto, se as partículas se movem no plano à medida que eles desliza, o que ocorre? 
	Bom, desde que nenhum par de partículas colida, veremos uma trança! E claramente toda trança 
	pode ser feita desse modo. Veja a figura abaixo. Note que estamos visualizando a direção 
	$x$ como o eixo temporal.
	
	\par\vspace{0.3cm} Para um exemplo mais visual, podemos observar o movimento dos planetas 
	do Sistema Solar, como 
	\href{https://www.youtube.com/watch?v=0jHsq36_NTU&list=WL&index=2&t=50s}{nesse vídeo}. 
	%
	\begin{figure}[H]
		\begin{center}
			\includegraphics[width=10cm]{Images/tranca_conf.png}
		\end{center}\caption{Uma trança não trivial em um espaço de configuração.}\label{tranca espaco configuracao}
	\end{figure} 
	%
	Para fazermos o produto de duas tranças usando essa ideia, precisamos ter certeza de que 
	o conjunto de partículas no final da trança está no mesmo lugar que no começo da trança, 
	pois assim podemos concatenar duas tranças sem haver nenhuma descontinuidade no meio.
	
	\par\vspace{0.3cm} A posição inicial é uma \textbf{configuração} de $n$ partículas. 
	O conjunto de todas as configurações possíveis é o espaço de configuração
	%
	\begin{align}
	\label{configuracao ordenada}
	    C_n(\mathbb{R}^2) 
	    = \{ (p_1, \dots, p_n)\in (\mathbb{R}^2)^n \ | \ p_i\neq p_j \text{ para }i\neq j \},
	\end{align}
	%
	em que a condição $p_i\neq p_j$ indica a necessidade de um par partículas não colidirem.
	
	\par\vspace{0.3cm} O que definimos em \eqref{configuracao ordenada} foi o conjunto de 
	configurações \textbf{ordenadas}. Como o nosso propósito é descrever tranças, gostaríamos 
	de ignorar a ordem e apenas focar no conjunto de partículas, e.g., queremos considerar $(p_1, p_2)$ 
	como o mesmo que $(p_2, p_1)$ e pensar em ambos apenas como $\{p_1, p_2\}$. Para isso, 
	definimos a versão não ordenada:
	%
	\begin{align}
	\label{configuracao nao ordenada}
	    UC_n(\mathbb{R}^2) 
	    = \{ \{ p_1, \dots, p_n\}\subset\mathbb{R}^2 \ | \ p_i\neq p_j \text{ para }i\neq j \}
	\end{align} 
	%
	ou, em palavras, o conjunto de subconjuntos de $n$ pontos do plano. 
	
	\par\vspace{0.3cm} Podemos também denotar $C_n(\mathbb{R}^2)$ por $\mathbb{F}_n(\mathbb{R}^2)$ 
	e $UC_n(\mathbb{R}^2)$ por $\widetilde{\mathbb{F}}_n(\mathbb{R}^2)$, sendo $\sim$ a 
	relação de equivalência cujas classes são os conjuntos de permutações de um dado 
	ponto em $(\mathbb{R}^2)^n$.
	
	\par\vspace{0.3cm} De forma geral, se $\mathbb{M}$ é uma variedade de dimensão maior ou 
	igual a 2, então $\mathbb{F}_n(\mathbb{M})$ denota o subespaço de $\mathbb{M}^{(n)}$ 
	definido por
	%
	\begin{align}
	\label{espaco de configuracao de M}
	    \mathbb{F}_n(\mathbb{M}) 
	    = \{ (x_1, \dots, x_n)\in \mathbb{M}^{(n)} \ | \ x_i\neq x_j \text{ para }i\neq j \}.
	\end{align}
	%
	Esse subespaço é chamado \textit{espaço de configuração} de $\mathbb{M}$. Sendo $\sim$ a mesma 
	relação de equivalência definida acima, o espaço quociente
	$^{\displaystyle{\mathbb{F}_n(\mathbb{M})}}/_{\sim}$, que também 
	pode ser denotado por $\widetilde{\mathbb{F}}_n(\mathbb{M})$, é definido por
	%
	\begin{align}
	\label{espaco de configuracao nao ordenado de M}
	    \widetilde{\mathbb{F}}_n(\mathbb{M}) 
	    = \{ \{ x_1, \dots, x_n\}\subset\mathbb{M} \ | \ x_i\neq x_j \text{ para }i\neq j \}.
	\end{align} 
	%
	Pelo modo como definimos $\sim$ acima, podemos ver que 
	$UC_n(\mathbb{R}^2) = \displaystyle{C_n(\mathbb{R}^2)}/\displaystyle{S_n}$. 
	Agora, voltando na Figura \ref{tranca espaco configuracao}, tanto a parede da esquerda 
	quando a parede da direita representam pontos de $UC_3(\mathbb{R}^2)$. Na verdade, 
	as duas paredes representam o mesmo ponto (mas não o mesmo ponto de $C_3(\mathbb{R}^2)$, 
	pois a configuração final é $(p_2, p_3, p_1)$, sendo $p_1$ o ponto inferior e $p_3$ o ponto superior).
	
	\par\vspace{0.3cm} Além disso, quando deslizamos a parede da esquerda para a direita de forma 
	a criar a trança, em cada instante de tempo temos um ponto de $UC_3(\mathbb{R}^2)$. Isso se 
	deve ao fato de que cada corda é monotônica. Em outras palavras, podemos ver a trança como um
	\textit{loop} em $UC_3(\mathbb{R}^2)$.
	
	\par\vspace{0.3cm} Para fixar, podemos pensar em esquerda-direita como o eixo temporal, 
	e então a trança tridimensional toda é como um filme de três partículas dançando no 
	plano bidimensional (sem colidir) e retornando ao lugar onde começaram.
	
	\par\vspace{0.3cm} Equivalentemente, a trança é o gráfico de uma função 
	$[0,1]\to UC_3(\mathbb{R}^2)$, cujos pontos inicial e final concordam, ou seja, 
	podemos pensar em tranças como \textit{loops} de configurações. De fato, essa é, 
	essencialmente, uma correspondência injetiva.
	
	\par\vspace{0.3cm} Antes de prosseguirmos, vamos definir de maneira informal 
	o que vem a ser um grupo fundamental.
	\medskip
	%
	\begin{definition}[Grupo Fundamental]
	\label{def informal grupo fundamental}
	\index{Grupo Fundamental}
		Dado um espaço topológico $X$ e um ponto qualquer $x_0\in X$, 
		o conjunto de loops que começam e 
		terminam em $x_0$ e que são caminhos fechados em $X$ é chamado grupo fundamental de $X$ 
		com ponto base $x_0$ e denotado por $\pi_1(X,x_0)$.
	\end{definition}
	%
	Um detalhe importante é que se $X$ é conexo por caminhos (i.e., dado um par de pontos em $X$, 
	existe um caminho totalmente contido em $X$ que liga esses dois pontos), então a escolha do 
	ponto base não faz diferença. Esse é o nosso caso, e então denotaremos o grupo fundamental 
	de $X$ por $\pi_1(X)$ apenas.
	
	\par\vspace{0.3cm} Com a Definição \ref{def informal grupo fundamental}, podemos enunciar 
	o seguinte teorema.
	%
	\begin{theorem}
	\label{grupo fundamental de tranca}
		O grupo fundamental $\pi_1(UC_n(\mathbb{R}^2))$ é isomorfo ao grupo de trança $B_n$, 
		e o grupo fundamental $\pi_1(C_n(\mathbb{R}^2))$ é isomorfo ao grupo de tranças puras $P_n$.
	\end{theorem} 
	%
	Da Definição \ref{def informal grupo fundamental}, podemos ver que o grupo fundamental trata 
	sobre os diferentes \textit{loops} em um espaço topológico. A noção de equivalência de 
	\textit{loops} (i.e., homotopia, que nada mais é que uma deformação contínua de um caminho 
	para outro) se traduz diretamente para nossa noção de equivalência de tranças.
	
	\par\vspace{0.3cm} Portanto, o Teorema \ref{grupo fundamental de tranca} nos diz não só 
	que tranças podem ser vistas como \textit{loops} em um espaço de configuração, mas também 
	que podemos construir todos os \textit{loops} nesse espaço de configuração dessa maneira.
	
	\par\vspace{0.3cm} Além de nos proporcionar um outro modo de pensar em tranças,
	essa perspectiva topológica é útil tanto para provar coisas sobre tranças quanto 
	para realizar generalizações interessantes. Por exemplo, é possível demonstrar o 
	Teorema \ref{apresentacao de B_n} usando essas ideias. 
	
	\par\vspace{0.3cm} Para outro exemplo, observe que a própria notação, 
	$C_n(\mathbb{R}^2)$ sugere uma generalização óbvia, de substituir $\mathbb{R}^2$ 
	por outros espaços topológicos, como superfícies, grafos ou variedades. 
	Da mesma forma, o conjunto de (classes de equivalência de) \textit{loops} em um 
	espaço de configuração de $n$ partículas em um espaço topológico $X$ é chamado de 
	\textit{grupo de trança associado a $X$}.
	
	\par\vspace{0.3cm} Em símbolos, escrevemos $\pi_1(UC_n(X)) = B_n(X)$ e $\pi_1(C_n(X)) = P_n(X)$. 
	Desse modo, os grupos usuais $B_n$ e $P_n$ são, na verdade, $B_n(\mathbb{R}^2)$ e $P_n(\mathbb{R}^2)$.
	
	\par\vspace{0.3cm} Por exemplo, sendo $X = \mathbb{R}$ e tomando $n=2$, temos que
	%
	\begin{align*}
    	UC_2(\mathbb{R}) = \big\{  \{p_1,p_2\}\subset\mathbb{R} \ | \ p_1\neq p_2  \big\}, \\
    	C_2(\mathbb{R}) = \big\{ (p_1,p_2)\in\mathbb{R}^2 \ | \ p_1\neq p_2 \big\},
	\end{align*}
	%
	ou seja, os espaços de configuração não ordenado e ordenado, respectivamente, de dois 
	pontos em uma reta. Não é difícil perceber que tanto $C_2(\mathbb{R})$ quanto 
	$UC_2(\mathbb{R})$ possuem 3 componentes conexos e, em geral, $C_n(\mathbb{R})$ e 
	$UC_n(\mathbb{R})$ possuem $n+1$ componentes conexos. Também podemos tomar $X = \mathbb{S}^1$, 
	ou seja, uma circunferência. Nesse caso, $C_n(\mathbb{S}^1)$ e $UC_n(\mathbb{S}^1)$ 
	têm $n$ componentes conexos.
	
	\par\vspace{0.3cm} Agora, vamos considerar $X = \mathbb{S}^2$, a esfera em 
	$\mathbb{R}^3$. Nesse caso, $B_n(X) = B_n(\mathbb{S}^2)$ é chamado de 
	\textit{grupo de tranças esféricas}. Uma primeira pergunta natural seria: qual a 
	diferença entre tranças esféricas\index{Tranças esféricas} e ``tranças planares''? Bom, podemos imaginar as 
	paredes dos diagramas das tranças planares como sendo pequenos pedaços de esferas enormes, 
	por exemplo, como mostra a Figura \ref{diagrama tranca esferica} 
	abaixo (na verdade, essa ideia pode ser aplicada a tranças em qualquer superfície, não apenas na esfera).
	%
	\begin{figure}[H]
		\begin{center}
			\includegraphics[width=12cm]{Images/diagrama_tranca_esferica.png}
		\end{center}\caption{Diagrama de uma trança esférica}
		\label{diagrama tranca esferica}
	\end{figure}
	%
	Logo, temos de modo natural a função $f: B_n(\mathbb{R}^2)\to B_n(\mathbb{S}^2)$. 
	Não é muito difícil perceber que podemos obter toda trança esférica a partir de 
	uma trança planar, i.e., que $f$ é sobrejetora. Contudo, $f$ não é injetora.
	
	\par\vspace{0.3cm} Isso se deve ao fato de que, para tranças esféricas, as cordas 
	podem ser deformadas de modo a ``darem a volta'', como mostra a figura a seguir. 
	Podemos pensar em $B_n(\mathbb{R}^2)$ como tranças dentro de um cubo, ficando ``presas'' 
	lá dentro e, portanto, impossibilitadas de realizar movimentos como os que estão abaixo.
	%
	\begin{figure}[H]
		\begin{center}
			\includegraphics[width=10cm]{Images/movimentos_trancas_esfericas.png}
		\end{center}\caption{Movimentos possíveis para tranças esféricas, mas impossíveis para tranças planares}
		\label{movimentos trancas esfericas}
	\end{figure} 
	%
	Como os geradores $\sigma_i$ de $B_n(\mathbb{R}^2)$ também são geradores de $B_n(\mathbb{S}^2)$, 
	as relações das tranças planares continuam válidas para as tranças esféricas. 
	Contudo, os novos movimentos mostrados na Figura \ref{movimentos trancas esfericas} 
	nos dão uma nova relação em $B_n(\mathbb{S}^2)$, a saber
	%
	\begin{align}
	\label{novas relacoes}
	    \sigma_{i-1}\sigma_{i-2}
	    \cdots\sigma_2\sigma_1^2\sigma_2\sigma_3
	    \cdots\sigma_{n-2}\sigma_{n-1}^2\sigma_{n-2}\sigma_{n-3}
	    \cdots\sigma_i = 1.
	\end{align}
	%
	Contudo, toda relação, para $i=1,2\dots,n-1$, em \eqref{novas relacoes} é consequência da 
	única relação
	%
	\begin{align}
	\label{nova relacao}
	    (\sigma_1\sigma_2\cdots\sigma_{n-1})(\sigma_{n-1}\sigma_{n-2}\cdots\sigma_1) = 1.
	\end{align}
	%
	De fato, essa relação é dada pelo mesmo movimento que o diagrama (b) da 
	Figura \ref{movimentos trancas esfericas}, mas com a corda $1$ indo para a direita, 
	como o diagrama abaixo.
	%
	\begin{center}
		\begin{tikzpicture}
		\braid[braid colour=black,strands=4,braid start={(0,0)}]
		{\sigma_1\sigma_2\sigma_3\sigma_3\sigma_2\sigma_1}
		%\node[font=\Huge] at (4.5,-1.0) {\(\neq\)};
		%\braid[strands=3,braid start={(5,0)}]
		%{\sigma_2\sigma_1}
		
		%gama em $B_4(\mathbb{S}^2)$
		\end{tikzpicture}
	\end{center}
	%
	Portanto, uma apresentação de $B_n(\mathbb{S}^2)$ tem os mesmos geradores e relações 
	que $B_n(\mathbb{R}^2)$, mas com a última relação em \eqref{nova relacao}. 
	De fato, essa é a apresentação completa de $B_n(\mathbb{S}^2)$, conforme o teorema abaixo, 
	que não será demonstrado.
	%
	\begin{theorem}
	\label{apresentacao de B_n(S^2)}
		O grupo de tranças esféricas $B_n(\mathbb{S}^2)$ tem apresentação
		%
		\begin{align*}
    		B_n(\mathbb{S}^2) 
    		= \langle \sigma_1, \sigma_2, \dots, \sigma_{n-1}|&\sigma_i\sigma_j 
    		= \sigma_j\sigma_i, \text{ para } |i - j|>1, \\ 
    		&\sigma_i\sigma_{i+1}\sigma_i 
    		= \sigma_{i+1}\sigma_i\sigma_{i+1}, \text{ para } 1\leq i\leq n-2 , \\
    		&\sigma_1\sigma_2\cdots\sigma_{n-2}\sigma_{n-1}^2\sigma_{n-2}\cdots\sigma_2\sigma_1 = 1 \rangle.
		\end{align*}
		%
	\end{theorem}  
	%
	Então, voltando à nossa função $f$: ela não é injetora, pois a trança em \eqref{nova relacao} 
	pertence ao núcleo de $f$, ou seja, $\Ker f$ não é trivial. Outra trança que pertence ao núcleo 
	de $f$ é $\displaystyle{(\Delta_n^2)^2}$, o quadrado da volta completa, que tem ordem 
	$2$ em $B_n(\mathbb{S}^2)$.
	
	\par\vspace{0.3cm} Na verdade, esse fato pode ser generalizado no seguinte lema.
	%
	\begin{lemma}
	\label{potencia da volta completa trivial}
		Para todo $n> 2$, a trança de Dirac 
		$\delta = (\sigma_1\sigma_2\cdots\sigma_{n-1})^{kn} = (\Delta_n^2)^k$ 
		é trivial em $B_n(\mathbb{S}^2)$ se, e só se, $k$ é par. 
	\end{lemma}
	%
	\begin{proof}
		Já sabemos que $(\Delta_n^2)^2 = 1$ em $B_n(\mathbb{S}^2)$. Então, se $k$ é par, 
		podemos escrever $k=2j$ e segue que $(\Delta_n^2)^k = [(\Delta_n^2)^2]^j = 1^j = 1$. 
		Por outro lado, como $|\Delta_n^2| = 2$ em $B_n(\mathbb{S}^2)$, então $(\Delta_n^2)^k = 1$ 
		implica que $2\mid k$, i.e, $k$ par.
	\end{proof}
	%
	Do Teorema \ref{apresentacao de B_n(S^2)}, podemos obter alguns fatos interessantes sobre 
	o grupo de tranças esféricas. Por exemplo, 
	$B_2(\mathbb{S}^2) = \langle \sigma_1 \ | \ \sigma_1^2=1 \rangle$, ou seja, $B_2(\mathbb{S}^2)$ 
	é um grupo finito de ordem $2$ com um gerador: $\mathbb{Z}_2$. Além disso, também podemos definir 
	o homomorfismo de comprimento que definimos para as tranças planares, mas com uma pequena alteração.
	%
	\begin{prop}
	\label{homomorfismo de comprimento em trancas esfericas}
		Seja $\beta\in B_n(\mathbb{S}^2)$. 
		Tomando $\beta = \sigma_{i_1}^{\varepsilon_1}\cdots\sigma_{i_k}^{\varepsilon_k}$, 
		sendo $\varepsilon_i = \pm1$ para $i=1,2,\dots,k$, então podemos definir a função $l$, 
		chamada \textit{homomorfismo de comprimento}, de $B_n(\mathbb{S}^2)$ em $\mathbb{Z}$ como	
		%
		\begin{align*}
		    l(\beta) = \sum_{i=1}^{k}\varepsilon_i\text{ }\mathrm{mod}(2(n-1)),
		\end{align*}
		%
		ou seja, a soma dos expoentes módulo $2(n-1)$. Então, $l$ é invariante 
		em $B_n(\mathbb{S}^2)$, i.e., se $\beta = \beta'$ em $B_n(\mathbb{S}^2)$, 
		então $l(\beta) = l(\beta')\text{ }\mathrm{mod}(2(n-1))$.
	\end{prop}
	%
	\begin{proof}
		Do Teorema \ref{apresentacao de B_n(S^2)}, sabemos todas as relações de $B_n(\mathbb{S}^2)$. 
		Para as duas primeiras, $l(\beta)$ é constante para cada uma dessas relações. 
		Contudo, para a terceira relação, $l(\beta)$ difere por um fator de $\pm2(n-1)$ para os 
		dois lados da relação. Como $l(1) = 0$ e $l$ está bem definida (demonstração análoga à 
		do Lema \ref{homomorfismo de comprimento}), devemos ter a igualdade módulo $2(n-1)$.
	\end{proof}
	%
	\begin{example}
    	Como $l((\sigma_1\sigma_2)^3) = 6\neq 0\text{ }\mathrm{mod}4$ e $l(1) = 0$, 
    	então, em $B_3(\mathbb{S}^2)$, $(\sigma_1\sigma_2)^3\neq 1$.
    \end{example}
    %
    \begin{example}
    	Temos $\sigma_1^4=1$ em $B_3(\mathbb{S}^2)$, mas $\sigma_1^4\neq1$ em $B_4(\mathbb{S}^2)$. 
    	Esse exemplo nos mostra uma diferença notável entre $B_n(\mathbb{R}^2)$ e $B_n(\mathbb{S}^2)$, 
    	pois se $\beta$ é trivial em $B_m(\mathbb{R}^2)$ e $m\leq n$, então $\beta$ também 
    	é trivial em $B_n(\mathbb{R}^2)$, enquanto que para $B_m(\mathbb{S}^2)$ e $B_n(\mathbb{S}^2)$ 
    	isso, em geral, não é verdade.
	\end{example}
	%
	Note também que a recíproca da Proposição \ref{homomorfismo de comprimento em trancas esfericas} 
	é falsa. Por exemplo, $\sigma_1^6\neq1$ em $B_4(\mathbb{S}^2)$ mas 
	$l((\sigma_1)^6) = 0\text{ }\mathrm{mod}6 = l(1)$.
	
	\par\vspace{0.3cm} A apresentação no Teorema \ref{apresentacao de B_n(S^2)} tem também 
	uma interessante consequência, enunciada no Lema \ref{B_3(S^2) tem ordem 12}. 
	Antes, contudo, introduziremos um pequeno conceito, as \textit{transformações de Tietze}.
	%
	\subsubsection{Transformações de Tietze}
	\index{Transformações de Tietze}
	Dada uma apresentação de um grupo $G$, temos os seguintes movimentos:
	%
	\begin{enumerate}
		\item se uma relação pode ser deduzida a partir das relações existentes, 
		então podemos adicionar essa nova relação à apresentação. Por exemplo, 
		se $G = \langle x \ | \ x^3=1 \rangle$, obtemos a relação $x^6 = 1$ a partir de $x^3=1$. 
		Logo, podemos escrever $G = \langle x \ | \ x^3=1,x^6=1 \rangle$.
		
		\item Reciprocamente, se uma relação da apresentação pode ser deduzida a partir das outras, 
		então podemos removê-la. Em $G = \langle x \ | \ x^3=1,x^6=1 \rangle$, a relação $x^6=1$ pode ser 
		deduzida de $x^3=1$ e, portanto, pode ser removida da apresentação. Contudo, note que 
		não podemos remover $x^3=1$, pois teríamos outro grupo.
		
		\item Podemos adicionar um gerador escrito como uma palavra nos geradores originais. 
		Começando com $G = \langle x \ | \ x^3=1 \rangle$ e fazendo $y=x^2$, a nova apresentação 
		$G = \langle x,y \ | \ x^3=1,y=x^2 \rangle$ define o mesmo grupo.
		
		\item De modo semelhante, se podemos formar uma relação em que um dos geradores é uma 
		palavra nos outros geradores, então esse gerador pode ser removido. Por exemplo, a 
		apresentação do grupo abeliano de ordem $4$, 
		$G = \langle x,y,z \ | \ x=yz, y^2=1, z^2=1, x=x^{-1} \rangle$ 
		pode ser substituído por $G = \langle y,z \ | \ y^2=1,z^2=1,(yz)=(yz)^{-1} \rangle$.
	\end{enumerate}
	%
	Usaremos essas transformações para demonstrar o seguinte lema.
	%
	\begin{lemma}
	\label{B_3(S^2) tem ordem 12}
		$B_3(\mathbb{S}^2)$ tem apresentação $\langle a,b \ | \ b^6=1,a^2=b^3=(ab)^2 \rangle$, 
		sendo, portanto, isomorfo a $Q_6$, o grupo dicíclico de ordem $12$. 
	\end{lemma} 
	%
	\begin{proof}
		Sabemos, do Teorema \ref{apresentacao de B_n(S^2)}, que
		%
		\[
		B_3(\mathbb{S}^2) 
		= \langle \sigma_1,\sigma_2 \ | \ \sigma_1\sigma_2\sigma_1 
		= \sigma_2\sigma_1\sigma_2\text{, } \sigma_1\sigma_2^2\sigma_1 
		= 1\rangle.
		\]
		%
		Fazendo $a = \sigma_1\sigma_2\sigma_1$ e $b = \sigma_1\sigma_2$, temos:
		%
		\begin{align*}
    		a^2 
    		= \sigma_1\sigma_2\sigma_1\sigma_1\sigma_2\sigma_1 
    		= \sigma_1\sigma_2\sigma_1\sigma_2\sigma_1\sigma_2 
    		= (\sigma_1\sigma_2)^3 
    		= b^3, \\
    		(ab)^2 
    		= \sigma_1\sigma_2\sigma_1\sigma_1\sigma_2\sigma_1\sigma_2\sigma_1\sigma_1\sigma_2 
    		= \sigma_1\sigma_2\sigma_1\sigma_2
    		\underbrace{\sigma_1\sigma_2\sigma_2\sigma_1}_{1}\sigma_1\sigma_2 
    		= (\sigma_1\sigma_2)^3 
    		= b^3, \\
    		b^6 
    		= \sigma_1\sigma_2\sigma_1\sigma_2\sigma_1\sigma_2\sigma_1\sigma_2\sigma_1\sigma_2 
    		= \underbrace{\sigma_1\sigma_2\sigma_2\sigma_1}_{1}\sigma_2\sigma_2
    		\underbrace{\sigma_1\sigma_2\sigma_2\sigma_1}_{1}\sigma_2\sigma_2 
    		= \sigma_2^4.
		\end{align*}
		%
		Vamos mostrar que tanto $\sigma_1$ quanto $\sigma_2$ têm ordem 4. Então, 
		seja $N$ o fecho normal de $a^2$, i.e., $N = \{1,a\}$. 
		Daí, $N\vartriangleleft B_3(\mathbb{S}^2)$ e, portanto, o grupo quociente 
		$B_3(\mathbb{S}^2)/N$ tem apresentação $\langle a,b|a^2=b^3=(ab)^2=1 \rangle$, 
		que é a apresentação de $S_3$, o grupo simétrico em $3$ elementos. Logo, 
		$|B_3(\mathbb{S}^2)| = |N|\cdot|S_3| = 12$.
		
		\par\vspace{0.3cm} Por fim, note que $(ab)^2 = a^2$ implica $a = b^{-1}ab^{-1}$, logo
		%
		\begin{align*}
		    \sigma_1^4 = (b^{-1}a)^4 = (b^{-1}ab^{-1})a(b^{-1}ab^{-1})a = a^4 = 1.
		\end{align*}  
		%
		Observe o diagrama abaixo.
		%
		\begin{figure}[H]
			\begin{center}
				\includegraphics[width=12cm]{Images/manipulacao_de_a4.png}
			\end{center}\caption{$a^4 = 1$ em $B_3(\mathbb{S}^2)$}
		\end{figure}
		%
		Como $\sigma_1\neq 1$ e $\sigma_1^2\neq 1$, então, pela 
		Proposição \ref{homomorfismo de comprimento em trancas esfericas}, $|\sigma_1| = 4$.
		
		\par\vspace{0.3cm} Similarmente, $(ab)^2 = a^2$ implica $bab=a$ e também $ba=a^{-1}b^2$, logo
		%
		\begin{align*}
		    \sigma_2^4 = (a^{-1}b^2)^4 = (ba)^4 = (bab)a(bab)a = a^4 = 1.
		\end{align*}
		%
		Como $\sigma_2\neq1$ e $\sigma_2^2\neq1$ então, da 
		Proposição \ref{homomorfismo de comprimento em trancas esfericas}, temos $|\sigma_2|=4$.
		Além disso, como $a^4 = 1$, temos $b^6=1$.
		Por fim, da Tabela \ref{tabela grupos}, vemos que $B_3(\mathbb{S}^2)\cong Q_6$, 
		o grupo dicíclico de ordem 12.
	\end{proof}
	%
	Do Lema \ref{B_3(S^2) tem ordem 12}, podemos listar os elementos de $B_3(\mathbb{S}^2)$:
	%
	\begin{align*}
	    B_3(\mathbb{S}^2) = \left\{ 1, a, a^2, a^3, b, b^2, b^4, b^5, ab, ab^2, ab^4, ab^5 \right\}.
	\end{align*}
	%
	Podemos ainda escrever esse elementos em termos dos geradores $\sigma_1$ e $\sigma_2$. 
	Com algumas simplificações, obtemos:
	%
	\begin{align*}
    	B_3(\mathbb{S}^2) 
    	= \left\{ 1, \sigma_1\sigma_2\sigma_1, (\sigma_1\sigma_2\sigma_1)^2, 
    	(\sigma_1\sigma_2\sigma_1)^3, \sigma_1\sigma_2, (\sigma_1\sigma_2)^2, \sigma_1^3\sigma_2,
    	\sigma_2\sigma_1, \sigma_1, \sigma_2^3, \sigma_2\sigma_1^3\sigma_2, \sigma_1^3 \right \}.
	\end{align*}
	%
	Fazendo os diagramas, vemos que apenas $1$ e $(\sigma_1\sigma_2\sigma_1)^2 = a^2 = b^3$ 
	são tranças puras. Além disso, como $a^4 = 1 = b^6$, concluímos que 
	$P_3(\mathbb{S}^2) = \langle a^2 \rangle = \langle b^3 \rangle$.
	
	\par\vspace{0.3cm} Agora, vamos considerar $B_4(\mathbb{S}^2)$, que tem apresentação
	%
	\begin{align*}
	    B_4(\mathbb{S}^2) 
	    = \langle \sigma_1, \sigma_2, \sigma_3| &\sigma_1\sigma_2\sigma_1 
	    = \sigma_2\sigma_1\sigma_2 \\ 
	    &\sigma_2\sigma_3\sigma_2 
	    = \sigma_3\sigma_2\sigma_3 \\
    	&\sigma_1\sigma_3 = \sigma_3\sigma_1 \\
    	&\sigma_1\sigma_2\sigma_3^2\sigma_2\sigma_1 = 1\rangle.
	\end{align*}
	%
	Seja $N$ o fecho normal de $\sigma_1\sigma_3^{-1}$. Então, a apresentação do grupo quociente 
	$G = B_4(\mathbb{S}^2)/N$, é obtida de $B_4(\mathbb{S}^2)$ adicionando a relação 
	$\sigma_1\sigma_3^{-1} = 1$ ou, equivalentemente, $\sigma_1=\sigma_3$. 
	O efeito dessa relação extra é reduzir o número de relações para duas, a saber
	%
	\begin{center}
		\begin{tabular}{ccc}
			$\sigma_1\sigma_2\sigma_1 = \sigma_2\sigma_1\sigma_2$ & e &
			$\sigma_1\sigma_2\sigma_1^2\sigma_2\sigma_1 = 1$.
		\end{tabular}
	\end{center}
	%
	A última relação pode ser manipulada como segue:
	%
	\begin{align*}
    	1 &= \sigma_1\sigma_2\sigma_1\sigma_1\sigma_2\sigma_1 \\
    	&= \sigma_1\sigma_2\sigma_1\sigma_2\sigma_1\sigma_2 \\ 
    	&= (\sigma_1\sigma_2)^3
	\end{align*}
	%
	Portanto, 
	$G = \langle \sigma_1, \sigma_2 \ | \ \sigma_1\sigma_2\sigma_1 = \sigma_2\sigma_1\sigma_2\text{, }
	(\sigma_1\sigma_2)^3 = 1 \rangle$. Como antes, tome 
	$a = \sigma_1\sigma_2\sigma_1$ e $b = \sigma_1\sigma_2$. Como $\sigma_1 = b^{-1}a$ 
	e $\sigma_2 = a^{-1}b^2$, as relações de $G$ se tornam $a^2 = b^3$ e $b^3 = 1$, logo 
	$G = \langle a,b \ | \ a^2=b^3=1 \rangle$. 
	Vamos mostrar que $G$ é infinito, usando o seguinte lema, que será aceito sem demonstração.
	%
	\begin{lemma}
	\label{grupo triangular}
	\index{Grupo de Dyck}
		O grupo triangular, ou grupo de Dyck, $T(l,m,n) = \langle a,b \ | \ a^l=b^m=(ab)^n=1 \rangle$ 
		é finito se, e só se, $\displaystyle{\frac{1}{l} + \frac{1}{m} + \frac{1}{n} - 1 > 0}$.
	\end{lemma}
	%
	\begin{prop}
	\label{quociente de B4(S2) infinito}
		$G = \langle a,b \ | \ a^2=b^3=1 \rangle$ é infinito e, consequentemente, 
		$B_4(\mathbb{S}^2)$ é infinito.
	\end{prop}
	%
	\begin{proof}
		Seja $\widehat{G} = \langle a,b|a^2=b^3=(ab)^7=1 \rangle$. Note que $\widehat{G}$ 
		é um grupo quociente de $G$ e, além disso, $\widehat{G} = T(2,3,7)$. 
		Do Lema \ref{grupo triangular}, temos $\widehat{G}$ infinito, pois 
		$\displaystyle{\frac{1}{2} + \frac{1}{3} + \frac{1}{7} - 1 < 0}$. Logo, como 
		$\widehat{G}$ é um grupo quociente de $G$, então $G$ também é infinito, pois se todo 
		grupo finito com apresentação finita tem, no máximo, tantos geradores quanto relações. 
		Por fim, como $G$ é um grupo quociente de $B_4(\mathbb{S}^2)$, temos $B_4(\mathbb{S}^2)$ 
		infinito, também pelo mesmo fato citado acima sobre grupos finitos e finitamente apresentados.
		Observe que o número $7$ não tem nada de especial: poderíamos tomar qualquer $k\geq 7$.
	\end{proof}
	%
	Podemos, ainda, generalizar esse fato no seguinte lema, que não será demonstrado.
	%
	\begin{lemma}
	\label{grupo de trancas esfericas infinitos}
		$|B_n(\mathbb{S}^2)| = \infty$ para todo $n\geq 4$.
	\end{lemma}
	%
	Observando o Lema \ref{grupo de trancas esfericas infinitos}, a distinção mais simples 
	que podemos fazer entre $B_n(\mathbb{R}^2)$ e $B_n(\mathbb{S}^2)$ é que 
	$B_n(\mathbb{R}^2)$ é finito apenas para $n=1$, enquanto que $B_n(\mathbb{S}^2)$ 
	é finito para $n=1,2$ e $3$.
	
	\par\vspace{0.3cm} Contudo, a maior diferença entre esses dois grupos 
	(ou famílias de grupos) é o fato de que $B_n(\mathbb{R}^2)<B_{n+1}(\mathbb{R}^2)$, 
	como mostrado na Proposição \ref{B_m subgrupo de B_n}, enquanto que 
	$B_n(\mathbb{S}^2)\nless B_{n+1}(\mathbb{S}^2)$.
	
	\par\vspace{0.3cm} De fato, a relação 
	$(\sigma_1\sigma_2\cdots\sigma_{n-1})(\sigma_{n-1}\cdots\sigma_2\sigma_1) = 1$ 
	em $B_n(\mathbb{S}^2)$ pode não ser válida em $B_{n+1}(\mathbb{S}^2)$. Por exemplo, do 
	Teorema \ref{apresentacao de B_n(S^2)}, sabemos que $\sigma_1^2 = 1$ em 
	$B_2(\mathbb{S}^2)$, mas não em $B_3(\mathbb{S}^2)$, devido à 
	Proposição \ref{homomorfismo de comprimento em trancas esfericas}.
	
	\par\vspace{0.3cm} Além disso, também do Teorema \ref{apresentacao de B_n(S^2)},
	$\sigma_1\sigma_2^2\sigma_1 = 1$ em $B_3(\mathbb{S}^2)$, mas não em $B_4(\mathbb{S}^2)$, 
	de novo devido à Proposição \ref{homomorfismo de comprimento em trancas esfericas}.
	
	\par\vspace{0.3cm} Consequentemente, a função identidade 
	$\displaystyle{\psi: \underset{\sigma_i\mapsto\sigma_i}{B_n(\mathbb{S}^2)
	\to B_{n+1}(\mathbb{S}^2)}}$, para $1\leq i\leq n-1$, não é um homomorfismo e 
	não podemos considerar $B_n(\mathbb{S}^2)$ como um subgrupo natural de $B_{n+1}(\mathbb{S}^2)$.
	
	\par\vspace{0.3cm} É interessante notar também que algumas tranças não identidades são 
	triviais em $B_n(\mathbb{S}^2)$, i.e., $B_n(\mathbb{S}^2)$ \textbf{não é} livre de torção, 
	como mostrado no seguinte lema.
	%
	\begin{lemma}
	\label{B_(S^2) nao livre de torcao}
		Para todo $n\geq2$, $\gamma = \sigma_1\sigma_2\cdots\sigma_{n-1}$ 
		(a raiz $n$-ésima da volta completa) tem ordem finita maior que $1$ e, portanto, 
		$B_n(\mathbb{S}^2)$ tem elementos de torção.
	\end{lemma}
	%
	\begin{proof}
		Primeiro, note que como $l(\gamma) = (n-1)\neq0\text{ }\mathrm{mod}(2(n-1))$,
		então, para todo $n\geq2$, $\gamma\neq1$. Por outro lado, sabemos que o quadrado 
		da volta completa é trivial, i.e., $(\Delta_n^2)^2 = 1$, logo, como $\Delta_n^2 = \gamma^n$, 
		temos $\gamma^{2n} = 1$.
		
		\par\vspace{0.3cm} Portanto, $\gamma$ tem ordem finita $k$ com $2\leq k\leq 2n$, para 
		todo $n\geq 2$. Consequentemente, $B_n(\mathbb{S}^2)$ tem elemento não trivial de ordem 
		finita, não sendo, portanto, livre de torção.
	\end{proof}
	%
	\section{Diagrama de van Kampen}
	\index{Diagrama de van Kampen}
	Vamos nos deter brevemente para descrever o \textbf{diagrama de van Kampen}. 
	Inicialmente, podemos descrever o diagrama de van Kampen de modo visual: ele ilustra o fato de que uma
	palavra $w\in F(X)$ no grupo livre sobre $X$ é uma relação em um grupo $G$, i.e., $w$ é um produto de
	palavras em $R\cup R^{-1}$. Em um nível mais rigoroso, os diagrama de van Kampen são a base de uma das
	técnicas mais poderosas da Teoria Combinatória dos Grupos.
	
	\par\vspace{0.3cm} Informalmente, o diagrama de van Kampen para uma apresentação 
	$G = \langle X \ | \ R \rangle$ é um grafo conexo finito planar $\Gamma\subseteq\mathbb{R}^2$, 
	cujas arestas são direcionadas e marcadas por elementos de $X$ em um caminho tal que toda face 
	de $\Gamma$ é um disco cuja fronteira é marcada e pertence a $R$ (ou seja, é uma relação). 
	Daí, é quase imediato que a palavra marcada sobre o bordo de $\Gamma$ é ela própria uma relação em $G$.
	Assim, o diagrama de van Kampen pode ser utilizado para ilustrar a dedução de novas relações a 
	partir das antigas relações.
	
	\par\vspace{0.3cm} Antes dos exemplos, vejamos como que cada relator de $G$ é uma palavra limitando 
	algum diagrama de van Kampen $\Gamma$ de $G$. Podemos, ainda, assumir que $\Gamma$ está reduzido, 
	no sentido de que nenhum circuito não trivial carrega a palavra vazia. Agora, o fato de que 
	$\Gamma$ está imerso no plano impõe várias restrições sobre sua estrutura, e.g., sobre a 
	característica de Euler. Essas restrições podem ser usadas para argumentar sobre propriedades 
	locais de $\Gamma$ (correspondendo a condições combinatoriais no conjunto $R$) e sobre 
	propriedades do bordo (correspondendo a propriedades grupo-teóricas de $G$).
	
	\par\vspace{0.3cm} Um modo de construir o diagrama de van Kampen para uma apresentação 
	$G = \langle X \ | \ R \rangle$ é o seguinte. Pensemos em cada relator como o bordo de uma 
	célula bidimensional. Podemos, então, colar coleções dessas células ao longo de arestas com 
	a mesma palavra e orientação.
	
	\par\vspace{0.3cm} Por exemplo, o grupo dos quatérnios, $Q_8$, tem ordem $8$ e apresentação:
	%
	\begin{equation*}
		\langle x,y \ | \ x^4=1, x^2=y^2, y^{-1}xy = x^{-1} \rangle.
	\end{equation*}
	%
	Vamos fazer $r = x^4$, $s = x^2y^{-2}$ e $t = y^{-1}xyx$. Daí, o diagrama de van Kampen é o seguinte.
	%
	\begin{figure}[H]
		\begin{center}
			\includegraphics[width=5cm]{Images/diagrama_quaternios.png}
		\end{center}
	\caption{Diagrama de van Kampen para o grupo dos quatérnios, $Q_8$}
	\label{figura diagrama quaternios}
	\end{figure}
	%
	Na parte superior da face esquerda, a fronteira, lida no sentido horário a partir de $A$ é 
	$s = x^2y^{-2}$. De modo análogo, a face contendo $B$, lida no sentido anti-horário a partir 
	de $B$ também nos dá $s = x^2y^{-2}$. Na face inferior contendo $B$, a delimitação do bordo 
	lida no sentido horário a partir de $A$ nos dá $t = y^{-1}xyx$. Do mesmo modo, a face mais à direita, 
	lida no sentido horário a partir de $C$ também nos dá $t = y^{-1}xyx$. Por fim, como a marca sobre a
	fronteira, lida no sentido horário a partir de $A$, é $x^4=1$, isso nos mostra que essa primeira 
	relação na apresentação de $Q_8$ é supérflua.
	%
	\par\vspace{0.3cm} Outro exemplo é o grupo 
	%
	\begin{equation*}
	    G = \langle a,b,c,d \ | \ ab = c, bc = d, cd = a, da = b \rangle.
	\end{equation*}
	%
	Ele é claramente gerado por $a$ e $b$, uma vez que $c = ab$ e $d = bc = bab$. Contudo, 
	esse fato pode ser verificado a partir do diagrama de van Kampen para essa apresentação. 
	Nele, as faces estão marcadas pelas seguintes relações definidoras:
	%
	\begin{equation*}
	    r_1 = abc^{-1}, \ r_2 = bcd^{-1}, \ r_3 = cda^{-1}, \ r_4 = dab^{-1}.
	\end{equation*}
	%
	\begin{figure}[H]
	\begin{center}
		\includegraphics[width=7.5cm]{Images/diagrama_ciclico.png}
	\end{center}
	\caption{Diagrama de van Kampen de $G$}
	\label{figura diagrama ciclico}
	\end{figure}
	%
    A partir do diagrama, vemos que lendo a fronteira, no sentido anti-horário, a partir do vértice 
    superior direito, temos $ab^3 = 1$, ou seja, $a = b^{-3}$ e, portanto, $G$ é gerado apenas por 
    $b$, sendo cíclico. De fato, podemos fazer uma manipulação desse diagrama para obter o seguinte 
    diagrama:
	%
	\begin{figure}[H]
		\begin{center}
			\includegraphics[width=7.5cm]{Images/diagrama_ciclico_2.png}
		\end{center}
	\caption{Diagrama de van Kampen manipulado de $G$}
	\label{figura diagrama ciclico 2}
	\end{figure}
	%
	A partir desse diagrama, vemos, a partir da leitura do bordo, que $b^5=1$. Logo, $G$ é o 
	grupo cíclico de ordem $5$ ($b$ não trivial) ou $1$ ($b$ trivial).
	
	\par\vspace{0.3cm} É interessante notar que se $a$, $b$, $c$ e $d$ são geradores de um grupo 
	tais que $a$ e $b$ comutam com $c$ e $d$, então o fato de que $ab$ comuta com $cd$ pode ser 
	deduzido pelo seguinte diagrama:
	%
	\begin{figure}[H]
	\begin{center}
		\includegraphics[width=5cm]{Images/diagrama_comutatividade.png}
	\end{center}
	\caption{Diagrama de van Kampen em que $a$ e $b$ comutam com $c$ e $d$}
	\label{figura diagrama comutatividade}
	\end{figure}
	%
	Do diagrama, é imediato que $abcd(ab)^{-1}(cd)^{-1} = 1$, ou seja, $[ab,cd] = 1$.
	
	\par\vspace{0.3cm} Um último exemplo é o seguinte: fazendo $r = x^2yxy^3 = 1 = y^2xyx^3 = s$, 
	obtemos o seguinte diagrama:
	%
	\begin{figure}[H]
	\begin{center}
		\includegraphics[width=7.5cm]{Images/diagrama_complicado.png}
	\end{center}
	\end{figure}
	%
	Da leitura do bordo no sentido anti-horário, obtemos $x^7=1$.
	
	\par\vspace{0.3cm} Dos exemplos acima, podemos ver que os diagramas de van Kampen nos ajudam 
	a deduzir relações que não são tão imediatas tendo apenas a apresentação do grupo.
	
	\section{Tranças como discos perfurados}\label{secao trancas como discos perfurados}
	Podemos, ainda, pensar em tranças de uma terceira maneira. Voltando na trança da 
	Figura \ref{tranca espaco configuracao}, vamos imaginar o seguinte: suponha que a 
	trança é feita de fios rígidos, e que o plano é, na verdade, uma seção quadrada do plano 
	(como está desenhado), ou seja, um disco topológico, exceto que esse disco tem buracos.
	
	\par\vspace{0.3cm} Agora, imagine que o interior do disco é feito de um material muito flexível 
	e que a fronteira (bordo) quadrada é uma armação rígida (parecida com aqueles \textit{frisbees} 
	de tecido, mas agora quadrados). Imagine o processo de empurrar esse disco perfurado ao longo da 
	trança de fios rígidos até chegar na parede da direita. A trança não se moveu, mas o disco em si 
	foi todo torcido, ou seja, a trança provocou uma mudança na superfície.
	
	\par\vspace{0.3cm} Mais especificamente, o que aconteceu ao disco perfurado foi que a trança 
	implementou um homeomorfismo da superfície nela mesma. O bordo fica fixo, porque é rígido, e 
	as perfurações são permutadas conforme a permutação associada à trança. Funções como essa, a 
	menos de homotopia, formam o \textit{grupo de classes} de $D_n$, um disco com $n$ perfurações 
	com bordo $\partial D_n$ (e não o grupo diedral):
	%
	\begin{align*}
    	\Mod(D_n) = \{ f:D_n\to D_n|& f\text{ é homeomorfismo}, \\
    	&f|_{\partial D_n} = 1 \}/\text{Homotopia}.
	\end{align*}
	%
	\begin{figure}[H]
	\begin{center}
		\includegraphics[width=10cm]{Images/tranca_disco_perfurado.png}
	\end{center}\caption{Diagrama de curva associado a $\sigma_1$.}
	\label{tranca disco perfurado}
	\end{figure}
	%
	A operação do grupo é a composição de funções: dois homeomorfismos $f$ e $g$ podem ser 
	compostos para formar um homeomorfismo $g\circ f$ (ou $f\circ g$, que geralmente é diferente). 
	Acima, descrevemos uma função $\psi: B_n\to\Mod(D_n)$, a saber, dada uma trança, deslize o 
	disco pela trança e considere o homeomorfismo resultante. De fato, $\psi$ é, na verdade, um isomorfismo.
	%
	\begin{theorem}
	\label{B_n isomorfo a Mod(D_n)}
		$B_n\cong \Mod(D_n)$ 
	\end{theorem}
	%
	\begin{proof}
		Não demonstraremos a sobrejetividade de $\psi$, pois exige conhecimentos além do escopo 
		deste texto.
		
		\par\vspace{0.3cm} Primeiro, note que $\psi$ está bem definida, pois se $\alpha, \beta\in B_n$ 
		são tais que $\alpha = \beta$, então elas induzem a mesma deformação no disco, i.e., 
		o mesmo homeomorfismo e, portanto, temos $\psi(\alpha) = \psi(\beta)$.
		
		\par\vspace{0.3cm} Além disso, o núcleo de $\psi$ é, por definição, formado pelas 
		tranças que induzem o homeomorfismo identidade, ou seja, as tranças que não fazem nada com 
		o interior do disco (o bordo fica fixo sempre). Ora, mas a única trança que não deforma o 
		interior do disco é a identidade e, portanto, $\Ker\psi = \{1\}$.
		
		\par\vspace{0.3cm} Agora, sejam $\alpha, \beta\in B_n$ duas tranças quaisquer tais que 
		$\psi(\alpha) = f$ e $\psi(\beta) = g$. Daí, temos
		%
		\begin{align*}
		    \psi(\alpha\beta) = g\circ f = \psi(\alpha)\circ\psi(\beta).
		\end{align*}
		%
		Essa última igualdade nos diz simplesmente que, fazendo o produto de duas tranças 
		$\alpha$ e $\beta$, a deformação resultante é equivalente a aplicar a deformação 
		da segunda trança ($\beta$) na deformação da primeira trança ($\alpha$), ou seja, 
		compor as duas deformações.
		
		\par\vspace{0.3cm} Portanto, como $\psi$ é um homomorfismo sobrejetor de núcleo 
		trivial, então $B_n\cong\Mod(D_n)$.
	\end{proof}
	%
	\subsection*{Diagramas de curvas}
	Vamos nos aprofundar um pouco nessa nova perspectiva dos grupos de tranças. 
	Olhe novamente a Figura \ref{tranca espaco configuracao}, mas foque agora nos discos quadrados 
	perfurados nas extremidades da imagem. Imagine que há uma linha vertical pontilhada no disco 
	da esquerda passando por todas as perfurações. A pergunta é: para onde irá a linha pontilhada 
	após deslizarmos o disco pela trança?
	
	\par\vspace{0.3cm} A Figura \ref{tranca disco perfurado} mostra um exemplo mais simples. 
	Nela, a trança é simplesmente $\sigma_1$, e tanto a linha pontilhada original quanto o 
	resultado torcido estão desenhados. A linha pontilhada na parede da direita é dita 
	\textit{diagrama de curva} induzido pela trança ($\sigma_1$, nesse caso).
	
	\par\vspace{0.3cm} Vamos chamar a linha pontilhada original de \textbf{eixo} do disco. 
	Ele consiste de $n+1$ segmentos pontilhados $a_0, \dots, a_n$, numerados em ordem de baixo 
	para cima. Então, em geral, o diagrama de curva associado a uma trança $\beta$ é a união 
	dos arcos $\psi_{\beta}(a_i)$ (os arcos que são as imagens dos $a_i$'s), i.e., você pensa 
	em $\beta$ como um homeomorfismo do disco e o diagrama de curva é para onde o eixo vai (a imagem do eixo).
	
	\par\vspace{0.3cm} Denotaremos os arcos do diagrama de curva por $c_i$, e o diagrama todo por $c$. 
	Note que, como $\beta$ fixa $\partial D_n$ (o bordo), os $c_i$'s (isto é, o diagrama de curva) 
	se encaixam ponta a ponta, em ordem, para formar o único arco $c$ que começa no centro inferior, 
	nunca se cruza, passa por cada perfuração uma única vez e termina no centro superior. 
	Veja novamente a Figura \ref{tranca disco perfurado}.
	
	\par\vspace{0.3cm} Diagramas de curva podem ser extremamente complicados, como poderíamos esperar 
	se tivermos uma trança longa. Por exemplo, o diagrama de curva da trança da 
	Figura \ref{tranca espaco configuracao} é bem complicado de desenhar.
	
	\par\vspace{0.3cm} Uma maneira preliminar de simplificar um diagrama de curva é ter certeza 
	de que ele está \textit{reduzido}, no seguinte sentido: dado um diagrama de curva em um disco 
	perfurado, desenhe o eixo no mesmo disco. O diagrama é dito \textit{reduzido} se não 
	existem \textit{biágonos} na figura, i.e., regiões cujas fronteiras consistem de um 
	sub-arco de um único $a_i$ e um sub-arco de um único $c_i$.
	
	\par\vspace{0.3cm} Podemos resumir o que foi dito no parágrafo anterior nas seguintes definições:
	%
	\begin{definition}[Biágono]
	\label{def biagono}
		Um biágono é um região cuja fronteira consiste de um sub-arco de um único $a_i$ e um 
		sub-arco de um único $c_i$.
	\end{definition}
	%
	\begin{definition}[Diagrama reduzido]
	\label{def reducao}
		Um diagrama de curva é dito reduzido quando não possui biágonos.
	\end{definition}
	%
	Os biágonos podem ser facilmente eliminados, de modo que todo diagrama de curva pode ser reduzido. 
	%
	\begin{figure}[H]
	\begin{center}
		\includegraphics[width=7cm]{Images/biagonos.png}
	\end{center}\caption{Esse diagrama possui 2 biágonos (em vermelho e em azul), e 
	pode ser reduzido ao diagrama da Figura \ref{tranca disco perfurado}.}
	\label{diagrama nao reduzido com biagonos}
	\end{figure}
	%
	É, então, natural considerar os diagramas das 
	Figuras \ref{tranca disco perfurado} e \ref{diagrama nao reduzido com biagonos} como equivalentes. 
	Esse fato nos leva à seguinte definição.
	%
	\begin{definition}[Diagramas equivalentes]
	\label{def equivalencia diagramas}
		Dois diagramas de curva são ditos equivalentes se têm a mesma forma reduzida.
	\end{definition} 
	%
	Vamos, agora, mostrar uma aplicação interessante dos diagramas de curva, a saber, na demonstração da Proposição \ref{geradores de B_n tem ordem infinita}.
	
	\par\vspace{0.3cm} Dada uma trança $\beta$, vamos olhar para o seu diagrama de curva $c$. 
	Comece na parte de baixo e observe o primeiro arco $c_i$ que não é igual ao arco correspondente $a_i$ 
	no eixo. Se $c_i$ está à direita de $a_i$, dizemos que $\beta$ é \textit{desviada à direita} e, 
	se $c_i$ está à esquerda de $a_i$, dizemos que $\beta$ é \textit{desviada à esquerda}. 
	Se $c_i = a_i$ para todo $i$, então $\beta$ é a trança identidade (tecnicamente, estamos usando o 
	Teorema \ref{B_n isomorfo a Mod(D_n)}). A trança da Figura \ref{tranca disco perfurado} é 
	desviada à esquerda, pois começando na parte inferior, o arco $c_0$ desvia para a esquerda.
	%
	\begin{prop}
	\label{desvio a direita desvio a esquerda}
		Se $\beta$ é desviada à esquerda, então $\beta^{-1}$ é desviada à direita.
	\end{prop}
	%
	\begin{proof}
		O efeito, no diagrama de curva, de inverter uma trança é simplesmente realizar uma 
		reflexão em torno de uma das arestas do disco paralelas ao eixo. Observe a 
		Figura \ref{diagrama inverso de sigma1}.
		%
		\begin{figure}[H]
		\begin{center}
			\includegraphics[width=5cm]{Images/inverso.png}
		\end{center}\caption{Diagrama de curva de $\sigma_1^{-1}$}\label{diagrama inverso de sigma1}
		\end{figure}
		%
		Portanto, como inverter uma trança equivale a refletir o seu diagrama de curva, segue que 
		se a trança original era desviada à esquerda, o seu inverso será desviado à direita e vice-versa.
	\end{proof}
	%
	Queremos mostrar que $B_n$ é livre de torção. Esse fato segue do seguinte lema.
	%
	\begin{lemma}
	\label{desviada a direita, sempre a direita}
		Se $\beta$ é desviada à direita, então $\beta^n$ é desviada à direita para todo $n>0$.
	\end{lemma}
	%
	\begin{proof}
		Suponha que $\beta$ é desviada à direita, e observe o diagrama de curva reduzido $c$ de $\beta$. 
		Encontre o primeiro $c_i$ que difere do $a_i$ correspondente, e chame esse arco de $c_k$. 
		Denotando por $c^2$ o diagrama de curva de $\beta^2$, e por $c_i^2$ os arcos de $c^2$, 
		imagine dois discos separados: um em que $a_i$ e $c_i$ estão desenhados e um em que 
		$c_i$ e $c_i^2$ estão desenhados. Observe que a segunda figura é obtida da primeira 
		aplicando a trança $\beta$ (pensada como um homeomorfismo). Então, como $c_k$ está à 
		direita de $a_k$, segue que $c_k^2$ está à direita de $c_k$. Além disso, como não há 
		biágonos na primeira figura, também não há biágonos na segunda figura.
		
		\par\vspace{0.3cm} Queremos mostrar que $\beta^2$ é desviada à direita, i.e, que $c_k^2$ 
		está à direita de $a_k$. Sabemos que $\beta$ não afeta os arcos $a_i$ para $i<k$, e 
		então $\beta^2$ também não. Também sabemos que, nas figuras, $c_k$ está à direita de 
		$a_k$ e $c_k^2$ está à direita de $c_k$.
		
		\par\vspace{0.3cm} O que falta observar é que o diagrama de $\beta^2$ pode não estar reduzido. 
		Sabemos que não há biágonos entre o eixo e $c$, e também que não há biágonos entre $c$ e $c^2$, 
		mas se $c^2$ e o eixo formarem um biágono, então $c^2$ precisa ser reduzido e pode acabar não 
		sendo desviado à direita.
		
		\par\vspace{0.3cm} Felizmente, isso não ocorre. Para ver que esse é o caso, seja $d$ o 
		segmento inicial de $c_k$ que vai até a primeira vez que $c_k$ intercepta o eixo, de forma 
		que $d$ está contido em uma metade (a metade da direita) do disco (é possível que $d$ seja 
		$c_k$ inteiro, mas em geral $c_k$ pode ``passear'' bastante antes de atingir uma perfuração). 
		Agora, a única maneira de $c_k^2$ estar à esquerda (ou ser igual) do arco $a_k$ do eixo é 
		formando um biágono com $a_k$. Mas isso forçaria $c_k^2$ a cruzar $d$, criando um biágono 
		entre $c$ e $c^2$, o que sabemos que não existe. Veja a figura abaixo.
		%
		\begin{figure}[H]
		\begin{center}
			\includegraphics[width=8cm]{Images/demosntracao_lema.png}
		\end{center}\caption{Ilustração da demonstração}
		\label{ilustracao da demonstracao}
		\end{figure}
		%
		Portanto, $\beta^2$ é, de fato, desviada à direita. Repetindo o argumento, mostramos que 
		$\beta^n$ é desviada à direita para todo $n>0$.
	\end{proof}
	%
	\begin{corollary}
	\label{B_n livre de torcao por diagramas de curva}
		$B_n$ é livre de torção.
	\end{corollary}
	%
	\begin{proof}
		Do Lema \ref{desviada a direita, sempre a direita}, sabemos que se $\beta$ é desviada à direita, 
		$\beta^n$ também o é para todo $n$ positivo. Da 
		Proposição \ref{desvio a direita desvio a esquerda}, sabemos que $\beta^{-1}$ é desviada 
		à esquerda e então, novamente do Lema \ref{desviada a direita, sempre a direita}, 
		$(\beta^{-1})^n$ também o é para todo $n$ positivo. Portanto, toda trança que é desviada 
		(seja à direita ou à esquerda) tem ordem infinita. Como, do 
		Teorema \ref{B_n isomorfo a Mod(D_n)}, a trança trivial é a única trança que não é desviada, 
		temos que $B_n$ é livre de torção.
	\end{proof}
	%
	\begin{remark}
		Na demonstração do Lema \ref{desviada a direita, sempre a direita}, alguns detalhes 
		foram omitidos. Por exemplo, homeomorfismos podem ser complicados: poderia ser o caso 
		de que o arco $c_0$ interceptasse $a_0$ infinitas vezes. Então, como reduzi-lo? 
		Esse e outros detalhes podem ser tratados de maneira rigorosa, mas isso está além do 
		escopo desse texto.
	\end{remark}
	%
	Uma última aplicação interessante dessa nova visualização dos grupos de trança é na 
	identificação de tranças (palavras) que são equivalentes à identidade, ou seja, o 
	problema da palavra (veja Seção \ref{secao o problema da palavra}). Usando o 
	Teorema \ref{B_n isomorfo a Mod(D_n)}, sabemos que uma trança $\beta$ é trivial se, 
	e só se, $\psi(\beta) = \text{Id}$, ou seja, se, e só se, $\beta$ induz o homeomorfismo 
	identidade (não faz nada com o disco). Portanto, dada uma trança $\beta$ qualquer, basta 
	observarmos o efeito de $\beta$ no disco perfurado: se provoca torção, não é trivial; se 
	não provoca torção, é trivial. Esse procedimento é complicado de computar para tranças longas, 
	mas continua sendo uma solução igualmente válida.