\chapter[Teoria de Grupos]{Teoria de Grupos}
\label{cap-1}
\chaptermark{}
%
\hfill%
\begin{minipage}{10cm}
\begin{flushright}
\rightskip=0.5cm
\textit{``Go down deep enough into anything and you will find mathematics.''}
\\[0.1cm]
\rightskip=0.5cm
--- Dean Schlicter
\end{flushright}
\end{minipage}

\section{Grupos}
%
\begin{definition}[Operação binária]
\label{def-operacao-binaria}
    Seja $G$ um conjunto. Uma operação binária em $G$ é uma função que associa, a cada par ordenado
    de elementos de $G$, um elemento de $G$.
\end{definition}
%
\begin{example}
    A adição, subtração e multiplicação de inteiros são exemplos de operações binárias, bem
    como a adição e multiplicação módulo $n$.
\end{example}
%
\begin{definition}[Grupo]
\label{def-grupo}
    Seja $G$ um conjunto munido de uma operação binária que, a cada par $(a,b)\in G\times G$, associa
    o elemento de $G$ denotado por $ab$. Dizemos que $G$ é um grupo sob essa operação se as seguintes
    condições são satisfeitas:
    %
    \begin{enumerate}[(i)]
        \item a operação é associativa, ou seja, $(ab)c = a(bc)$ para todos $a,b,c\in G$;
        \item existe um elemento $e\in G$, chamado identidade, tal que $ae = ea = a$ para todo $a\in G$;
        \item para todo $a\in G$, existe $b\in G$ (chamado o inverso de $a$) tal que $ab = ba = e$.
    \end{enumerate}
    %
\end{definition}
%
\begin{remark}
    É comum denotar a identidade simplesmente por $1$ ao invés de $e$. Faremos isso sem cerimônia
    no decorrer do texto onde for conveniente.
\end{remark}
%
\begin{example}
    Os conjuntos dos inteiros, dos racionais e dos reais são todos grupos sob a operação de adição.
    Em cada um dos casos, a identidade é o 0 e o inverso de $a$ é $-a$. Além disso, $a+b = b+a$ para
    todos $a,b$ no respectivo conjunto e, por isso, dizemos que o grupo é abeliano.
\end{example}
%
\begin{example}
    O conjunto dos inteiros sob a multiplicação ordinária não é um grupo. De fato, não existe
    nenhum inteiro $b$ tal que $5b = 1$.
\end{example}
%
\subsection*{Propriedades elementares de grupos}
%
\begin{theorem}
\label{teo-unicidade-ident}
    O elemento identidade de um grupo é único.
\end{theorem}
%
\begin{proof}
    Suponha que $e$ e $e'$ são identidades em $G$. Então, temos
    %
    \[
    e' = e'e = e
    \]
    %
    e, portanto, a identidade é única.
\end{proof}
%
\begin{theorem}
\label{teo-cancelamento}
    Dados um grupo $G$ e $a,b,c\in G$ quaisquer, temos que
    %
    \begin{enumerate}
        \item $ba = ca \implies b = c$;
        \item $ab = ac \implies b = c$.
    \end{enumerate}
    %
\end{theorem}
%
\begin{proof}
    Suponha $ba = ca$ e seja $a^{-1}$ o inverso de $a$. Multiplicando à direita, temos
    %
    \[
    (ba)a^{-1} = (ca)a^{-1} \iff b(aa^{-1}) = c(aa^{-1}) \iff b = c. 
    \]
    %
    De maneira análoga, mostra-se que $ab = ac$ implica $b=c$ multiplicando-se por $a^{-1}$ à esquerda.
\end{proof}
%
\begin{theorem}
\label{teo-unicidade-inv}
    Para cada $a\in G$, existe um único $b\in G$ tal que $ab = ba = e$.
\end{theorem}
%
\begin{proof}
    Suponha que $b$ e $c$ são inversos de $a$. Então $ab = e = ac$ e, portanto, $b = c$.
\end{proof}
%
\begin{theorem}
\label{teo-inv-prod}
    Dados um grupo $G$ e $a,b\in G$, temos $(ab)^{-1} = b^{-1}a^{-1}$.
\end{theorem}
%
\begin{proof}
    Note que
    %
    \[
    (ab)(ab)^{-1} = e
    \]
    %
    e
    %
    \[
    (ab)(b^{-1}a^{-1}) = a(bb^{-1})a^{-1} = aa^{-1} = e,
    \]
    %
    donde segue, pelo teorema anterior, que $(ab)^{-1} = b^{-1}a^{-1}$.
\end{proof}
%
\subsection*{Subgrupos}
%
\begin{definition}
\label{def-ordem-grupo}
    A ordem de um grupo finito $G$ é o número de elementos que ele contém, denotada por
    $|G|$ ou $\sharp G$.
\end{definition}
%
\begin{definition}
\label{def-ordem-elemento}
    A ordem de um elemento $g$ de um grupo $G$ é o menor inteiro positivo $n$ tal que
    $g^n = e$. Se um tal inteiro não existir, dizemos que $g$ tem ordem infinita. Denotamos
    a ordem de $g$ por $|g|$.
\end{definition}
%
\begin{example}
    Considere $\mathbb{Z}_{10}$ sob a adição módulo 10. Como $1\cdot 2 = 2$, $2\cdot 2 = 4$,
    $3\cdot 2 = 6$, $4\cdot 2 = 8$ e $5\cdot 2 = 0$, sabemos que $|2| = 5$.
\end{example}
%
\begin{example}
    Considere $\mathbb{Z}$ sob a adição. Todo elemento não nulo tem ordem infinita, já que a
    sequência $a, 2a, 3a, \dots$ não inclui o 0 se $a\neq 0$.
\end{example}
%
\begin{definition}
\label{def-subgrupo}
    Se um subconjunto $H$ de um grupo $G$ é também um grupo sob a operação de $G$,
    dizemos que $H$ é um subgrupo de $G$, e denotamos $H \leq G$ ou $H < G$ se $H\subseteq G$
    e $H \subset G$, respectivamente.
\end{definition}
%
\subsection*{Testes de subgrupos}
%
\begin{theorem}
\label{teo-teste-subgrupo}
    Sejam $G$ um grupo e $H\neq\varnothing$ um subconjunto de $G$. Se $ab^{-1}\in H$ sempre que
    $a,b\in H$, então $H\leq G$.
\end{theorem}
%
\begin{proof}
    Como a operação de $H$ é a mesma de $G$, é claro que ela é associativa. Vamos mostrar que
    $e\in H$.
    
    Como $H\neq\varnothing$, existe $x\in H$. Daí, tomando $a=x=b$, temos que $e = xx^{-1} = ab^{-1}\in H$.
    
    Para checar que $x^{-1}\in H$ sempre que $x\in H$, basta tomarmos $a = e$ e $b = x$. 
    
    Por fim, resta mostrarmos que $H$ é fechado para a operação, ou seja, que se $x,y\in H$
    então $xy\in H$. Ora, para tanto basta tomar $a = x$ e $b = y^{-1}$.
\end{proof}
%
\begin{example}
    Seja $G$ um grupo abeliano com identidade $e$. Então $H = \{x\in G \ | \ x^2 = e\}$ é um subgrupo
    de $G$. De fato, temos $e^2 = e$, logo $e\in H$ e, portanto, $H\neq\varnothing$. Agora, suponha
    que $a,b\in H$, ou seja, $a^2 = e = b^2$. Queremos mostrar que $(ab^{-1})^2 = e$. Ora, mas
    como $G$ é abeliano, então
    %
    \[
    (ab^{-1})^2 = ab^{-1}ab^{-1} = aab^{-1}b^{-1} = a^2(b^2)^{-1} = ee^{-1} = e.
    \]
    %
    Segue então do teorema anterior que $H\leq G$.
\end{example}
%
\begin{theorem}
\label{teo-teste2-subgrupo}
    Sejam $G$ um grupo e $H\neq\varnothing$ um subconjunto de $G$. Se $ab\in H$ sempre
    que $a,b\in H$ e $a^{-1}\in H$ sempre que $a\in H$, então $H\leq G$.
\end{theorem}
%
\begin{proof}
    Pelo Teorema \ref{teo-teste-subgrupo}, basta mostrar que se $a,b\in H$ então $ab^{-1} \in H$.
    Então, suponhamos $a,b\in H$. Por hipótese, $H$ contém os inversos de $a$ e de $b$, de modo que
    $ab^{-1}\in H$ pois $H$ é fechado para a multiplicação por hipótese.
\end{proof}
%
\begin{example}
    Sejam $G$ um grupo abeliano e $H = \{ x\in G \ | \ |x| < \infty \}$. Vamos mostrar que $H\leq G$
    usando o teorema acima. Primeiro, como $e^1 = e$, segue que $e\in H$, ou seja, $H\neq\varnothing$.
    Agora, assuma que $a,b\in H$ e sejam $|a| = m$ e $|b| = n$. Como $G$ é abeliano, temos que
    %
    \[
    (ab)^{mn} = (a^m)^n(b^n)^m = e^ne^m = e.
    \]
    %
    Portanto, $|ab| < \infty$ (não necessariamente $|ab| = mn$, cuidado!), ou seja, $ab\in H$. Por fim,
    como
    %
    \[
    (a^{-1})^m = (a^m)^{-1} = e^{-1} = e,
    \]
    %
    então $a^{-1}\in H$ e, pelo teorema anterior, segue que $H\leq G$.
\end{example}
%
\begin{example}
    Sejam $G$ o grupo dos números reais não nulos com a multiplicação, 
    %
    \[
    H = \{ x\in G \ | \ x = 1 \text{ ou } x\in G\setminus\mathbb{Q} \}
    \]
    %
    e
    %
    \[
    K = \{ x\in G \ | \ x\geq 1 \}.
    \]
    %
    Note que $H\not\leq G$ pois $\sqrt{2}\in H$ mas $\sqrt{2}\cdot\sqrt{2} = 2\notin H$.
    Além disso, $K$ também não é subgrupo de $G$ pois $2\in K$ mas $2^{-1} = 1/2 \notin K$.
\end{example}
%
\begin{theorem}
\label{teo-teste-subgrupo-finito}
    Seja $H\neq\varnothing$ um subconjunto finito de um grupo $G$. Se $H$ é fechado sob
    a operação de $G$, então $H$ é um subgrupo de $G$.
\end{theorem}
%
\begin{proof}
    Do Teorema \ref{teo-teste2-subgrupo}, para provarmos o resultado basta mostrar que
    $a^{-1}\in H$ sempre que $a\in H$.
    
    Note que se $a = e$, então $a^{-1} = a$ e terminamos. Do contrário, considere a
    sequência $a, a^2, \dots$ cujos elementos todos pertencem a $H$ por hipótese.
    Pela finitude de $H$, segue que os elementos dessa sequência não são todos distintos,
    ou seja, existem $i,j\in\mathbb{N}$ tais que $a^i = a^j$ com $i>j$. Ora, então
    $a^{i-j} = e$ e, como $a\neq e$, devemos ter $i-j>1$. Portanto, $a^{-1} = a^{i-j-1}$.
    Como $i-j-1\geq 1$, segue que $a^{i-j-1}\in H$ e terminamos.
\end{proof}
%
\begin{prop}
    Se $G$ é um grupo e $a\in G$, então $\langle a \rangle \leq G$.
\end{prop}
%
\begin{proof}
    Como $a\in\langle a \rangle$, temos $\langle a \rangle\neq\varnothing$. Sejam
    $a^n, a^m\in \langle a \rangle$. Então $a^n(a^m)^{-1} = a^{n-m}\in \langle a \rangle$ de modo que,
    pelo Teorema \ref{teo-teste-subgrupo}, temos $\langle a \rangle\leq G$.
\end{proof}
%
\begin{example}
    Em $\mathbb{Z}_{10}$, $\langle 2 \rangle = \{2, 4, 6, 8, 0\}$.
\end{example}
%
\begin{definition}[Centro]
\label{def-centro}
    O centro $Z(G)$ de um grupo $G$ é o subconjunto dos elementos de $G$ que comutam
    com todos os elementos de $G$. Em símbolos,
    %
    \[
    Z(G) = \{ g\in G \ | \ gx = xg, \, \forall x\in G \}.
    \]
    %
\end{definition}
%
\begin{theorem}
\label{teo-centro-subgrupo}
    O centro de um grupo $G$ é um subgrupo de $G$.
\end{theorem}
%
\begin{proof}
    Vamos mostrar esse resultado usando o Teorema \ref{teo-teste2-subgrupo}. Como
    $eg = ge$ para todo $g\in G$, temos $e\in Z(G)$, de modo que $Z(G)\neq\varnothing$.
    Agora, suponha que $a,b\in Z(G)$. Então $(ab)x = a(bx) = a(xb) = (ax)b = (xa)b = x(ab)$ para
    todo $x\in G$. Logo, $ab\in Z(G)$.
    
    Por fim, suponha que $a\in Z(G)$, ou seja,
    %
    \[
    ax = xa \iff a^{-1}(ax)a^{-1} = a^{-1}(xa)a^{-1} \iff exa^{-1} = a^{-1}xe \iff xa^{-1} = a^{-1}x. 
    \]
    %
    Logo, $a^{-1}\in Z(G)$ e o resultado segue do Teorema \ref{teo-teste2-subgrupo}.
\end{proof}
%
\begin{definition}[Centralizador]
\label{def-centralizador}
    Seja $a\in G$ um elemento fixado. O centralizador de $a\in G$, denotado $C_G(a)$,
    é o conjunto dos elementos de $G$ que comutam com $a$. Em símbolos,
    %
    \[
    C_G(a) = \{ g\in G \ | \ ga = ag \}.
    \]
    %
\end{definition}
%
\begin{theorem}
\label{teo-centralizador-subgrupo}
    Para cada elemento $a$ em um grupo $G$, o centralizador de $a$ é um subgrupo de $G$.
\end{theorem}
%
\begin{proof}
    A prova é similar àquela do Teorema \ref{teo-centro-subgrupo} e é deixada ao leitor.
\end{proof}
%
\begin{remark}
    Note que, para todo elemento $a$ de um grupo $G$, temos $Z(G)\subseteq C_G(a)$. Ademais,
    note que $G$ é abeliano se, e somente se, $C_G(a) = G$ para todo $a\in G$. Equivalentemente,
    $G$ é abeliano se, e somente se, $Z(G) = G$.
\end{remark}
%
\subsection*{Propriedades de grupos cíclicos}
%
\begin{theorem}
\label{teo-crit-potencias-iguais}
    Sejam $G$ um grupo e $a\in G$. Se $a$ tem ordem infinita, então $a^i = a^j$ se, e somente se, $i=j$.
    Se $a$ tem ordem finita $n$, então 
    %
    \[
    \langle a \rangle = \{e, a, a^2, \dots, a^{n-1}\}
    \]
    %
    e $a^i = a^j$ se, e somente se, $n\mid i-j$.
\end{theorem}
%
\begin{proof}
    Se $a$ tem ordem infinita, então não existe $n\neq 0$ tal que $a^n = e$. Como $a^i = a^j$ implica
    $a^{i-j} = e$, devemos ter $i=j$ e fica provada a primeira afirmação do teorema.
    
    Agora, suponha que $|a| = n$. Vamos mostrar que $\langle a \rangle = \{e, a, \dots, a^{n-1}\}$.
    A inclusão $\{e, a, \dots, a^{n-1}\} \subseteq \langle a \rangle$ é imediata. Para a inclusão
    no outro sentido, seja $a^k \in \langle a \rangle$ qualquer. Pelo algoritmo da divisão, existem
    inteiros $q, r$ tais que
    %
    \[
    k = qn + r, \, 0\leq r < n.
    \]
    %
    Sendo assim, temos
    %
    \[
    a^k = a^{qn + r} = (a^n)^qa^r = a^r,
    \]
    %
    ou seja, $a^k \in\{e, a, \dots, a^{n-1}\}$ e fica provada a nossa afirmação.
    
    Por fim, resta mostrar que $a^i = a^j \iff n\mid i-j$. Suponha, inicialmente, que $a^i = a^j$.
    Ora, então $a^{i-j} = e$ e, usando o algoritmo da divisão novamente, existem inteiros $q,r$ tais
    que
    %
    \[
    i-j = qn + r, \, 0\leq r < n.
    \]
    %
    Daí, segue que $e = a^{i-j} = a^r$. Como $n$ é o menor inteiro positivo tal que $a^n = e$, devemos
    ter $r=0$ e, portanto, $n\mid i-j$.
    
    Reciprocamente, se $i-j = nq$ então $a^{i-j} = a^{nq} = e^q = e$, de modo que $a^i = a^j$.
\end{proof}
%
\begin{corollary}
\label{cor-ordem-elemento-ordem-subgrupo}
    Para qualquer elemento $a$ de um grupo, $|a| = |\langle a \rangle|$.
\end{corollary}
%
\begin{corollary}
\label{cor-num-divide-ordem}
    Sejam $G$ em grupo e $a\in G$ de ordem $n$. Se $a^k = e$, então $n\mid k$.
\end{corollary}
%
\begin{proof}
    Como $a^k = e = a^0$, segue do Teorema \ref{teo-crit-potencias-iguais} que $n\mid k-0 = k$.
\end{proof}
%
\begin{theorem}
\label{teo-ordem-gerado}
    Sejam $a$ um elemento de ordem $n$ de um grupo e $k$ um inteiro positivo qualquer.
    Temos que
    %
    \[
    \langle a^k \rangle = \langle a^{\mdc(n,k)} \rangle
    \]
    %
    e
    %
    \[
    |a^k| = \frac{n}{\mdc(n,k)}.
    \]
    %
\end{theorem}
%
\begin{proof}
    Por simplicidade, vamos denotar $d = \mdc(n,k)$. Escreva $k = dr$. Como $a^k = (a^d)^r$, segue
    que $\langle a^k \rangle \subseteq \langle a^d \rangle$. Ademais, sabemos que existem inteiros
    $s,t$ tais que $d = ns + kt$ e, portanto,
    %
    \[
    a^d = a^{ns}a^{kt} = (a^k)^t \in \langle a^k \rangle.
    \]
    %
    Logo, $\langle a^d \rangle \subseteq \langle a^k \rangle$ e, consequentemente,
    $\langle a^d \rangle = \langle a^k \rangle$, provando a primeira parte do teorema.
    
    Para a segunda parte, vamos primeiro mostrar que $|a^d| = n/d$ para qualquer divisor $d$ de $n$.
    De fato, $(a^d)^{n/d} = a^n = e$, de modo que $|a^d| \leq n/d$. Por outro lado, se $i$ é um
    inteiro positivo menor que $n/d$ então $(a^d)^i\neq e$ por definição de $|a|$, pois $di < n$.
    Aplicando esse fato para $d = \mdc(n,k)$, obtemos
    %
    \[
    |a^k| = |\langle a^k \rangle| = |\langle a^{\mdc(n,k)} \rangle| = |a^{\mdc(n,k)}| = \frac{n}{\mdc(n,k)}.
    \]
    %
\end{proof}
%
\begin{corollary}
\label{cor-ordem-elemento-divide-ordem-grupo}
    Em um grupo cíclico finito, a ordem de um elemento divide a ordem do grupo.
\end{corollary}
%
\begin{corollary}
\label{cor-igualdade-subgrupos-gerados}
    Sejam $G$ um grupo e $a\in G$ com $|a|=n$. Então $\langle a^i \rangle = \langle a^j \rangle$
    se, e somente se, $\mdc(n,i) = \mdc(n,j)$ e $|a^i| = |a^j|$ se, e somente se, $\mdc(n,i) = \mdc(n,j)$.
\end{corollary}
%
\begin{proof}
    Pelo Teorema \ref{teo-ordem-gerado}, basta mostrarmos que 
    $\langle a^{\mdc(n,i)} \rangle = \langle a^{\mdc(n,j)} \rangle$ se, e somente se, $\mdc(n,i) = \mdc(n,j)$.
    Ora, é claro que se $\mdc(n,i) = \mdc(n,j)$ então os subgrupos gerados são iguais. Reciprocamente,
    se os subgrupos gerados são iguais então $|a^{\mdc(n,i)}| = |a^{\mdc(n,j)}|$ e, pela segunda afirmação
    do Teorema \ref{teo-ordem-gerado}, temos $n/\mdc(n,i) = n/\mdc(n,j)$ e, consequentemente,
    $\mdc(n,i) = \mdc(n,j)$.
\end{proof}
%
\begin{corollary}
\label{cor-geradores-ciclicos}
    Sejam $G$ um grupo e $a\in G$ com $|a|=n$. Então $\langle a \rangle = \langle a^j \rangle$ se,
    e somente se, $\mdc(n,j) = 1$ e $|a| = |\langle a^j \rangle|$ se, e somente se, $\mdc(n,j) = 1$.
\end{corollary}
%
\begin{corollary}
\label{cor-geradores-Zn}
    Um inteiro $k\in \mathbb{Z}_n$ é um gerador de $\mathbb{Z}_n$ se, e somente se, $\mdc(n,k) = 1$.
\end{corollary}
%
\begin{theorem}[Teorema Fundamental dos Grupos Cíclicos]
\label{teo-fund-grupos-ciclicos}
    Todo subgrupo de um grupo cíclico é cíclico. Ademais, se $|\langle a \rangle| = n$, então
    a ordem de qualquer subgrupo de $\langle a \rangle$ é um divisor de $n$ e, para cada divisor
    positivo $k$ de $n$, o grupo $\langle a \rangle$ tem exatamente um subgrupo de ordem $k$, a saber
    $\langle a^{n/k} \rangle$.
\end{theorem}
%
\begin{proof}
    Sejam $G = \langle a \rangle$ e $H\leq G$. Vamos mostrar que $H$ é cíclico.
    
    Ora, se $H = \{e\}$, então claramente $H = \langle e \rangle$. Então, vamos assumir
    $H\neq\{e\}$. Afirmamos que $H$ contém um elemento da forma $a^t, t>0$. De fato,
    como $G = \langle a \rangle$, todo elemento de $H$ é da forma $a^t$ e, quando
    $t<0$, então o inverso dessa potência tem expoente positivo e também pertence a $H$,
    verificando nossa afirmação.
    
    Agora, seja $m$ o menor inteiro positivo tal que $a^m\in H$, que existe pois $H\neq\{e\}$.
    Como $H$ é fechado para a multiplicação, temos $\langle a^m \rangle\subseteq H$. Afirmamos
    que vale a inclusão $\langle a^m \rangle \supseteq H$, ou seja, que $H = \langle a^m \rangle$.
    De fato, se $b$ é um elemento qualquer de $H$, então $b = a^k$ para algum $k$. Aplicando o 
    algoritmo da divisão para $k$ e $m$, temos que existem inteiros $q,r$ tais que
    %
    \[
    k = mq + r, \, 0\leq r < m.
    \]
    %
    Logo,
    %
    \[
    a^k = a^{mq}a^r \iff a^r = a^{-mq}a^k.
    \]
    %
    Como $a^k, a^{-mq} \in H$, segue que $a^r \in H$. Ora, mas $m$ foi tomado como sendo o menor
    inteiro positivo tal que $a^m \in H$. Portanto, devemos ter $r = 0$ e, daí, $b = a^k \in \langle a^m \rangle$,
    concluindo a demonstração de que $H$ é cíclico.
    
    Para provar a próxima parte do teorema, suponha que $|\langle a \rangle| = n$ e seja $H\leq\langle a \rangle$.
    Já mostramos que $H = \langle a^m \rangle$, sendo $m$ o menor inteiro positivo tal que $a^m\in H$.
    Tomando $a^n = b = e$ no parágrafo anterior, segue que $n = mq$.
    
    Para a parte final do teorema, seja $k$ um divisor positivo de $n$ qualquer. Do Teorema \ref{teo-ordem-gerado},
    sabemos que $\langle a^{n/k} \rangle$ tem ordem $n/\mdc(n,n/k) = n/(n/k) = k$. Seja $H\leq\langle a \rangle$
    de ordem $k$. Mostramos acima que $H = \langle a^m \rangle$, onde $m\mid n$. Ora, então $m = \mdc(n,m)$ e
    %
    \[
    k = |a^m| = |a^{\mdc(m,n)}| = n/\mdc(n,m) = n/m.
    \]
    %
    Portanto, $m = n/k$ e $H = \langle a^{n/k} \rangle$.
\end{proof}
%
\begin{corollary}
\label{cor-subgrupos-Zn}
    Para cada divisor positivo $k$ de $n$, o conjunto $\langle n/k \rangle$ é o único subgrupo
    de $\mathbb{Z}_n$ de ordem $k$. Além disso, os subgrupos de $\mathbb{Z}_n$ são todos dessa
    forma.
\end{corollary}
%
Antes de seguir para o próximo teorema, convém definir a seguinte função, que é bastante relevante
em Teoria dos Números.
%
\begin{definition}[Função $\phi$ de Euler]
\label{def-funcao-phi-euler}
    A função $\phi:\mathbb{N}^*\to\mathbb{N}^*$ tal que
    %
    \[
    \begin{cases}
        \phi(1) = 1 \\
        \phi(n) = \sharp\{ d\in\mathbb{N}^* \ | \ \mdc(d,n) = 1 \}, \, n>1
    \end{cases}
    \]
    %
    é chamada função phi de Euler. Dito de outro modo, ela associa, a cada $n>1$, a quantidade
    de inteiros positivos que são coprimos com $n$.
\end{definition}
%
\begin{theorem}
\label{teo-num-elementos-por-ordem}
    Se $d$ é um divisor positivo de $n$, o número de elementos de ordem $d$ em um grupo
    cíclico de ordem $n$ é $\phi(d)$.
\end{theorem}
%
\begin{proof}
    Pelo Teorema \ref{teo-fund-grupos-ciclicos}, o grupo tem exatamente um subgrupo de ordem $d$, digamos
    $\langle a \rangle$. Então todo elemento de ordem $d$ também gera o subgrupo $\langle a \rangle$ e,
    pelo Corolário \ref{cor-geradores-ciclicos}, um elemento $a^k$ gera $\langle a \rangle$ se, e somente
    se, $\mdc(k,d) = 1$. A quantidade de tais elementos é precisamente a quantidade de valores de $k$ tais
    que $\mdc(k,d) = 1$, ou seja, $\phi(d)$.
\end{proof}
%
\begin{corollary}
\label{cor-num-elementos-por-ordem}
    Em um grupo finito, o número de elementos de ordem $d$ é um múltiplo de $\phi(d)$.
\end{corollary}
%
\begin{proof}
    Se o grupo não tiver elementos de ordem $d$, então o teorema é verdadeiro pois $\phi(d)\mid 0$.
    Suponhamos então que existe $a\in G$ com $|a| = d$. Pelo Teorema \ref{teo-num-elementos-por-ordem},
    sabemos que $\langle a \rangle$ tem $\phi(d)$ elementos de ordem $d$. Se todos os elementos de ordem
    $d$ em $G$ estiverem em $\langle a \rangle$, terminamos. Do contrário, existe $b\in G$ de ordem $d$
    tal que $b\notin\langle a \rangle$. Ora, mas $\langle b \rangle$ também tem $\phi(d)$ elementos
    de ordem $d$, ou seja, até o momento temos $2\phi(d)$ elementos de ordem $d$ em $G$ dado que
    $\langle a \rangle$ e $\langle b \rangle$ não tenham elementos de ordem $d$ em comum.
    
    Caso exista $c$ de ordem $d$ que seja comum a ambos os subgrupos, então 
    $\langle a \rangle = \langle c \rangle = \langle b \rangle$ e $b\in\langle a \rangle$, absurdo.
    
    Procedendo desta forma, vemos que o número de elementos de ordem $d$ em um grupo finito
    é um múltiplo de $\phi(d)$.
\end{proof}
%

\section{Grupos de permutação}
    %
    \begin{definition}
    \label{def-grupo-perm}
        Um grupo de permutações de um conjunto $A$ é um conjunto de permutações de $A$ que forma
        um grupo sob a operação de composição.
    \end{definition}
    %
    É comum denotar uma permutação utilizando matrizes. Por exemplo, se $\alpha$ é a permutação
    em $\{1, 2, 3, 4\}$ que manda 1 em 2, 2 em 3, 3 em 1 e 4 em 4, escrevemos
    %
    \[
    \alpha = \begin{bmatrix}
        1 & 2 & 3 & 4 \\
        2 & 3 & 1 & 4
    \end{bmatrix}.
    \]
    %
    Com essa notação, o produto de duas permutações é feito da direita para a esquerda. Por exemplo,
    %
    \[
    \begin{bmatrix}
        1 & 2 & 3 & 4 \\
        2 & 3 & 1 & 4
    \end{bmatrix}\cdot
    \begin{bmatrix}
        1 & 2 & 3 & 4 \\
        3 & 4 & 1 & 2
    \end{bmatrix} =
    \begin{bmatrix}
        1 & 2 & 3 & 4 \\
        1 & 4 & 2 & 3
    \end{bmatrix}.
    \]
    %
    \begin{example}[Grupo Simétrico $S_n$]
        Seja $A = \{1, 2, \dots, n\}$. O grupo de todas as permutações de $A$ é chamado
        grupo simétrico de grau $n$, denotado por $S_n$. Os seus elementos têm a forma
        %
        \[
        \alpha = \begin{bmatrix}
            1 & 2 & \cdots & n \\
            \alpha(1) & \alpha(2) & \cdots & \alpha(n)
        \end{bmatrix}.
        \]
        %
        Deixamos a cargo do leitor mostrar que $|S_n| = n!$ e que $S_n$ não é abeliano se $n\geq 3$.
    \end{example}
    %
    A notação de matrizes, apesar de bem visual, pode ser um pouco difícil de trabalhar por vezes.
    Uma notação alternativa, também muito comum, é a de ciclos. Por exemplo, se $\alpha$ é a 
    permutação que manda 1 em 2, 2 em 1, 3 em 4, 4 em 6, 5 em 5 e 6 em 3, escrevemos
    %
    \[
    \alpha = (1,2)(3,4,6)(5).
    \]
    %
    É comum omitir o ciclo de tamanho um e escrever simplesmente $\alpha = (1,2)(3,4,6)$. 
    Ademais, é comum omitir as vírgulas quando isso não provoca ambiguidade.
    
    Com essa notação, a multiplicação de duas permutações é feita da direita para a esquerda, como
    no caso de matrizes. Por exemplo,
    %
    \[
    (13)(27)(456)\cdot (1237)(648) = (1732)(48)(56).
    \]
    %
    \subsection*{Propriedades das permutações}
    %
    \begin{theorem}
    \label{teo-prod-ciclos-disj}
        Toda permutação de um conjunto finito pode ser escrita como um ciclo ou
        como um produto de ciclos disjuntos.
    \end{theorem}
    %
    \begin{proof}
        Seja $\alpha$ uma permutação em $A = \{1, 2, \dots, n\}$. Para escrever $\alpha$ na forma
        de ciclos disjuntos, começamos escolhendo qualquer elemento de $A$, digamos $a_1$, e definindo
        %
        \[
        a_2 = \alpha(a_1), \qquad a_3 = \alpha^2(a_1), \qquad a_4 = \alpha^3(a_1),
        \]
        %
        e assim por diante, até que encontremos $m$ tal que $a_1 = \alpha^m(a_1)$. Tal $m$ existe
        pois a sequência
        %
        \[
        a_1, \alpha(a_1), \alpha^2(a_1), \alpha^3(a_1), \dots
        \]
        %
        é finita, haja vista que $A$ é finito. Então, até o momento temos
        %
        \[
        \alpha = (a_1, a_2, \dots, a_m)\cdots,
        \]
        %
        onde as reticências indicam a possibilidade de ainda não termos exaurido os elementos de $A$.
        Se isso acontecer, basta escolhermos $b_1\in A$ que não aparece no ciclo dos $a_i$'s e criar
        um novo ciclo usando o mesmo passo a passo acima. Note que esse novo ciclo não tem elementos
        em comum com o antigo, uma vez que, se tivesse, então existiriam $i,j$ tais que
        %
        \[
        \alpha^i(a_1) = \alpha^j(b_1) \iff \alpha^{i-j}(a_1) = b_1 \iff b_1 = a_t, 
        \]
        %
        para algum $t$, absurdo. 
        
        Procedendo desta forma até exaurirmos os elementos de $A$, nossa permutação terá a forma
        %
        \[
        \alpha = (a_1, a_2, \dots, a_m)(b_1, b_2, \dots, b_k)\cdots(c_1, c_2, \dots, c_s),
        \]
        %
        demonstrando o resultado desejado.
    \end{proof}
    %
    \begin{theorem}
    \label{teo-ciclos-disj-comutam}
        Se o par de ciclos $\alpha = (a_1, a_2, \dots, a_m)$ e $\beta = (b_1, b_2, \dots, b_n)$
        não têm entradas em comum, então $\alpha\beta = \beta\alpha$.
    \end{theorem}
    %
    \begin{proof}
        Podemos dizer que $\alpha$ e $\beta$ são permutações no conjunto
        %
        \[
        S = \{ a_1, a_2, \dots, a_m, b_1, b_2, \dots, b_n, c_1, c_2, \dots, c_k \},
        \]
        %
        onde os $c$'s são os elementos fixados por $\alpha$ e $\beta$ (pode ser que não exista nenhum $c$).
        Para provar o resultado, vamos mostrar que $(\alpha\beta)(x) = (\beta\alpha)(x)$ para todo $x\in S$.
        
        Se $x = a_i$ para algum $i$, então
        %
        \[
        (\alpha\beta)(a_i) = \alpha(a_i) = a_{i+1},
        \]
        %
        já que $\beta$ fixa os $a$'s (se $i=m$, então interpretamos $a_{i+1}$ como $a_1$). Similarmente,
        %
        \[
        (\beta\alpha)(a_i) = \beta(a_{i+1}) = a_{i+1}.
        \]
        %
        Um argumento inteiramente análogo mostra que $\alpha\beta$ e $\beta\alpha$ concordam nos $b$'s também.
        
        Por fim, se $x = c_i$ para algum $i$ então
        %
        \[
        (\alpha\beta)(c_i) = \alpha(c_i) = c_i = \beta(c_i) = (\beta\alpha)(c_i).
        \]
        %
    \end{proof}
    %
    \begin{theorem}
    \label{teo-ordem-perm}
        A ordem de uma permutação em um conjunto finito escrita como produto de ciclos disjuntos
        é o menor múltiplo comum dos comprimentos dos ciclos.
    \end{theorem}
    %
    \begin{proof}
        Primeiro, observe que um ciclo de comprimento $n$ tem ordem $n$ (verifique!). Suponha então
        que $\alpha$ e $\beta$ são ciclos disjuntos de comprimentos $m$ e $n$, respectivamente, e seja
        $k = \mmc(m,n)$. Segue do Teorema \ref{teo-prod-ciclos-disj} que $\alpha^k = e = \beta^k$ e,
        como $\alpha$ e $\beta$ comutam, $(\alpha\beta)^k$ também é a identidade. Portanto, pelo
        Corolário \ref{cor-num-divide-ordem}, temos que a ordem $t$ de $\alpha\beta$ divide $k$.
        
        Ora, mas então $\alpha^t = \beta^{-t}$, e essas duas permutações são disjuntas
        pois $\alpha$ e $\beta$ são disjuntos. Logo, ambas devem ser a identidade e, portanto,
        $m\mid t$ e $n\mid t$. Portanto, $k\mid t$ e segue que $k=t$.
        
        O caso do produto de mais de duas permutações disjuntas é demonstrado de maneira análoga
        e deixado para o leitor.
    \end{proof}
    %
    \begin{example}
        Esse teorema pode ser usado para mostrar que as únicas ordens possíveis para os elementos de
        $S_7$ são 1, 2, 3, 4, 5, 6, 7, 10 e 12.
    \end{example}
    %
    \begin{theorem}
    \label{teo-perm-2ciclos}
        Toda permutação de $S_n$, $n>1$, é um produto de 2-ciclos.
    \end{theorem}
    %
    \begin{proof}
        Primeiro, note que podemos escrever $e = (12)(12)$. Pelo Teorema \ref{teo-prod-ciclos-disj}, toda
        permutação pode ser escrita como
        %
        \[
        (a_1a_1\cdots a_k)(b_1b_2\cdots a_t)\cdots(c_1c_2\cdots c_s).
        \]
        %
        Ora, mas esse produto é igual a
        %
        \[
        (a_1a_k)(a_1a_{k-1})\cdots(a_1a_2)(b_1b_t)(b_1b_{t-1})\cdots(b_1b_2)(c_1c_s)(c_1c_{s-1})\cdots (c_1c_2),
        \]
        %
        e segue o resultado.
    \end{proof}
    %
    \begin{example}
        Podemos escrever
        %
        \[
        (12345) = (54)(53)(52)(51)
        \]
        %
        ou também
        %
        \[
        (12345) = (54)(52)(21)(25)(23)(13).
        \]
        %
    \end{example}
    %
    \begin{lemma}
	\label{lema identidade permutacoes}
        Se $e = \beta_{1}\beta_{2}\cdots\beta_{r}$, sendo os $\beta_i$
        ciclos de tamanho 2, então $r$ é par.
	\end{lemma}
	%
	\begin{proof}
		Suponha $r>2$. Então, $\beta_{r-1}\beta_{r}$ tem uma das
		seguintes formas:
		%
		\begin{align}
		    \label{1} e &= (ab)(ab) \\
		    \label{2} (ab)(bc) &= (ac)(ab) \\
		    \label{3} (ac)(cb) &= (bc)(ab) \\
		    \label{4} (ab)(cd) &= (cd)(ab)
		\end{align}
		%
		Se \eqref{1} ocorre, então 
		$e = \beta_{1}\beta_{2}\cdots\beta_{r-2}$ e, pelo 
		Segundo Princípio da Indução, $r-2$ é par e, portanto, 
		$r$ é par. Se \eqref{2}, \eqref{3} ou \eqref{4} ocorre,
		podemos substituir o lado direito pelo lado esquerdo em
		$e$, levando a última ocorrência de $a$ para o penúltimo
		ciclo da esquerda para a direita. Prosseguindo dessa
		maneira para reescrever os pares da forma
		$\beta_{i-1}\beta_{i}$, temos de, em algum momento, 
		obter uma sequência com $r-2$ ciclos de tamanho 2 
		porque, do contrário, teríamos
		%
		\begin{equation*}
		    e = (a \star)\cdots\gamma_{r}
		\end{equation*}
		%
		sendo $(a\star)$ a única ocorrência de $a$. Logo, 
		$a$ seria levado em um elemento distinto de $a$ e 
		teríamos um absurdo, pois $e$ fixa todos os elementos.
		Portanto, devemos ter uma sequência de $r-2$ ciclos. 
		Pelo Segundo Princípio da Indução Matemática, 
		$r-2$ é par, logo $r$ é par.
	\end{proof}
	%
	\par\vspace{0.3cm} Ao invés de ``ciclos de tamanho $2$'' ou
	``$2$-ciclos'', podemos usar o termo \textit{transposições}. 
	De fato, essa nomenclatura é mais intuitiva, pois o que uma
	permutação de $2$ elementos faz é trocar os dois de lugar, 
	ou seja, transpô-los.
	%
	\par\vspace{0.3cm} Vamos chamar de $A_n$ o conjunto das
	permutações pares de $n$ elementos, ou seja, as permutações 
	de $n$ elementos que podem ser decompostas em um produto 
	de um número par de transposições. De fato, $A_n$ é 
	subgrupo de $S_n$, chamado subgrupo alternado, como demonstrado abaixo.
	%
	\begin{theorem}
	\label{grupo alternante subgrupo de S_n}
		$A_n \leq S_n$.
	\end{theorem}
	%
	\begin{proof}
        Se $\alpha, \beta \in A_n$, então
        %
        \begin{equation*}
        \alpha\beta^{-1}
        = 
        \sigma_1\sigma_2\cdots\sigma_r\gamma_1\gamma_2\cdots\gamma_s \in A_n,
        \end{equation*}
        %
        pois $r+s$ é par já que $r$ e $s$ são pares. 
    \end{proof}
    %
	\begin{theorem}
	\label{ordem do grupo alternante}
		Para todo $n>1, |A_n| = \displaystyle{\frac{n!}{2}}$.
	\end{theorem}
	%
	\begin{proof}
		Como toda permutação pode ser escrita como produto de 
		ciclos de tamanho 2, sabemos que se 
		$\alpha\in S_n$, $\alpha$ é par ou $\alpha$ é ímpar.
		%
		\par\vspace{0.3cm} Se $\alpha$ é par, então $(12)\alpha$ é ímpar.
		Além disso, $(12)\alpha\neq (12)\beta$ para $\alpha\neq\beta$.
		Logo, a quantidade de permutações pares é maior ou igual à de
		permutações ímpares, pois multiplicar uma permutação par por
		$(12)$ gera uma permutação ímpar, mas pode não gerar
		\textbf{todas} as permutações ímpares.
		%
		\par\vspace{0.3cm} Por outro lado, se $\alpha$ é ímpar, então
		$(12)\alpha$ é par. Novamente, $(12)\alpha\neq (12)\beta$ para
		$\alpha\neq\beta$. Logo, a quantidade de permutações ímpares é
		maior ou igual à de permutações pares, pois multiplicar uma
		permutação ímpar por $(12)$ gera uma permutação par, mas pode 
		não gerar \textbf{todas} as permutações pares.
		%
		\par\vspace{0.3cm} Portanto, a quantidade de permutações pares é
		igual à quantidade de permutações ímpares, logo 
		$|A_n| = \displaystyle{\frac{|S_n|}{2} = \frac{n!}{2} }$.
	\end{proof} 
	%
% 	\par\vspace{0.3cm} Por exemplo, podemos escrever a permutação
% 	$(16523)$ como combinação das permutações $(13456)$ e $(132)$. 
% 	De fato, façamos $r = (13456)$ e $s = (132)$. Queremos obter 
% 	o ciclo $\alpha = (16523) = (13)(12)(15)(16)$. Note que $\alpha$ 
% 	é uma permutação par. Com alguns cálculos (preferencialmente em
% 	computadores), podemos ver que 
% 	$|\langle r,s \rangle| = 360 = |A_6|$, logo 
% 	$A_6 = \langle r,s \rangle$. Como $\alpha\in A_6$, então é 
% 	claro que podemos obtê-la a partir de $r$ e $s$.

\section{Isomorfismos}
    %
    \begin{definition}
	\label{def isomorfismo}
	\index{Isomorfismo}
		Um \textbf{isomorfismo} $\phi$ de um grupo $G$ em um grupo
		$\overline{G}$ é uma aplicação bijetora $\phi$ de $G$ em
		$\overline{G}$ que preserva a operação do grupo, isto é, 
		tal que $\phi(ab) = \phi(a)\phi(b), \, \forall a,b\in G$.
		Nesse caso, denotamos $G\cong\overline{G}$.
	\end{definition}
	%
	Na Definição \ref{def isomorfismo}, no lado
	esquerdo da igualdade a operação é a de $G$, enquanto que no lado
	direito a operação é a de $\overline{G}$.
	%
	\begin{theorem}[Teorema de Cayley]
	\label{Cayley}
		Todo grupo é isomorfo a um grupo de permutações.
	\end{theorem}
	%
	\begin{proof}
		Seja $G$ um grupo. Para qualquer $g\in G$, defina $T_g:G\to G$
		por $T_g(x) = gx, \forall x\in G$. Note que $T_g$ é uma
		permutação no conjunto de elementos de $G$, pois cada elemento
		$x\in G$ é levado no elemento $gx$, também em $G$.
		%
		\par\vspace{0.3cm} Agora, seja 
		%
		\begin{equation*}
		    \overline{G} = \{ T_g | g\in G \}.
		\end{equation*}
		%
		Então $\overline{G}$ é um grupo sob a composição. 
		Para verificar isso, observe que para quaisquer $g$ e $h$ em $G$
		temos 
		%
		\begin{equation*}
		    T_gT_h(x) = T_g(T_h(x)) = T_g(hx) = (gh)x = T_{gh}(x), 
        \end{equation*}
		%
		logo $T_gT_h = T_{gh}$.
		%
		\par\vspace{0.3cm} Daí, segue que $T_e$ é a identidade e
		$(T_g)^{-1} = T_{g^{-1}}$. Como a composição é associativa,
		$\overline{G}$ é grupo.
		%
		\par\vspace{0.3cm} Agora, podemos definir 
		$\phi: G\to\overline{G}$ dada por 
		$\phi(g) = T_g, \forall g\in G$. Se $T_g = T_h$, então 
		$T_g(e) = T_h(e)$ ou, equivalentemente, $ge = he$. Então, 
		$g = h$ e $\phi$ é injetora. Pela maneira que construímos
		$\overline{G}$, vemos que $\phi$ é sobrejetora. Por fim, sejam
		$a,b\in G$ quaisquer. Então
		%
		\[
		    \phi(ab) = T_{ab} = T_aT_b = \phi(a)\phi(b)
		\]
		%
		e terminamos a demonstração.
	\end{proof}
	%
	Os isomorfismos têm algumas propriedades, 
	listadas a seguir nos Teoremas \ref{isomorfismos em elementos} 
	e \ref{isomorfismos em grupos}.
	%
	\begin{theorem}
	\label{isomorfismos em elementos}
	    Valem os seguintes itens:
	    %
		\begin{enumerate}
			\item \vspace{0.3cm} $\phi$ leva a identidade de $G$ na
			identidade de $\overline{G}$;
			\item $\forall n\in\N$ e $\forall a\in G, \phi (a^n) = [\phi (a)]^n$;
			\item Dados $a,b \in G$ quaisquer, 
			$ab = ba \iff \phi (a)\phi (b) = \phi (b)\phi (a)$;
			\item $G = \langle a \rangle \iff \overline{G} =
			\langle\phi(a) \rangle$;
			\item $|a| = |\phi(a)|, \forall a\in G$;
			\item Dados $k\in\mathbb{Z}$ e $b\in G$, a equação 
			$x^k = b$ tem a mesma quantidade de soluções em $G$ que a
			equação $x^k = \phi(b)$ tem em $\overline{G}$;
			\item Se $G$ é finito, $G$ e $\overline{G}$ têm o mesmo
			número de elementos de cada ordem. 
		\end{enumerate}
		%
	\end{theorem}
	%
	\begin{proof}
		\textbf{1.} Sejam $e$, $\overline{e}$ as identidades de $G$ 
		e $\overline{G}$, respectivamente. Logo, temos:
		%
		\begin{equation*}
		    e = e\cdot e \Rightarrow \phi (e) = \phi (e)\phi (e)
		\end{equation*} 
		%
		Como $\phi (e)\in \overline{G}$, segue que 
		$\phi(e) = \overline{e}$. 
		%
		\par\vspace{0.4cm}
		\textbf{2.} Se $n=0$, temos 
		$\phi (e) = \overline{e} = (\overline{e})^0$. 
		Para $n>0$, temos 
		$\phi(a^n)=\underbrace{\phi(a)\phi(a)\dots\phi(a)}_{n} 
		= [\phi(a)]^n$.
		Agora, se $n<0$, então $-n>0$ e temos:
		%
		\begin{equation*}
		    \phi (e) = \overline{e} = \phi (a^n a^{-n}) 
		    = \phi (a^n)[\phi (a)]^{-n} \Rightarrow [\phi (a)]^n 
		    = \phi (a^n)
		\end{equation*}
		%
		\par\vspace{0.4cm}
		\textbf{3.} Sejam $a,b \in G$. Temos que:
		%
		\begin{equation*}
		(\Rightarrow) \hspace{12pt}ab = ba \Rightarrow \phi (ab) 
		= \phi (ba) \Rightarrow \phi (a)\phi (b) = \phi (b)\phi (a)
		\end{equation*}
		%
		\begin{equation*}
		(\Leftarrow)\hspace{12pt}\phi(a)\phi(b) = \phi(b)\phi(a)
		\Rightarrow \phi(ab) = \phi(ba) \Rightarrow ab = ba
		\end{equation*}
		%
		em que a volta se deve à injetividade de $\phi$.
		%
		\par\vspace{0.4cm}
		\textbf{4.} Seja $G = \langle a \rangle $. Como $\overline{G}$ 
		é fechado para a sua operação, $\langle \phi(a)\rangle \subseteq \overline{G}$. 
		Como $\phi$ é sobrejetora, então para todo $b\in\overline{G}$,
		existe $k$ inteiro tal que 
		$b = \phi(a^k) = [\phi(a)]^k \in\langle \phi(a)\rangle $, 
		logo $\overline{G}\subseteq\langle \phi(a)\rangle$ e, portanto,
		$\overline{G} = \langle \phi(a)\rangle.$
		%
		\par\vspace{0.3cm} Agora, seja 
		$\overline{G} = \langle \phi(a)\rangle$. Como $G$ é fechado para a sua operação,
		$\langle a\rangle \subseteq G$. Note que 
		$\forall b\in G, \phi(b)\in\langle \phi(a)\rangle$, i.e., 
		existe $k\in\mathbb{Z}$ tal que 
		$\phi(b) = [\phi(a)]^k = \phi(a^k)$. Logo, como $\phi$ é
		injetora, $b = a^k\in\langle a\rangle $ e 
		$G \subseteq \langle a\rangle$. Portanto, $G = \langle a\rangle$.
		%
		\par\vspace{0.4cm}
		\textbf{5.} Seja $a\in G$ com $|a| = n$ e $|\phi(a)| = k$. Então:
		%
		\begin{equation*}
		    \phi(a^n) = \overline{e} = [\phi(a)]^n.
		\end{equation*}
		%
		logo $k\mid n$.	Por outro lado, temos que:
		%
		\begin{equation*}
		    [\phi(a)]^k = \overline{e} \Rightarrow a^k = e
		\end{equation*}
		%
		logo $n\mid k$. Portanto, $n = k$.
		\par\vspace{0.4cm}
		\textbf{6.} Seja $a \in G$ tal que $a^k = b$. Então, 
		$\phi(a^k) = [\phi(a)]^k = \phi(b)$, ou seja, se $a$ é 
		solução de $x^k = b$ em $G$, então $\phi(a)$ é solução de 
		$x^k = b$ em $\overline{G}$. Como $\phi$ é injetora,
		$\phi(a)\neq\phi(b)$ quando $a\neq b$, ou seja, soluções
		diferentes da equação em $G$ levam a soluções diferentes da
		equação em $\overline{G}$.
		%
		\par\vspace{0.4cm}
		\textbf{7.} Como $|a| = |\phi(a)|$ para todo $a\in G$, segue que 
		se $G$ tem $k$ elementos de ordem $n_k$, então $\overline{G}$
		também terá $k$ elementos de ordem $n_k$ devido à injetividade 
		de $\phi$.	
	\end{proof}
	%
	\begin{theorem}
	\label{isomorfismos em grupos}
	    Valem os seguintes itens:
	    %
		\begin{enumerate}
			\item Se $\phi$ é um isomorfismo de $G$ em $\overline{G}$,
			então $\phi ^{-1}$ é um isomorfismo de $\overline{G}$ em $G$;
			\item $G$ é abeliano se, e só se, $\overline{G}$ é abeliano;
			\item $G$ é cíclico se, e só se, $\overline{G}$ é cíclico;
			\item Se $K$ é um subgrupo de $G$, então 
			$\phi(K) = \{\phi(k) \ | \ k\in K\}$ é um subgrupo de
			$\overline{G}$;
			\item Se $\overline{K}$ é subgrupo de $\overline{G}$, 
			então $\phi^{-1}(\overline{K}) 
			= \{ g\in G \ | \ \phi(g)\in\overline{K}\}$ é subgrupo de $G$;
			\item $\phi(Z(G)) = Z(\overline{G})$.
		\end{enumerate}
		%
	\end{theorem}
	%
	\begin{proof}
		\textbf{1.} Como $\phi$ é bijetiva, $\phi^{-1}$ também é. 
		Logo, basta verificarmos se $\phi^{-1}$ preserva a operação.
		%
		\par\vspace{0.3cm} Para isso, note que 
		%
		\begin{align*}
		    \phi^{-1}(ab) 
		    = \phi^{-1}(a)\phi^{-1}(b) \iff ab 
		    = \phi(\phi^{-1}(a))\phi(\phi^{-1}(b)) \iff ab = ab.
		\end{align*} 
		%
		Logo, $\phi^{-1}$ de fato preserva a operação.
		%
		\par\vspace{0.4cm}
		\textbf{2.} Pela propriedade 3 do Teorema 
		\ref{isomorfismos em elementos}, sabemos que $ab = ba$ se, e
		somente se, $\phi(a)\phi(b) = \phi(b)\phi(a)$, ou seja, os
		elementos de $G$ comutam se, e só se, os elementos de 
		$\overline{G}$ comutam.
		%
		\par\vspace{0.4cm}
		\textbf{3.} O resultado segue da propriedade 4 do
		Teorema \ref{isomorfismos em elementos}, que diz que 
		$G = \langle a\rangle \iff \overline{G} 
		= \langle \phi(a)\rangle$.
		%
		\par\vspace{0.4cm}
		\textbf{4.} Sejam $k_1, k_2 \in K$ quaisquer. Temos que:
		%
		\begin{equation*}
		    \phi(k_1)\phi(k_2^{-1}) = \phi(k_1k_2^{-1}) \in\phi(K)
		\end{equation*}
		%
		pois $k_1k_2^{-1} \in K$ já que $K$ é subgrupo.
		%
		\par\vspace{0.4cm}
		\textbf{5.} Se $\overline{K}$ é subgrupo de $\overline{G}$, 
		então para quaisquer $\phi(g_1), \phi(g_2^{-1})\in \overline{K}$,
		temos:
		%
		\begin{align*}
		    \phi^{-1}(\phi(g_1))\phi^{-1}(\phi(g_2^{-1})) 
		    = \phi^{-1}(\underbrace{\phi(g_1)\phi(g_2^{-1})}_{\in\overline{K}}) 
		    \in \phi^{-1}(\overline{K}).
		\end{align*}
		%
		\par\vspace{0.3cm} Portanto, $\phi^{-1}(\overline{K})$ é 
		subgrupo de $G$.
		%
		\par\vspace{0.4cm}
		\textbf{6.} Por definição, sabemos que 
		$Z(G) = \{ z\in G \ | \ gz = zg, \forall g\in G \}$. Daí, 
		$\phi(Z(G)) = 
		\{ \phi(z)\in \overline{G} \ | \ \phi(g)\phi(z) 
		= \phi(z)\phi(g), \forall \phi(g)\in\overline{G}\}$, que é, 
		por definição, $Z(\overline{G})$.
	\end{proof}
	%
	Na Definição \ref{def isomorfismo}, nada nos
	impede de tomar $G = \overline{G}$. De fato, se o fizermos, obtemos
	um tipo de isomorfismo especial, chamado {\bf automorfismo}.
	%
	\begin{definition}
	\label{def automorfismo}
		Um isomorfismo de um grupo $G$ em si mesmo é chamado 
		automorfismo. O conjunto de todos os automorfismos de um 
		grupo $G$ é denotado por $\aut(G)$.
	\end{definition}
	%
	Além disso, podemos ainda definir um tipo 
	especial de automorfismo, chamado {\bf automorfismo interno}.
	%
	\begin{definition}
	\label{def automorfismo interno}
		O automorfismo de $G$ definido por $\phi_a(x) = axa^{-1}$ 
		para todo $x$ em $G$ é chamado automorfismo interno de $G$
		induzido por $a$. O conjunto de todos os automorfismos
		internos de um grupo $G$ é denotado por $\inn(G)$.
	\end{definition}
	%
	Um fato interessante de $\aut(G)$ e $\inn(G)$ 
	é o seguinte.
	%
	\begin{theorem}
		$\aut(G)$ e $\inn(G)$ são grupos com a operação de composição.
	\end{theorem}
	%
	\begin{proof}
		Sejam $\phi_a$ e $\phi_b$ elementos quaisquer de $\inn(G)$. 
		Note que $(\phi_a\circ\phi_b)(x) = \phi_a(\phi_b(x)) 
		= a(bxb^{-1})a^{-1} = abxb^{-1}a^{-1} = (ab)x(ab)^{-1} 
		= \phi_{ab}(x) \in \inn(G)$, logo $\inn(G)$ é fechado para a
		composição. Além disso, a composição é associativa, 
		$\phi_e(x) = x$ e $(\phi_a\circ\phi_{a^{-1}})(x) = \phi_e(x)$, 
		ou seja, $\inn(G)$ tem identidade e contém os inversos. 
		Logo, $\inn(G)$ é grupo sob composição.
		%
		\vspace{0.3cm}\par Agora, sejam $\alpha$ e $\beta$ elementos
		quaisquer de $\aut(G)$. Note que
		%
		\begin{equation*}
		    \alpha^{-1}(xy) = \alpha^{-1}(x)\alpha^{-1}(y) 
		    \iff xy = \alpha(\alpha^{-1}(x)\alpha^{-1}(y)) 
		    \iff xy = \alpha(\alpha^{-1}(xy))\ \iff xy = xy
		\end{equation*}
		%
		logo $\alpha^{-1}$ preserva a operação de $G$. 
		Como $\alpha^{-1}$ é bijetiva, então é um isomorfismo, ou seja,
		temos $\alpha^{-1}\in \aut(G)$. Além disso, como a composição
		de funções é associativa, $(\alpha\circ\beta)(x) 
		= \alpha(\underbrace{\beta(x)}_{\in G}) \in \aut(G)$ e o
		isomorfismo $\theta(x) = x$ é a identidade, concluímos que
		$\aut(G)$ é grupo sob composição.
	\end{proof}
	%
	\begin{example}
	Por exemplo, se tomarmos 
	$(\mathbb{C^{*}}, \cdot)$, o grupo dos complexos não nulos com a
	multiplicação, e definirmos a função 
	$\phi : \mathbb{C^{*}}\to\mathbb{C^{*}}$ tal que 
	$\phi(a+bi) = a-bi$, com $a,b\in\R$, então $\phi$ é automorfismo 
	de $\mathbb{C^{*}}$.
	%
	\par\vspace{0.3cm} De fato, note que se $\phi(a+bi) = \phi(c+di)$,
	então:
	%
	\begin{equation*}
        a = bi = c - di 
        \iff
        \begin{cases}
            a = b \\ 
            c = d
        \end{cases}
	\end{equation*}
	%
	\par\vspace{0.3cm} logo $\phi$ é injetora.  Além disso, se
	$\alpha\in\mathbb{C^{*}}$, então existe $\beta\in\mathbb{C^{*}}$ 
	tal que $\phi(\beta) = \alpha$, a saber, $\beta = \overline{\alpha}$, i.e.,
	$\alpha$ e $\beta$ são conjugados complexos.
	Logo, $\phi$ é sobrejetora.
	%
	\par\vspace{0.3cm} Por fim, temos que
	%
	\begin{align*}
        &\phi[(a + bi)(c + di)] = \phi[(ac - bd) + (ad + bc)i] \\ 
        &= (ac - bd) - (ad + bc)i = (a- bi)(c - di) \\
        &= \phi(a + bi)\cdot\phi(c + di).
	\end{align*}
	%
	\par\vspace{0.3cm} Logo, $\phi$ preserva a operação e, portanto, 
	é automorfismo.
	\end{example}
	%
	\begin{example}
	Outro exemplo é o conjunto de automorfismos
	internos de $D_4$, o grupo diedral de ordem $8$. Vamos mostrar que
	%
	\begin{equation*}
        \inn(D_4) =
        \displaystyle{
        \left\{
        \phi_{R_0},\phi_{R_{90}},\phi_{H},\phi_{D}
        \right\}},
	\end{equation*}
	%
	onde $\phi_{R_{\theta}}$ denota a rotação de $\theta$ graus, $\phi_H$ denota
	a reflexão em torno de uma reta horizontal e $\phi_D$ denota a reflexão em torno
	da diagonal.
	\par\vspace{0.3cm} De fato, para tal basta mostrarmos que 
	$\phi_{R_0}, \phi_{R_{90}}, \phi_{H}, \phi_{D}$ são todos distintos.
	Para isso, basta notar que
	%
	\begin{align*}
        \phi_{R_{0}}(H) = H &\neq V = \phi_{R_{90}}(H) \\
        \phi_{R_{0}}(R_{90}) = R_{90} &\neq R_{270} = \phi_{H}(R_{90}) \\
        \phi_{R_{0}}(R_{270}) = R_{270} &\neq R_{90} = \phi_{D}(R_{270}) \\
        \phi_{R_{90}}(R_{90}) = R_{90} &\neq R_{270} = \phi_{H}(R_{90}) \\
        \phi_{R_{90}}(R_{90}) = R_{90} &\neq R_{270} = \phi_{D}(R_{90}) \\
        \phi_{H}(V) = V &\neq H = \phi_{D}(V)
	\end{align*}
	%
	\par\vspace{0.3cm} Logo, $\phi_{R_0}, \phi_{R_{90}}, \phi_{H}, \phi_{D}$
	são, de fato, todos distintos.
	\end{example}
	%
	\par\vspace{0.3cm} Em geral, dado um grupo $G$, não é fácil determinar
	$\aut(G)$ e $\inn(G)$. Contudo, para alguns grupos conseguimos fazê-lo 
	com relativa facilidade. Um desses grupos é $\mathbb{Z}_n$.
	%
	\begin{theorem}
	\label{automorfismos de Z_n}
		$\aut(\mathbb{Z}_n) \cong U(n)$.
	\end{theorem}
	%
	\begin{proof}
		Seja $T: \aut(\mathbb{Z}_n)\to U(n)$ tal que $\alpha\mapsto\alpha(1)$,
		ou seja, o automorfismo $\alpha$ é levado na imagem de $1$ por
		$\alpha$. Como $\alpha(k) = k\alpha(1)$, $T$ é injetora, pois se 
		$\alpha(1) = \beta(1)$, então 
		$\alpha(k) = k\alpha(1) = k\beta(1) = \beta(k)$, logo 
		$\alpha = \beta$.
		%
		\par\vspace{0.3cm} Agora, seja $r\in U(n)$ e tome o automorfismo
		$\alpha$ de $\mathbb{Z}_n$ dado por $\alpha(s) = sr \pmod n$. 
		Como $T(\alpha) = \alpha(1) = r\pmod n = r$, $T$ é sobrejetora.
		%
		\par\vspace{0.3cm} Por fim, sejam $\alpha$ e $\beta$ elementos
		quaisquer de $\aut(\mathbb{Z}_n)$. Então, temos:
		%
		\begin{align*}
            T(\alpha\circ\beta) = (\alpha\circ\beta)(1) = \alpha(\beta(1)) 
            = \alpha(\underbrace{1 + 1 + \cdots + 1}_{\beta(1)}) 
            = \underbrace{\alpha(1) + \alpha(1) + \cdots +
            \alpha(1)}_{\beta(1)} 
            = \alpha(1)\cdot\beta(1).
		\end{align*}
		%
		\par\vspace{0.3cm} Logo, $T$ preserva a operação e, portanto, 
		é isomorfismo.
	\end{proof}
	%
	\par\vspace{0.3cm} Outro exemplo é $\aut(\mathbb{Z})$.	
	Pela propriedade 4 do Teorema \ref{isomorfismos em elementos}, 
	um isomorfismo deve levar gerador em gerador. Como $\mathbb{Z}$ 
	possui apenas dois geradores, $1$ e $-1$, há apenas dois 
	automorfismos em $\aut(\mathbb{Z})$: o automorfismo identidade, 
	que leva $1$ em $1$ e $-1$ em $-1$; e o automorfismo que leva 
	$1$ em $-1$ e $-1$ em $1$.
	%
	\begin{remark}
		Pelo Teorema \ref{automorfismos de Z_n}, 
		$|\aut(\mathbb{Z}_n)| = |U(n)| = \phi(n)$ (sendo $\phi(n)$ a 
		função totiente de Euler). Por outro lado, vemos, a partir do que
		fizemos acima, que $|\aut(\mathbb{Z})| = 2$. Logo, em geral,
		$|\aut(\mathbb{Z}_n)| > |\aut(\mathbb{Z})|$, pois em geral 
		$\phi(n) > 2$.
	\end{remark}
    %
\section{Classes laterais}
\label{sec-classes-laterais}
    %
    Um conceito importante no estudo de grupos é o de 
    \textbf{classes laterais}.
    %
	\begin{definition}
		\label{def classes laterais}
		Sejam $G$ um grupo e $H$ um subconjunto não vazio de $G$. 
		Para qualquer $a\in G$,	o conjunto $\{ah \ | \ h \in H \}$ é 
		denotado por $aH$. Analogamente, $Ha = \{ha \ | \ h \in H\}$
		e $aHa^{-1} = \{aha^{-1} \ | \ h \in H\}$. Quando $H$ é subgrupo de $G$,
		o conjunto $aH$ é dito classe lateral à esquerda de $H$ em $G$
		contendo $a$, enquanto que $Ha$ é dito classe lateral à direita 
		de $H$ em $G$ contendo $a$. Nesse caso, o elemento $a$ é dito o
		representante da classe lateral $aH$ (ou $Ha$). Usamos $|aH|$ para
		denotar o número de elementos no conjunto $aH$, e $|Ha|$ para 
		denotar o número de elementos em $Ha$.
	\end{definition}
	%
	Assim como os isomorfismos, as classes laterais têm propriedades, listadas a seguir no Lema \ref{propriedades}.
	%
	\begin{lemma}
	\label{propriedades}
	    Valem os seguintes itens:
	    %
		\begin{enumerate}
			\item $a\in aH$ (a classe lateral à esquerda de $H$ contendo 
			$a$ contém $a$); 
			\item $aH = H \iff a\in H$ (a classe lateral absorve um 
			elemento se, e só se, esse elemento está em $H$);
			\item $(ab)H = a(bH)$ e $H(ab) = (Ha)b$;
			\item $aH = bH \iff a\in bH$ (uma classe lateral é unicamente
			determinada por um de seus elementos);
			\item $aH = bH$ ou $aH \cap bH = \emptyset$ (duas classes 
			laterais ou são iguais ou disjuntas);
			\item $aH = bH \iff a^{-1}b\in H$ (uma questão de classe 
			lateral se torna uma questão sobre $H$);
			\item $|aH| = |bH|$ (todas as classes laterais têm mesmo tamanho);
			\item $aH = Ha \iff H = aHa^{-1}$;
			\item $aH$ é subgrupo de $G$ se, e só se, $a\in H$ ($H$ é a 
			única classe lateral que é subgrupo de $G$).
		\end{enumerate}
		%
	\end{lemma}
	%
	\begin{proof}
		\textbf{1.} Como $e\in H$, então $a = ae \in H$.
		%
		\par\vspace{0.4cm}
		\textbf{2.} Se $aH = H$, então $a\in aH = H$. Por outro lado, se 
		$a\in H$, é claro que $aH\subseteq H$. Além disso, se $h\in H$, 
		então $a^{-1}h\in H$, pois $a^{-1}\in H$ já que $H$ é subgrupo. 
		Logo, $h = (aa^{-1})h = a(a^{-1}h)\in aH$. Portanto, $H\subseteq aH$ 
		e $H = aH$. 
		%
		\par\vspace{0.4cm}
		\textbf{3.} Como $(ab)h = a(bh)$ e $h(ab) = (ha)b$, o resultado segue.
		%
		\par\vspace{0.4cm}
		\textbf{4.} Se $aH = bH$, então $a = ae\in aH = bH$. Por outro lado,
		se $a\in bH$, então $a = bh$, para algum $h\in H$, logo, 
		$aH = (bh)H = b(hH) \overset{\textbf{2}}{=} bH$.
		%
		\par\vspace{0.4cm}
		\textbf{5.} Note que se $c\in aH\cap bH$, então $c\in aH$ e $c\in bH$, i.e., $cH = aH = bH$. Logo, se $aH\cap bH\neq\emptyset$, $aH = bH$.
		%
		\par\vspace{0.4cm}
		\textbf{6.} Temos que
		%
		\begin{equation*}
		    aH = bH 
		    \iff H = (a^{-1}b)H \overset{\textbf{2}}{\iff} a^{-1}b \in H
		\end{equation*}
		%
		\par\vspace{0.4cm}
		\textbf{7.} Se $aH = bH$, terminamos. Então, seja $f: aH\to bH$ 
		tal que $f(x) = b^{-1}ax$. Tomando $x_1$ e $x_2$ em $aH$, temos que
		$f(x_1) = f(x_2)$ implica $x_1=x_2$, logo $f$ é injetiva. 
		Além disso, tomando $a^{-1}by$ em $aH$, sendo $y\in bH$, temos
		$f(a^{-1}by) = y$, logo $f$ é sobrejetora. Consequentemente, 
		definimos uma bijeção de $aH$ em $bH$ e, portanto, $|aH| = |bH|$.
		%
		\par\vspace{0.4cm}
		\textbf{8.} Note que $aH = Ha \iff (aH)a^{-1} = H(aa^{-1}) = H$,
		i.e., se, e só se, $aHa^{-1} = H$.
		%
		\par\vspace{0.4cm}
		\textbf{9.} Se $aH$ é subgrupo de $G$, então $e\in aH$. Então, 
		$aH\cap eH \neq \emptyset$, logo $aH = eH = H$ e, por isso, 
		$a\in H$. Por outro lado, se $a\in H, aH = H$, que é subgrupo de $G$.
	\end{proof}
	%
	Com a Definição \ref{def classes laterais} e o 
	Lema \ref{propriedades}, podemos enunciar o Teorema \ref{lagrange}.
	%
	\begin{theorem}[Lagrange]
	\label{lagrange}
		Se $|G|<\infty$ e $H$ é subgrupo de $G$, então a ordem de $H$ divide 
		a ordem de $G$. Ademais, a quantidade de classes laterais à esquerda
		(direita) de $H$ em $G$ é $|G|/|H|$, ou seja, a ordem de um subgrupo
		divide a ordem do grupo. 
	\end{theorem}
	%
	\begin{proof}
		Sejam $a_1H, a_2H, \dots , a_rH$ as classes laterais distintas à
		esquerda de $H$ em $G$. Então, temos $aH = a_iH$ para todo $a$ em 
		$G$ e algum $i = 1,2,\dots,r$. Pela propriedade \textbf{1} do 
		Lema \ref{propriedades}, $a\in aH$. Logo, temos $aH = H = a_iH$ 
		e, daí:
		%
		\begin{align*}
		    G = \bigcup_{1\leq i\leq r}^{}a_iH \Rightarrow |G| 
		    = \sum_{1\leq i\leq r}^{}|a_iH| = \sum_{1\leq i\leq r}|H| 
		    = r|H|.
		\end{align*} 
		%
		Portanto, $|G|/|H| = r$.
	\end{proof}
	%
	Alguns resultados seguem como consequência imediata 
	do Teorema \ref{lagrange}.
	%
	\begin{corollary}
	\label{c2}
		Em um grupo finito, a ordem de cada elemento divide a ordem do grupo.
	\end{corollary}
	%
	\begin{proof}
		Como $|a| = |\langle a\rangle|$ e $\langle a \rangle$ é um subgrupo 
		de $G$, então $|a|$ divide $|G|$. 
	\end{proof}
	%
	\begin{corollary}
	\label{c3}
		Um grupo de ordem prima é cíclico.
	\end{corollary}
	%
	\begin{proof}
		Se $|G| = p$, $p$ primo, e $a\in G$, $a\neq e$, então 
		$|a| = |\langle a\rangle | \neq 1$ divide $|G|$, logo 
		$|a| = p = |G|$. Portanto, $G = \langle a\rangle$. 
	\end{proof}
	%
	\begin{corollary}
	\label{c4}
		Seja $G$ um grupo finito, $a\in G$. Então, $a^{|G|} = e.$
	\end{corollary}
	
	\begin{proof}
		Pelo Corolário \ref{c2}, $|G| = |a| k, k\in\mathbb{Z}_{+}^{*}$.
		Logo, $a^{|G|} = a^{|a|k} = e^k = e$. 
	\end{proof}
	%
	\begin{corollary}[Pequeno Teorema de Fermat]
	\label{c5}
		Para todo $a$ inteiro e para todo primo $p$, $a^p \pmod p = a\pmod p$.
	\end{corollary}
	%
	\begin{proof}
		Pelo algoritmo da divisão, podemos escrever $a = pm + r, 0\leq r < p$.
		Daí, $a\pmod p = r$ e só precisamos mostrar que $r^p\pmod p = r$.
		%
		\par\vspace{0.3cm} Se $r = 0$, então $p\mid a$ e é claro que 
		$r^p\pmod p = r\pmod p$.
		%
		\par\vspace{0.3cm} Suponha $r>0$. Então, 
		$r\in U(p) = \{1, 2, \dots, p-1\}$, sendo a operação de $U(p)$ a
		multiplicação módulo $p$. Note que $|U(p)| = p-1$. Daí, pelo
		Corolário \ref{c4}, temos que $r^{p-1}\pmod p = 1$ e,
		consequentemente, $r^p\pmod p = r$. 
	\end{proof}
	%
	\begin{theorem}
	\label{ordem de HK}
		Para dois subgrupos finitos $H$ e $K$ de um grupo $G$, seja 
		$HK = \left\{ hk \ | \ h\in H, k\in K \right\}$. Então, 
		$|HK| = |H||K|/|H\cap K|$.
	\end{theorem}
	%
	\begin{proof}
		Apesar do conjunto $HK$ ter $|H||K|$ produtos, nem todos eles são,
		necessariamente, distintos, isto é, podemos ter $hk = h'k'$ com 
		$h\neq h'$ e $k\neq k'$. Para determinar $|HK|$, devemos descobrir
		quantas vezes isso ocorre.
		%
		\par\vspace{0.3cm} Para todo $t$ em $H\cap K$, podemos escrever 
		$hk = (ht)(t^{-1}k)$, então cada elemento de $HK$ pode ser
		representado por pelo menos $|H\cap K|$ produtos. Mas note que 
		$hk = h'k'$ implica $t = h^{-1}h' = kk'^{-1}\in H\cap K$, logo 
		$h' = ht$ e $k' = t^{-1}k$. Consequentemente, cada elemento em $HK$
		pode ser representado por exatamente $|H\cap K|$ produtos. Daí, 
		$|HK| = |H||K|/|H\cap K|$, como afirmado.
	\end{proof}
	%
	\begin{example}
	Um exemplo interessante de isomorfismo é $S_3\cong GL(2,\mathbb{Z}_2)$. De fato, seja
	$\phi:GL(2,\mathbb{Z}_2)\to S_3$, e sejam ainda
	%
	\begin{align*}
        v_1 = 
        \begin{bmatrix}
            1 \\
            0
        \end{bmatrix}, 
        v_2 =
        \begin{bmatrix}
            0 \\ 
            1
        \end{bmatrix}, 
        v_3 =
        \begin{bmatrix}
            1 \\ 
            1
        \end{bmatrix}.
	\end{align*}
	%
	Todo elemento de $GL(2,\mathbb{Z}_2)$ permuta $v_1, v_2$ e $v_3$.
	%
	\par\vspace{0.3cm} Por exemplo, 
	$\begin{bmatrix}
	1 & 1 \\ 
	0 & 1
	\end{bmatrix}$
	manda $v_1$ em $v_1$, $v_2$ em $v_3$ e $v_3$ em $v_2$, ou seja, 
	nos dá a permutação $(v_2v_3)$.
	%
	\par\vspace{0.3cm} Além disso, note que matrizes diferentes em 
	$GL(2, \mathbb{Z}_2)$ nos dão permutações diferentes de 
	$\{v_1, v_2,v_3\}$, logo $\phi$ é injetiva.
	%
	\par\vspace{0.3cm} Por fim, como $|GL(2, \mathbb{Z}_2)| = 6 = |S_3|$,
	$\phi$ é sobrejetiva e, portanto, é bijetiva. Logo, é isomorfismo.
	%
	\par\vspace{0.3cm} Outra demonstração possível é notar que para $S_3$
	temos a tabela de multiplicação:
	%
	\begin{table}[H]
		\centering
		\noindent\begin{tabular}{c|cccccc}
			& 1 & (12) & (13) & (23) & (123) & (132) \\
			\hline
			1 & 1 & (12) & (13) & (23) & (123) & (132) \\
			(12) & (12) & 1 & (132) & (123) & (23) & (13) \\
			(13) & (13) & (123) & 1 & (132) & (12) & (23) \\
			(23) & (23) & (132) & (123) & 1 & (13) & (12) \\
			(123) & (123) & (13) & (23) & (12) & (132) & 1 \\
			(132) & (132) & (23) & (12) & (13) & 1 & (123) \\
		\end{tabular}
	\end{table}
	%
	\par \vspace{0.3cm} Fazendo
	%
	\begin{align*}
        1 = \begin{bmatrix}
            1 & 0 \\
            0 & 1
        \end{bmatrix}, 
        i = \begin{bmatrix}
            0 & 1 \\
            1 & 0
        \end{bmatrix}, 
        j = \begin{bmatrix}
            1 & 1 \\
            0 & 1
        \end{bmatrix}, 
        k = \begin{bmatrix}
            1 & 0 \\
            1 & 1
        \end{bmatrix},
        a = \begin{bmatrix}
            1 & 1 \\
            1 & 0
        \end{bmatrix}, 
        b = \begin{bmatrix}
            0 & 1 \\
            1 & 1
        \end{bmatrix}
	\end{align*}
	%
	obtemos a seguinte tabela de multiplicação para $GL(2, \mathbb{Z}_2)$:
	%
	\begin{table}[H]
		\centering
		\noindent\begin{tabular}{c|cccccc}
			& 1 & $i$ & $j$ & $k$ & $a$ & $b$ \\
			\hline
			1 & 1 & $i$ & $j$ & $k$ & $a$ & $b$ \\
			$i$ & $i$ & 1 & $b$ & $a$ & $k$ & $j$ \\
			$j$ & $j$ & $a$ & 1 & $b$ & $i$ & $k$ \\
			$k$ & $k$ & $b$ & $a$ & 1 & $j$ & $i$ \\
			$a$ & $a$ & $j$ & $k$ & $i$ & $b$ & 1 \\
			$b$ & $b$ & $k$ & $i$ & $j$ & 1 & $a$ \\
		\end{tabular}
	\end{table}
	%
	\par\vspace{0.3cm} Daí, podemos ver que os dois grupos são isomorfos 
	via $\phi:GL(2, \mathbb{Z}_2)\to S_3$ com
	%
	\begin{equation*}
	    1\mapsto 1, i\mapsto(12), j\mapsto(13), k\mapsto(23), a\mapsto(123),
	    b\mapsto(132).
	\end{equation*}
	\end{example}
	%
	\begin{definition}
	\label{def estabilizador}
		Seja $G$ um grupo de permutações de um conjunto $S$. Para cada 
		$i\in S$, seja $\stab_G(i) = \{ \phi\in G \ | \ \phi(i) = i\}$. 
		Chamamos $\stab_G(i)$ de estabilizador de $i$ em $G$. Usamos
		$|\stab_G(i)|$ para denotar o número de elementos em $\stab_G(i)$.
	\end{definition}
	%
	\begin{definition}
	\label{def orbita}
		Seja $G$ um grupo de permutações de um conjunto $S$. Para cada 
		$s\in S$, seja $\orb_G(s) = \{ \phi(s) \ | \ \phi\in G \}$. O conjunto
		$\orb_G(s)$ é um subconjunto de $S$ e é chamado de órbita de $s$ 
		sob $G$. Usamos $|\orb_G(s)|$ para denotar o número de elementos 
		em $\orb_G(s)$.
	\end{definition}
	%
	\begin{theorem}
	\label{orb-stab}
		Seja $G$ um grupo finito de permutações de um conjunto $S$. 
		Então, para todo $i$ em $G$, $|G| = |\orb_G(s)|\cdot|\stab_G(i)|$.
	\end{theorem}
	%
	\begin{proof}
		Pelo Teorema \ref{lagrange}, $|G|/|\stab_G(i)|$ é o número de
		classes laterais distintas à esquerda de $\stab_G(i)$ em $G$. 
		Logo, para provar o teorema, basta estabelecer uma correspondência
		biunívoca entre as classes laterais à esquerda de $\stab_G(i)$ e os
		elementos de $\orb_G(s)$.
		%
		\par\vspace{0.3cm} Para isso, definimos a correspondência $T$ que
		mapeia a classe lateral $\phi\stab_G(i)$ para $\phi(i)$. 
		%
		\par\vspace{0.3cm} Note que $T$ está bem definida, pois se 
		$\alpha\stab_G(i) = \beta\stab_G(i)$, então, pela propriedade 6 do 
		Lema \ref{propriedades}, $\alpha^{-1}\beta\in\stab_G(i)$, ou seja, 
		$(\alpha^{-1}\circ\beta)(i) = i$ e, portanto, $\alpha(i) = \beta(i)$.
		%
		\par\vspace{0.3cm} Por outro lado, se $\alpha(i) = \beta(i)$, então 
		$(\alpha^{-1}\circ\beta)(i) = i$. Logo, 
		$\alpha^{-1}\beta\in\stab_G(i)$ e, pela propriedade 6 do 
		Lema \ref{propriedades}, $\alpha\stab_G(i) = \beta\stab_G(i)$. 
		Logo, $T$ é injetiva.
		%
		\par\vspace{0.3cm} Por fim, seja $j\in\orb_G(s)$. Então 
		$\alpha(i) = j$ para algum $\alpha \in G$ e é claro que
		$T(\alpha\stab_G(i)) = \alpha(i) = j$, logo $T$ é sobrejetiva e,
		portanto, é bijetiva.
		%
		\par\vspace{0.3cm} Mostramos então que 
		$|G|/|\stab_G(i)| = |\orb_G(i)|$, ou seja, 
		$|G| = |\orb_G(i)|\cdot|\stab_G(i)|$. 
	\end{proof}
	%
	\begin{theorem}
	\label{part}
		O conjunto das órbitas dos elementos de $S$ sob um grupo 
		$G$ particionam $S$.
	\end{theorem}
	%
	\begin{proof}
		Sejam $a,b\in S$ quaisquer. É claro que $a\in\orb_G(a)$ e
		$b\in\orb_G(b)$. 
		%
		\par\vspace{0.3cm} Agora, seja $c\in\orb_G(a)\cap\orb_G(b)$. 
		Então, $c = \alpha(a) = \beta(b)$ para algum $\alpha$ 
		e algum $\beta$.
		%
		\par\vspace{0.3cm} Por um lado, temos que 
		$b = (\beta^{-1}\circ\alpha)(a)$. Logo, se $x\in\orb_G(b)$, então 
		$x = \gamma(b) = (\gamma\circ\beta^{-1}\circ\alpha)(a)\in\orb_G(a)$,
		para algum $\gamma$, ou seja, $\orb_G(b)\subseteq\orb_G(a)$.
		%
		\par\vspace{0.3cm} Por outro lado, temos que 
		$a = (\alpha^{-1}\circ\beta)(b)$. Daí, se $y\in\orb_G(a)$, então 
		$y = \sigma(a) = (\sigma\circ\alpha^{-1}\circ\beta)(b)\in\orb_G(b)$,
		para algum $\sigma$, ou seja, $\orb_G(b)\supseteq\orb_G(a)$. 
		%
		\par\vspace{0.3cm} Logo, as órbitas de elementos distintos ou 
		são iguais ou são disjuntas. Por isso, as órbitas particionam $S$, 
		mas não necessariamente particionam igualmente, i.e., não necessariamente
		têm a mesma ordem.
	\end{proof}
	%
	\begin{theorem}
	\label{rotacoes iso a S_4}
		O grupo de rotações de um cubo é isomorfo a $S_4$.
	\end{theorem}
	%
	\begin{proof}
		Como o grupo de rotações de um cubo tem a mesma ordem de $S_4$, 
		basta mostrar que o grupo de rotações é isomorfo a um subgrupo de
		$S_4$ (pela propriedade 5 do Teorema \ref{isomorfismos em grupos}).
		%
		\par\vspace{0.3cm} Para isso, note que um cubo possui 4 diagonais 
		e o grupo de rotações no cubo induz um grupo de permutações nas
		diagonais. Contudo, rotações diferentes não provocam, 
		necessariamente, permutações diferentes. Para ver que esse de fato 
		é o caso, vamos mostrar que todas as 24 permutações são obtidas a
		partir das rotações.
		%
		\par\vspace{0.3cm} Numerando as diagonais consecutivas com 
		1, 2, 3 e 4, podemos ver que existe uma rotação de 90 graus que nos 
		dá a permutação $\alpha = (1234)$; outra rotação de 90 graus em torno
		de um eixo perpendicular ao nosso primeiro eixo nos dá a 
		permutação $\beta = (1423)$.
		%
		\par\vspace{0.3cm} Logo, o grupo de permutações induzido pelas
		rotações contém o subgrupo 
		%
		\begin{equation*}
		    \{e, \alpha, \alpha^2, \alpha^3, \beta^2, \beta^2\alpha,
		    \beta^2\alpha^2, \beta^2\alpha^3\}
		\end{equation*}
		%
		de 8 elementos e também contém $\alpha\beta$, que tem ordem 3.
		%
		\par\vspace{0.3cm} Portanto, o grupo de permutações induzido pelas
		rotações tem ordem 24 (já que sua ordem deve ser divisível por 8 e 
		por 3) sendo, por isso, isomorfo a $S_4$, uma vez que conseguimos
		obter todas as permutações das diagonais a partir das rotações
		$\alpha$ e $\beta$ e suas combinações. 
	\end{proof}
    %
\section{Produto direto externo de grupos}
\label{sec-prod-direto-grupos}
    %
    Vamos definir o produto direto externo de grupos, uma maneira de
    obter novos grupos a partir de grupos já conhecidos.
    %
	\begin{definition}
	\label{def prod direto externo}
		Seja $\{G_1, G_2, \dots, G_n\}$ uma coleção finita de grupos. 
		O produto direto externo de $G_1, G_2, \dots, G_n$, denotado por
		$\displaystyle{G_1\oplus G_2\oplus\cdots\oplus G_n}$, é o conjunto 
		de todas as $n$-tuplas para as quais o $i$-ésimo componente é um
		elemento de $G_i$ e a operação é efetuada componente a componente.
	\end{definition}
	%
	\begin{remark}
	    Ao longo do texto, optamos por utilizar a notação $G\oplus H$ para
	    denotar o produto direto de $G$ e $H$. Entretanto, é mais comum
	    encontrar a notação $G\times H$.
	\end{remark}
	%
	\begin{theorem}
		Seja $\displaystyle{H = \bigoplus_{i=1}^{n}G_i}$, com $G_i$ grupos.
		Então, $H$ é também um grupo.
	\end{theorem}
	%
	\begin{proof}
		Sejam $a_i, b_i\in G_i$. Da Definição \ref{def prod direto externo},
		sabemos que:
		%
		\begin{align*}
		    (a_1, a_2, \dots, a_n)(b_1, b_2, \dots, b_n) 
		    = 
		    (a_1b_1, a_2b_2, \dots, a_nb_n)\in H 
		    \text{ pois cada um dos } G_i \text{ é grupo.}
		\end{align*}
		%
		\par Portanto $H$ é fechado. Além disso, temos que:
		%
		\begin{align*}
		    (a_1, a_2, \dots, a_n)[(b_1, b_2, \dots, b_n)(c_1, c_2, \dots, c_n)] 
		    &= (a_1, a_2, \dots, a_n)(b_1c_1, b_2c_2, \dots, b_nc_n) \\ 
		    &= ((a_1b_1)c_1, (a_2b_2)c_2, \dots, (a_nb_n)c_n) \\ 
		    &= (a_1b_1, a_2b_2, \dots, a_nb_n)(c_1, c_2, \dots, c_n) \\ 
		    &= [(a_1, a_2, \dots, a_n)(b_1, b_2, \dots, b_n)](c_1, c_2, \dots, c_n),
		\end{align*}
		%
		logo a associatividade se mantém em $H$.
		%
		\par\vspace{0.3cm} Podemos ver que a identidade de $H$ é 
		$(e_1, e_2, \dots, e_n)$, sendo $e_i$ a identidade de $G_i$.
		%
		\par\vspace{0.3cm} Por fim, basta notar que o inverso de todo 
		elemento $(a_1, a_2, \dots, a_n)$ em $H$ é dado por
		$(a_1^{-1}, a_2^{-1}, \dots, a_n^{-1})$.
	\end{proof}
	%
	\begin{remark}
		Segue da Definição \ref{def prod direto externo} que 
		$(g_1, g_2, \dots, g_n)^k = (g_1^k, g_2^k, \dots, g_n^k)$, ou seja, 
		a potência ``se distribui'' nas entradas.
	\end{remark}
	%
	\begin{theorem}
		Todo grupo de ordem 4 é isomorfo a $\mathbb{Z}_4$ 
		ou $\mathbb{Z}_2\oplus\mathbb{Z}_2$.
	\end{theorem}
	%
	\begin{proof}
		Seja $G = \{e, a, b, ab\}$. Se $G$ não é cíclico, então 
		$|a| = |b| = |ab| = 2$, pelo Teorema \ref{lagrange}. 
		Logo, podemos definir a correspondência 
		$e\mapsto (0,0)$, $a\mapsto (1,0), b\mapsto (0,1)$, $ab\mapsto (1,1)$, que é um
		isomorfismo de $G$ em $\mathbb{Z}_2\oplus\mathbb{Z}_2$.
		%
		\par\vspace{0.3cm} Se $G$ é cíclico, então é isomorfo a
		$\mathbb{Z}_4$, pois todo grupo cíclico de ordem $n$ é isomorfo 
		a $\mathbb{Z}_n$. 
	\end{proof}
	%
	\begin{theorem}
	\label{ordem}
		Seja 
		%
		\begin{equation*}
		\bigoplus_{i \leq n} G_i = \left\{(g_1, g_2, \dots , g_n) \ | \ g_i\in G_i \right\}. 
		\end{equation*}
		%
		Então,
		$|(g_1, g_2, \dots , g_n)| = \mmc(|g_1|, |g_2|, \dots , |g_n|)$.
	\end{theorem}
	%
	\begin{proof}
		Sejam $s = \mmc(|g_1|, |g_2|, \dots, |g_n|)$ e 
		$t = |(g_1, g_2, \dots , g_n)|$.
		%
		\par\vspace{0.3cm} Por um lado, temos que:
		%
		\begin{align*}
		    (g_1, g_2, \dots , g_n)^s 
		    = (g_1^s, g_2^s, \dots , g_n^s) 
		    = (e_1, e_2, \dots , e_n)
		\end{align*}
		%
		pois $s$ é múltiplo de cada $|g_i|$. Logo, $t\leq s$.
		%
		\par\vspace{0.3cm} Por outro lado, temos que:
		%
		\begin{align*}
		    (g_1^t, g_2^t, \dots , g_n^t) 
		    = (g_1, g_2, \dots , g_n)^t 
		    = (e_1, e_2, \dots , e_n)
		\end{align*}
		%
		logo $t$ é um múltiplo comum de $|g_1|, |g_2|, \dots , |g_n|$.
		Portanto, $s\leq t$ e concluímos que $s = t$, pelo 
		Princípio da Tricotomia. 
	\end{proof}
	%
	\begin{theorem}
	\label{crit}
		Sejam $G$ e $H$ grupos cíclicos finitos. Então, $G\oplus H$ é 
		cíclico se, e só se, $\mdc(|G|, |H|) = 1$, isto é, $|G|$ e $|H|$ 
		são primos entre si.
	\end{theorem}
	%
	\begin{proof}
		Sejam $|G| = m$ e $|H| = n$. Daí, $|G\oplus H| = mn$ 
		(pelo Princípio Fundamental da Contagem). Suponha que $G\oplus H$ 
		é cíclico e que $\langle (g, h) \rangle = G\oplus H$. Suponha que
		$\mdc(m,n) = d$. Como 
		$(g, h)^{mn/d} = ((g^m)^{n/d} , (h^n)^{m/d}) = (e,e)$, 
		temos que $mn = |(g, h)| \leq mn/d$ e, portanto, $d = 1$.
		%
		\par\vspace{0.3cm} Agora, suponha que $\mdc(m, n) = 1$ e 
		$G = \langle g  \rangle $ e $H = \langle h \rangle$. Então, 
		$|(g,h)|=\mmc(|g|,|h|)=\mmc(m,n)\overset{\star}{=}mn=|G\oplus H|$, 
		em que em $\star$ usamos o fato de que $\mmc(a,b)\mdc(a,b) = ab$ 
		para quaisquer $a,b\in\mathbb{Z}$.
		Logo, $G\oplus H = \langle (g, h) \rangle$. 
	\end{proof}
	%
	\begin{corollary}
	\label{C1}
		Um produto externo direto $G_1\oplus G_2\oplus\cdots\oplus G_n$ 
		de um número finito de grupos cíclicos finitos é cíclico 
		se, e só se, $|G_i|$ e $|G_j|$ são relativamente primos quando 
		$i\neq j$.
	\end{corollary}
	%
	\begin{proof}
		Para $n = 2$ temos o Teorema \ref{crit}. Suponha que o corolário
		vale para $n = k>2$. Então, para $n = k+1$, temos
		%
		\begin{equation*}
		    \underbrace{G_1\oplus G_2\oplus\cdots\oplus G_k}_{G}\oplus\underbrace{ G_{k+1}}_{H}. 
		\end{equation*}
		%
		Sejam $|G_i| = n_i$ para 
		$i = 1, 2, \dots , k + 1$, $\displaystyle{G = \bigoplus_{i=1}^{k}G_i}$
		de modo que $\displaystyle{|G| = \prod_{i =1}^{k}n_i}$ e 
		$H = G_{k+1}$. Daí, 
		$\displaystyle{|G\oplus H| = \prod_{i = 1}^{k+1}n_i}$.
		%
		\par\vspace{0.3cm} Suponha que 
		$G\oplus H = \langle (g_1, g_2, \dots , g_k, h) \rangle$ e que
		$\displaystyle{\mdc(|G|,|H|)=\mdc\Bigg(\prod_{i=1}^{k}n_i,n_{k+1}\Bigg)=d}$. Daí, temos:
		%
		\begin{align*}
		    (g_1, g_2, \dots, g_k, h)^{\displaystyle{\frac{1}{d}\prod_{i=1}^{k+1}n_i}} 
		    = \Bigg((g_1^{n_1})^{\displaystyle{\frac{1}{d}\prod_{i=2}^{k+1}n_i}},\dots,(h^{n_{k+1}})^{\displaystyle{\frac{1}{d}\prod_{i = 1}^{k}n_i}} \Bigg) 
		    = (e, e, \dots, e).
		\end{align*}
		%
		Logo, $\displaystyle{\prod_{i=1}^{k+1}n_i=|(g_1, g_2,\dots, g_k, h)|
		\leq\frac{1}{d}\prod_{i=1}^{k+1}n_i}$, ou seja, $d = 1$.
		%
		\par\vspace{0.3cm} Como nenhum dos $n_i$ compartilha fator primo
		para $i = 1, 2, 3, \dots, k$ e $d = 1$ implica que $n_{k+1}$ não
		compartilha fator primo com nenhum dos $n_i$, concluímos que todos 
		os $n_i$ são primos entre si para $i = 1, 2, 3, \dots, k, k+1$.
		%
		\par\vspace{0.3cm} Agora, suponha que
		$\displaystyle{\mdc\Bigg(\prod_{i=1}^{k}n_i , n_{k+1}\Bigg)} = 1$.
		Daí, pela hipótese de indução, sabemos que todos os $n_i$ são
		relativamente primos para $1\leq i\leq k$. Pela nossa hipótese,
		$n_{k+1}$ é relativamente primo ao produto dos $n_i$, que são primos
		entre si, logo $n_{k+1}$ é relativamente primo a todos os $n_i$, 
		isto é, $\mdc(n_i, n_j) = 1$ para $i\neq j$.
		%
		\par\vspace{0.3cm} Tome $G = \langle (g_1, g_2, \dots, g_k) \rangle$ 
		e $H = \langle h \rangle$. Então, temos:
		%
		\begin{align*}
		    |(g_1, g_2, \dots, g_k, h)| 
		    = \mmc(|g_1|, |g_2|, \dots, |g_k|, |h|) 
		    = \mmc(n_1, n_2, \dots, n_k, n_{k+1}) 
		    \overset{\star}{=} \prod_{i=1}^{k+1}n_i = |G\oplus H|
		\end{align*} 
		%
		\par\vspace{0.3cm} em que $\star$ ressalta o fato de que como nenhum
		dos $n_i$ compartilha fator primo (já que o mdc de cada par distinto 
		é 1), então o mmc será dado pelo produto dos $n_i$.
		%
		\par\vspace{0.3cm} Portanto, 
		$G\oplus H = \langle (g_1, g_2, \dots, g_k, h) \rangle$.
	\end{proof}
	%
	\begin{corollary}
	\label{C2}
		Seja $\displaystyle{m = \prod_{i=1}^{k}n_i}$. Então $\mathbb{Z}_m$ 
		é isomorfo a
		$\mathbb{Z}_{n_1}\oplus\mathbb{Z}_{n_2}\oplus\cdots\oplus\mathbb{Z}_{n_k}$ 
		se, e só se, $n_i$ e $n_j$ são relativamente primos quando $i\neq j$.
	\end{corollary}
	%
	\begin{proof}
		Do Corolário \ref{C1}, sabemos que
		$\displaystyle{\bigoplus_{i=n_1}^{n_k} \mathbb{Z}_{n_i}}$ é cíclico
		se, e só se, $|\mathbb{Z}_{n_i}|$ e $|\mathbb{Z}_{n_j}|$ são primos
		entre si para $n_i\neq n_j$, ou seja, se, e só se, $n_i$ e $n_j$ são
		primos entre si para $i\neq j$. 
		%
		\par\vspace{0.3cm} Além disso, 
		$\Bigg|\displaystyle{\bigoplus_{i=n_1}^{n_k} \mathbb{Z}_{n_i}\Bigg| 
		= \prod_{i = 1}^{k}n_i = m = |\mathbb{Z}_m|}$.
		%
		\par\vspace{0.3cm} Logo, como todo grupo cíclico finito de ordem 
		$n$ é isomorfo a $\mathbb{Z}_n$, então
		$\displaystyle{\bigoplus_{i=n_1}^{n_k} \mathbb{Z}_{n_i} 
		\cong \mathbb{Z}_{n_1n_2\cdots n_k}} = \mathbb{Z}_m$ se, e só se,
		$\mdc(n_i, n_j) = 1$ para $i\neq j$. 
	\end{proof}
	%
	\begin{remark}
        Em \cite{Livro-do-Fraleigh}, o Teorema \ref{crit} e o 
        Corolário \ref{C1} são enunciados (de certo modo) como o 
        seguinte teorema: o grupo $\mathbb{Z}_m\oplus\mathbb{Z}_n$ é 
        cíclico e isomorfo a $\mathbb{Z}_{mn}$ se, e só se, $\mdc(m,n) = 1$.
	\end{remark}
	%
	\begin{lemma}
	\label{lema1}
		Se $\mdc(s, t) = 1$, então 
		$a\pmod {st} = b\pmod {st} \iff 
		\begin{cases}
		a\pmod s = b\pmod s \\ a\pmod t = b\pmod t
		\end{cases}$
	\end{lemma}
	%
	\begin{proof}
		Sejam 
		%
		\begin{align*}
		    a - b = \prod_{i = 1}^{n}p_i^{\alpha_i}, \hspace{0.2cm} s 
		    = \prod_{i = 1}^{n}p_i^{\beta_i}, \hspace{0.2cm} t 
		    = \prod_{i=1}^{n}p_i^{\gamma_i}, \hspace{0.4cm} p_i
		    \text{ primos }, \alpha_i, \beta_i, \gamma_i\in\N.
		\end{align*}
		%
		($\Rightarrow$) Se $a\pmod {st} = b\pmod {st}$, então $st\mid a-b$. 
		Logo:
		%
		\begin{align*}
            \begin{cases}
                s \mid a-b \\
                t \mid a-b
            \end{cases}
		\Rightarrow
            \begin{cases}
                a\pmod s = b\pmod s \\
                a\pmod t = b\pmod t
            \end{cases}
		\end{align*}
		%
		\par\vspace{0.3cm}($\Leftarrow$) Se
		%
		\begin{align*}
            \begin{cases}
                a\pmod s = b\pmod s \\
                a\pmod t = b\pmod t
            \end{cases}
		\end{align*}
		%
		\par\vspace{0.3cm} então $s\mid a-b$ e $t \mid a-b$. 
		Isso é equivalente a dizer que $\beta_i\leq\alpha_i$ e
		$\gamma_i\leq\alpha_i$. Como $\mdc(s,t) = 1$, então 
		$\min(\beta_i, \gamma_i) = 0$, logo 
		$\beta_i + \gamma_i \leq \alpha_i$.
		%
		\par\vspace{0.3cm} Mas isso implica que $st\mid a-b$, ou seja, 
		$a\pmod {st} = b\pmod {st}$.
	\end{proof}
	%
	\begin{lemma}
	\label{lema2}
		$\mdc(a,bc) = 1 \iff \mdc(a,b) = 1 = \mdc(a,c)$.
	\end{lemma}
	%
	\begin{proof}
		Sejam 
		%
		\begin{align*}
        a = \prod_{i = 1}^{n}p_i^{\alpha_i},\hspace{0.2cm}
		b = \prod_{i = 1}^{n}p_i^{\beta_i}, \hspace{0.2cm}
		c = \prod_{i = 1}^{n}p_i^{\psi_i}, \hspace{0.4cm}
		\alpha_i, \beta_i, \psi_i\in\N.
		\end{align*}
		%
		\par\vspace{0.3cm} Note que mostrar que 
		$\mdc(a,bc) = 1 \iff \mdc(a,b) = 1 = \mdc(a,c)$ é equivalente a
		mostrar que 
		$\min(\alpha_i,\beta_i+\psi_i)=0\iff\min(\alpha_i,\beta_i)=0
		=\min(\alpha_i,\psi_i)$.
		%
		\par\vspace{0.3cm} Note que $\min(\alpha_i, \beta_i + \psi_i) = 0$ 
		se, e só se, $\alpha_i = 0$ ou $\beta_i = \psi_i = 0$. Em ambos os
		casos, temos $\min(\alpha_i, \beta_i) = 0 = \min(\alpha_i, \psi_i)$.
	\end{proof}
	%
	\par\vspace{0.3cm} Antes de prosseguir para o próximo teorema, uma
	definição é necessária.
	%
	\begin{definition}
	\label{def U_k(n)}
		$U_k(n) = \{x\in U(n) \ | \ x\pmod k = 1 \}$.
	\end{definition}
	%
	\begin{theorem}
	\label{produto direto}
		Suponha que $s$ e $t$ são primos entre si. Então, 
		$U(st)\cong U(s)\oplus U(t)$. Além disso, $U_s(st)\cong U(t)$ e
		$U_t(st)\cong U(s).$
	\end{theorem}
	%
	\begin{proof}
		Um isomorfismo ($\phi_1$) de $U(st)$ em $U(s)\oplus U(t)$ é 
		$x\mapsto (x\pmod s, x\pmod t)$.
		%
		\par\vspace{0.3cm} Um isomorfismo ($\phi_2$) de $U_s(st)$ em $U(t)$ 
		é $x\mapsto x\pmod t$.
		%
		\par\vspace{0.3cm} Um isomorfismo ($\phi_3$) de $U_t(st)$ em $U(s)$ 
		é $x\mapsto x\pmod s$.
		%
		\par\vspace{0.3cm} Vamos mostrar que $\phi_1$ de fato é isomorfismo.
		%
		\vspace{0.3cm}\par Note que se $x, y \in U(st)$, então: 
		%
		\begin{align*} 
		    \phi_1(xy) 
		    = (xy\text{ }(\mathrm{mod} s), xy\text{ }(\mathrm{mod} t)) 
		    = [(x\text{ }\mathrm{mod} s)(y\text{ }\mathrm{mod} s), 
		    (x\text{ }\mathrm{mod} t)(y\text{ }\mathrm{mod} t)] 
		    = \phi_1(x)\phi_1(y)
		\end{align*} 
		%
		\par\vspace{0.3cm} logo $\phi_1$ preserva a operação.
		%
		\par\vspace{0.3cm} Agora, se $(x\pmod s, x\pmod t) 
		= (y\pmod s, y\pmod t)$, então 
		%
		\begin{align*}
            \begin{cases}
                x\pmod s = y\pmod s \\ 
                x\pmod t = y\pmod t
            \end{cases} 
        \overset{\text{Lema \ref{lema1}}}{\iff } x\pmod {st} = y\pmod {st} 
		\end{align*}
		%
		\par\vspace{0.3cm} logo $\phi_1$ é injetora.
		%
		\par\vspace{0.3cm} Agora vamos mostrar que $\phi_1$ é sobrejetora.
		%
		\par\vspace{0.3cm} Seja $(a,b)\in U(s)\oplus U(t)$. Então, 
		$\mdc(a,s) = 1 = \mdc(b,t)$. Como 
		$\mdc(s,t) = 1, \exists q_1, q_2\in \mathbb{Z}: sq_1 + tq_2 = 1$.
		Logo, $\mdc(t,q_1) = 1 = \mdc(s,q_2)$.
		%
		\par\vspace{0.3cm} Tome $z = bsq_1 + atq_2$. Suponha que um primo 
		$p$ divide $st$. Então $p\mid s$ ou $p\mid t$. Se $p\mid s$, 
		então $p\mid bsq_1$ mas $p\nmid atq_2$ pois $\mdc(s,a) = 1 = \mdc(s,q_2)$,
		o que implica que $\mdc(s,aq_2) = 1$ pelo Lema \ref{lema2}. 
		Além disso, $\mdc(s,t) = 1$, logo $\mdc(s,atq_2) = 1$ pelo 
		Lema \ref{lema2} novamente.
		%
		\par\vspace{0.3cm} Então, se $p\mid s$, $p\nmid z$.
		%
		\par\vspace{0.3cm} Se $p\mid t$, então $p\mid atq_2$ mas 
		$p\nmid bsq_1$ pois $\mdc(b,t) = 1 = \mdc(q_1,t)$ logo 
		$\mdc(t,bq_1) = 1$ pelo Lema \ref{lema2} e, como $\mdc(t,s) = 1$,
		então $\mdc(t,bsq_1) = 1$ pelo Lema \ref{lema2} novamente.
		%
		\par\vspace{0.3cm} Então, se $p\mid t$, $p\nmid z$. 
		%
		\par\vspace{0.3cm} Logo, nenhum primo $p$ que divide $st$ divide $z$.
		Daí, $\mdc(z,st) = 1$, ou seja, $z\in U(st)$.
		%
		\par\vspace{0.3cm} Finalmente, tendo em mente que $sq_1 + tq_2 = 1$,
		obtemos:
		%
		\begin{align*}
		    \phi_1(z) 
		    &= 
		    \Big((bsq_1+atq_2)\text{ }\mathrm{mod} s,(bsq_1+atq_2)
		    \text{ }\mathrm{mod} t\Big) \\ 
		    &= ( atq_2\pmod s, bsq_1\pmod t) \\ 
		    &= \Big((a - asq_1)\text{ }\mathrm{mod} s , (b - btq_2)
		    \text{ }\mathrm{mod} t \Big) \\ 
		    &= (a\pmod s, b\pmod t)
		\end{align*}
		%
		\par\vspace{0.3cm} logo $\phi_1$ é sobrejetora.
		%
		\vspace{0.3cm}\par Vamos mostrar que $\phi_2$ de fato é isomorfismo.
		%
		\par\vspace{0.3cm} Sejam $x, y\in U_s(st)$ quaisquer. Então, temos:
		%
		\begin{align*}
		    \phi_2(xy) = 
		    (xy)\text{ }(\mathrm{mod}t)=(x\text{ }\mathrm{mod}t)(y\text{ }\mathrm{mod}t) 
		    = \phi_2(x)\phi_2(y)
		\end{align*}
		%
		\par\vspace{0.3cm} logo $\phi_2$ preserva a operação.
		%
		\par\vspace{0.3cm} Agora, suponha $x,y\in U_s(st)$ com 
		$\phi_2(x) = \phi_2(y)$. Sabemos que 
		$x\text{ }\mathrm{mod} s = 1 = y\text{ }\mathrm{mod} s$. Logo, temos
		%
		\begin{align*}
            \begin{cases}
                \phi_2(x) = \phi_2(y) \\
                x\pmod s = y\pmod s
            \end{cases}
            &\iff
            \begin{cases}
                x\pmod t = y\pmod t \\
                y\pmod s = y\pmod s
            \end{cases} \\
            &\overset{\text{Lema } \ref{lema1}}{\iff } x\pmod {st} = y\pmod{st},
		\end{align*}
		%
		\par\vspace{0.3cm} logo $\phi_2$ é injetora.
		%
		\par\vspace{0.3cm} Agora, vamos mostrar que $\phi_2$ é sobrejetora.
		%
		\par\vspace{0.3cm} Seja $b\in U(t)$. Então, $\mdc(b,t) = 1$. 
		Além disso, sabemos que 
		$\exists q_1, q_2\in\mathbb{Z}: sq_1 + tq_2 = 1$, pois 
		$\mdc(s,t) = 1$ e também sabemos que $\mdc(t,q_1) = 1 = \mdc(s, q_2)$.
		%
		\par\vspace{0.3cm} Tome $z = bsq_1 + tq_2$ e suponha que um primo 
		$p$ divide $st$. Então, $p\mid s$ ou $p\mid t$. Se $p\mid s$, então
		$p\mid bsq_1$ mas $p\nmid tq_2$, pois $\mdc(s,t) = 1 = \mdc(s,q_2)$,
		logo $\mdc(s, tq_2) = 1$ pelo Lema \ref{lema2}.
		%
		\par\vspace{0.3cm} Logo, se $p\mid s$, $p\nmid z$.
		%
		\par\vspace{0.3cm} Se $p\mid t$, então $p\mid tq_2$ mas 
		$p\nmid bsq_1$, pois $\mdc(b,t) = 1 = \mdc(q_1, t) = \mdc(s,t)$, 
		logo $\mdc(t, bsq_1) = 1$ pelo Lema \ref{lema2}.
		%
		\par\vspace{0.3cm} Logo, se $p\mid t$, $p\nmid z$.
		%
		\par\vspace{0.3cm} Portanto, $\mdc(z, st) = 1$, pois nenhum primo 
		que divida $st$ divide $z$. Além disso, note que 
		%
		\begin{align*}
		    z \pmod s 
		    = tq_2 \pmod s 
		    = 1 - sq_1 \pmod s = 1.
		\end{align*} 
		%
		\par\vspace{0.3cm} Logo, $z\in U_s(st)$. 
		%
		\par\vspace{0.3cm} Finalmente, tendo em mente que 
		$sq_1 + tq_2 = 1$, temos
		%
		\begin{align*}
		\phi_2(z) 
		&= (bsq_1 + tq_2)\text{ }\mathrm{mod} t \\ 
		&= bsq_1\text{ }\mathrm{mod} t \\ 
		&= (b - btq_2)\text{ }\mathrm{mod} t \\ 
		&= b\text{ }\mathrm{mod} t
		\end{align*}
		%
		\par\vspace{0.3cm}logo $\phi_2$ é sobrejetora.
		%
		\vspace{0.3cm}\par Vamos mostrar que $\phi_3$ de fato é isomorfismo.
		%
		\par\vspace{0.3cm} Sejam $x,y\in U_t(st)$ quaisquer. Então, temos:
		%
		\begin{align*}
		    \phi_3(xy) 
		    = (xy)\pmod s 
		    = (x\pmod s)(y\pmod s) 
		    = \phi_3(x)\phi_3(y)
		\end{align*}
		%
		\par\vspace{0.3cm} logo $\phi_3$ preserva a operação.
		%
		\par\vspace{0.3cm} Agora, suponha $x,y\in U_t(st)$ com 
		$\phi_3(x) = \phi_3(y)$. Sabemos que 
		$x\pmod t = 1 = y\pmod t$.
		Logo, temos:
		%
		\begin{align*}
            \begin{cases}
                \phi_3(x) = \phi_3(y) \\
                x\pmod t = y\pmod t
            \end{cases}
            &\iff
            \begin{cases}
                x\pmod s = y\pmod s \\
                y\pmod t = y\pmod t
            \end{cases} \\
            &\overset{\text{Lema }\ref{lema1}}{\iff}x\pmod {st} 
            = y\pmod {st},
		\end{align*}
		%
		\par\vspace{0.3cm} logo $\phi_3$ é injetora. 
		%
		\par\vspace{0.3cm} Agora, vamos mostrar que $\phi_3$ é sobrejetora.
		%
		\par\vspace{0.3cm} Seja $b\in U(s)$. Então, $\mdc(b,s) = 1$.
		Além disso, sabemos que 
		$\exists q_1, q_2\in\mathbb{Z}: sq_1 + tq_2 = 1$, pois 
		$\mdc(s,t) = 1$ e também sabemos que $\mdc(t,q_1) = 1 = \mdc(s, q_2)$.
		%
		\par\vspace{0.3cm} Tome $z = sq_1 + btq_2$ e suponha que um primo 
		$p$ divide $st$. Então, $p\mid s$ ou $p\mid t$. Se $p\mid s$, então
		$p\mid sq_1$ mas $p\nmid btq_2$, pois 
		$\mdc(s,t) = 1 = \mdc(s,q_2) = \mdc(s,b)$, logo $\mdc(s, btq_2) = 1$
		pelo Lema \ref{lema2}.
		%
		\par\vspace{0.3cm} Logo, se $p\mid s$, $p\nmid z$.
		%
		\par\vspace{0.3cm} Se $p\mid t$, então $p\mid btq_2$ mas 
		$p\nmid sq_1$, pois $\mdc(q_1, t) = 1 = \mdc(s,t)$, logo 
		$\mdc(t, sq_1) = 1$ pelo Lema \ref{lema2}.
		%
		\par\vspace{0.3cm} Logo, se $p\mid t$, $p\nmid z$.
		%
		\par\vspace{0.3cm} Portanto, $\mdc(z, st) = 1$, pois nenhum 
		primo que divida $st$ divide $z$. Além disso, note que 
		%
		\begin{align*}
		    z\pmod t 
		    = sq_1\pmod t 
		    = 1 - tq_2 \pmod t 
		    = 1.
		\end{align*} 
		%
		\par\vspace{0.3cm} Logo, $z\in U_t(st)$.
		%
		\par\vspace{0.3cm} Finalmente, tendo em mente que 
		$sq_1 + tq_2 = 1$, temos
		%
		\begin{align*}
		\phi_3(z) 
		&= (sq_1 + btq_2)\pmod s \\ 
		&= btq_2\pmod s \\ 
		&= (b - bsq_1)\pmod s \\ 
		&= b\pmod s
		\end{align*}
		%
		\par\vspace{0.3cm}logo $\phi_3$ é sobrejetora.
	\end{proof}
	%
	\begin{corollary}
	\label{produto direto de U(m)}
		Seja $\displaystyle{m = \prod_{i = 1}^{k}n_i}$ com 
		$\mdc(n_i, n_j) = 1$ quando $i\neq j$. Então,
		$\displaystyle{U(m)\cong\bigoplus_{i = 1}^{k} U(n_i)}$.
	\end{corollary}
	%
	\begin{proof}
		Para o caso $k = 2$ temos o Teorema \ref{produto direto}. Suponha então que o corolário 
		vale para $i = k-1$, sendo $k$ um inteiro maior que 3. Vamos mostrar, por indução, que o 
		corolário vale para $i = k$.
		Por hipótese, sabemos que $U(n_1n_2\cdots n_{k-1})\cong \displaystyle{\bigoplus_{i=1}^{k-1}U(n_i)}$. 
		Como $\mdc \Bigg(\displaystyle{ \prod_{i=1}^{k-1}n_i }, n_k\Bigg) = 1$, devido às condições do 
		enunciado, temos:
		%
		\begin{align*}
		U(m) \cong U\Bigg(\displaystyle{\prod_{i=1}^{k-1}n_i}\Bigg)\oplus U(n_k)
		     \cong \bigoplus_{i=1}^{k-1}U(n_i)\oplus U(n_k)
		     \cong \bigoplus_{i=1}^{k}U(n_i),
		\end{align*}
		%
		como afirmado.
	\end{proof}

\section{Subgrupos normais e grupo quociente}
\label{sec-subgrupos-normais}
    %
    \begin{definition}
    \label{def:subgrupo-normal}
		Um subgrupo $H$ de um grupo $G$ é dito subgrupo \textbf{normal} de $G$ se 
		$aH = Ha$ para todo $a$ em $G$. Denotamos isso por $H\vartriangleleft G$.
	\end{definition}
	%
	\begin{theorem}
	\label{condicao}
		Um subgrupo $H$ de $G$ é normal em $G$ se, e somente se, $xHx^{-1}\subseteq H, \forall x\in G$. 
	\end{theorem} 
	%
	\begin{proof}
		Se $H$ é normal em $G$, então para todo $x\in G$ e $h\in H$ existe $h'\in H$ tal que 
		$xh = h'x$, ou seja, $xhx^{-1} = h'$ e, portanto, $xHx^{-1}\subseteq H.$
		\par\vspace{0.3cm} Agora, suponha que $xHx^{-1}\subseteq H$. Tomando $x=a$, obtemos 
		$aH\subseteq Ha$ e, tomando $x = a^{-1}$, obtemos $Ha\subseteq aH$, logo $aH = Ha$ e, 
		portanto, $H$ é normal em $G$.
	\end{proof}
	%
	Note que o Teorema \ref{condicao} é uma versão mais fraca da propriedade 8 
	do Lema \ref{propriedades}. Por exemplo, a partir do Teorema \ref{condicao}, temos que todo 
	subgrupo de um grupo abeliano é normal. Nesse caso, $ah = ha$ para $a$ no grupo e $h$ no subgrupo. 
	\par\vspace{0.3cm} A partir dos grupos normais, podemos definir o quociente de dois grupos.
	%
	\begin{theorem}
	\label{quociente}
		Seja $G$ um grupo e $H$ um subgrupo normal de $G$. O conjunto $G/H = \{ aH \ | \ a\in G \}$ 
		é um grupo sob a operação $(aH)(bH) = abH$. Ademais, o grupo $G/H$ é chamado grupo fator 
		ou grupo quociente de $G$ por $H$.
	\end{theorem}
	%
	\begin{proof}
		Primeiro, vamos mostrar que a operação é bem definida. Para isso, suponha que para alguns 
		elementos $a, a', b, b'$ em $G$, $aH = a'H$ e $bH = b'H$. Daí, sabemos que $a' = ah_1$ e 
		$b' = bh_2$, para alguns $h_1, h_2$ em $H$. Logo, $a'b'H = ah_1bh_2H = ah_1bH = ah_1Hb = aHb = abH$. 
		\par\vspace{0.3cm} Por fim, basta notar que $a^{-1}H$ é o inverso, $eH = H$ é a identidade e 
		$(aHbH)cH = (ab)HcH = (ab)cH = a(bc)H = aH(bcH) = aH(bHcH)$. Logo, $G/H$ é grupo.
	\end{proof}
	%
	\begin{remark}
		O grupo quociente de $G$ por $H$ é o grupo formado pelo conjunto das classes laterais à esquerda 
		(ou direita) de $H$. Eles são úteis pois estudando um grupo quociente podemos obter informações 
		acerca do grupo em si.
	\end{remark}
	%
	\begin{theorem}
	\label{teorema G/Z}
		Sejam $G$ um grupo e $Z(G)$ o centro de $G$. Se $G/Z(G)$ é cíclico, então $G$ é abeliano.
	\end{theorem}
	%
	\begin{proof}
		Note que $G$ ser abeliano é equivalente a $Z(G) = G$, logo basta mostrar que o único elemento 
		de $G/Z(G)$ é a classe lateral identidade $Z(G)$.
		\par\vspace{0.3cm} Para isso, tome $G/Z(G) = \langle gZ(G) \rangle$ e seja $a\in G$ qualquer. 
		Daí, existe $i\in\mathbb{Z}$ tal que $aZ(G) = (gZ(G))^i = g^iZ(G)$. Logo, $a = g^iz$, para algum 
		$z$ em $Z(G)$. Como $g^i$ e $z$ comutam com $g$, então $a$ comuta com $g$. Mas $a$ é um elemento 
		qualquer de $G$, logo todo elemento de $G$ comuta com $g$, ou seja, $g\in Z(G)$.
		\par\vspace{0.3cm} Portanto, $gZ(G) = Z(G)$ e $G/Z(G) = \langle Z(G) \rangle$.
	\end{proof}
	%
	\begin{remark}
		Note que essa demonstração revela que se $G/Z(G)$ é cíclico, então ele tem de ser trivial.
	\end{remark}
	%
	\begin{remark}
		Essa demonstração também revela que se $G/H$ é cíclico, sendo $H$ um subgrupo de $Z(G)$, 
		então $G$ é abeliano.
	\end{remark}
	%
	\begin{remark}
		Por fim, a contra-positiva do Teorema \ref{teorema G/Z} é mais usada, isto é, se $G$ não é abeliano,
		$G/Z(G)$ não é cíclico. Por exemplo, segue dessa sentença e do Teorema \ref{lagrange} que 
		um grupo não abeliano de ordem $pq$, com $p, q$ primos, deve ter um centro trivial, pois como todo 
		grupo de ordem prima é cíclico, então $|G/Z(G)| \overset{!}{=} pq$ 
		(pois $G/Z(G)$ não pode ser cíclico), 
		o que implica $|Z(G)| = 1$. 
	\end{remark}
	%
	\begin{theorem}
		Para todo grupo $G$, $G/Z(G)$ é isomorfo a $\inn(G)$.
	\end{theorem}
	%
	\begin{proof}
		Tome a correspondência $T:gZ(G)\to \phi_g$. Primeiro, vamos mostar que $T$ é bem definida. 
		\par\vspace{0.3cm} Para isso, suponha que $gZ(G) = hZ(G)$; daí, $h^{-1}g\in Z(G)$. 
		Portanto, para todo $x$ em $G$, $h^{-1}gx = xh^{-1}g$, logo $gxg^{-1} = hxh^{-1}$ para todo 
		$x$ em $G$ e, portanto, $\phi_g = \phi_h$. 
		\par\vspace{0.3cm} Agora, suponha que $\phi_g = \phi_h$. Então, $gxg^{-1} = hxh^{-1}$, 
		para todo $x$ em $G$, ou seja, $h^{-1}gx = xh^{-1}g$. Daí, $h^{-1}g\in Z(G)$, ou seja, 
		$gZ(G) = hZ(G)$. Logo, $T$ é injetora.
		\par\vspace{0.3cm} Pela maneira que definimos $T$, ela é naturalmente sobrejetora. 
		\par\vspace{0.3cm} Por fim,
		%
		\begin{align*}
		    T[(gZ(G))(hZ(G))] = T(ghZ(G)) = \phi_{gh} = \phi_g\circ\phi_h = T(gZ(G))T(hZ(G)),
		\end{align*}
		%
		logo $T$ preserva a operação.
	\end{proof}
	%
	\begin{theorem}[Cauchy]
	\label{cauchy}
		Seja $G$ um grupo e $p$ um primo que divide a ordem de $G$. Então, $G$ tem um elemento de ordem $p$.
	\end{theorem}
	
	\begin{proof}
		Vamos provar o teorema usando o Segundo Princípio da Indução. 
		\par\vspace{0.3cm} Note que se $G$ tem ordem 2, nosso teorema é satisfeito. 
		Suponha que $G$ tem ordem maior que 2 e que para todo grupo com ordem menor que $|G|$, 
		o teorema é satisfeito. 
		\par\vspace{0.3cm} Certamente $G$ tem elementos de ordem prima, pois se $|x| = m$ e $m = qn$, 
		com $q$ primo, então $(x^n)^q = e$, ou seja, $|x^n| = q$, ou seja, $x^n$ tem ordem prima $q$.
		\par\vspace{0.3cm} Então, seja $x$ um elemento de $G$ de ordem prima $q$. Se $q = p$, acabamos. 
		Então, suponha $q\neq p$ e seja $\overline{G} = G/\langle x \rangle$.
		\par\vspace{0.3cm} Daí, $p$ divide $|\overline{G}|$, pois $|\overline{G}| = |G|/q$. Por indução, 
		pois $|\overline{G}|<|G|$, $\overline{G}$ tem um elemento de ordem $p$, digamos 
		$y\cdot\langle x \rangle$.
		\par\vspace{0.3cm} Logo, $(y\langle x \rangle)^p = y^p\langle x \rangle = \langle x\rangle$, ou seja,
		$y^p\in\langle x \rangle$. Como $|\langle x \rangle| = q$, então $y^p$ ou é a identidade ou 
		tem ordem $q$, pelo Teorema \ref{lagrange}.
		\par\vspace{0.3cm} Se $y^p = e$, terminamos, pois $y\in G$ e $|y|=p$. Se $(y^p)^q = e$, 
		então $(y^q)^p = e$ e terminamos, pois $y^q\in G$ e $|y^q| = p$.
	\end{proof}
	%
	\begin{deff}
	\label{def prod direto interno}
		Dizemos que $G$ é o produto direto interno de $H$ e $K$ e escrevemos $G = H\times K$ se 
		$H$ e $K$ são subgrupos normais de $G$ com $G = HK$ e $H\cap K = \{e\}$.
		\par\vspace{0.3cm} Mais geralmente, se $H_1, H_2, \dots, H_n$ é uma coleção finita de 
		subgrupos normais de $G$, então dizemos que $G$ é o produto direto interno de $H_1, H_2, \dots, H_n$ 
		e escrevemos $G = H_1\times H_2\times\cdots\times H_n$ se
		%
		\begin{enumerate}
			\item $G = H_1H_2\cdots H_n = \left\{ h_1h_2\cdots h_n \ | \ h_i\in H_i\right\}$;
			\item $(H_1H_2\cdots H_i)\cap H_{i+1} = \{e\}$, para $i = 1, 2, \dots, n-1$.
		\end{enumerate}
		%
	\end{deff}
	%
	\begin{theorem}
	\label{isomorfismo entre interno e externo}
		Se um grupo $G$ é o produto direto interno de um número finito de subgrupos 
		$H_1, H_2, \dots, H_n$, então $G$ é isomorfo ao produto direto externo de $H_1, H_2, \dots, H_n$.
	\end{theorem}
	%
	\begin{proof}
		Primeiro vamos mostrar que a normalidade dos $H's$, junto com a segunda condição da 
		Definição \ref{def prod direto interno}, implica na comutatividade dos $h's$. 
		Tome $h_i\in H_i$ e $h_j\in H_j$ com $i\neq j$. Então, temos
		%
		\begin{align*}  
		    (h_ih_jh_i^{-1})h_j^{-1} \in H_jh_j^{-1} &= H_j, \\
		    h_i(h_jh_i^{-1}h_j^{-1}) \in h_iH_i &= H_i.
		\end{align*}
		%
		\par\vspace{0.3cm} Então, $h_ih_jh_i^{-1}h_j^{-1}\in H_i\cap H_j = \{e\}$, logo $h_ih_j = h_jh_i$. 
		Além disso, vamos provar que cada elemento de $G$ pode ser expresso unicamente na forma 
		$h_1\cdots h_n$, com $h_i\in H_i$. Pela condição 1 da Definição \ref{def prod direto interno}, 
		sabemos que existe pelo menos uma representação. Então, suponha que $g = h_1\cdots h_n$ 
		e $g = h_1'\cdots h_n'$. Daí, usando a comutatividade dos $h's$, podemos resolver a seguinte equação
		%
		\begin{equation*}
		    h_1\cdots h_n = h_1'\cdots h_n'
		\end{equation*}
		%
		\par\vspace{0.3cm} para $h_n'h_n^{-1}$ para obter
		%
		\begin{equation*}
		    h_n'h_n^{-1} = (h_1')^{-1}h_1\cdots (h_{n-1}')^{-1}h_{n-1}.
		\end{equation*}
		%
		\par\vspace{0.3cm} Mas então $h_n'h_n^{-1}\in H_1\cdots H_{n-1}\cap H_n = \{e\}$, ou seja, 
		$h_n' = h_n$. Com isso, podemos cancelar $h_n'$ e $h_n$ na equação acima e resolver de modo 
		análogo para $h_{n-1}'h_{n-1}^{-1}$. Procedendo dessa maneira, concluímos que $h_i' = h_i$ 
		para $i = 1, 2, \dots, n$.
		
		\par\vspace{0.3cm} Com isso, podemos definir uma função $\phi$ de $G$ em 
		$\displaystyle{\bigoplus_{i = 1}^{n}H_i}$ por $\phi(h_1\cdots h_n) = (h_1, \dots, h_n)$. 
		Vamos mostrar que $\phi$ é um isomorfismo.
		
		\par\vspace{0.3cm} Note que se $\phi(h_1\cdots h_n) = \phi(h_1'\cdots h_n')$, então 
		$(h_1, \dots, h_n) = (h_1', \dots, h_n')$, ou seja, $h_i = h_i'$ para $i = 1, 2, \dots, n$. 
		Logo, $\phi$ é injetiva.
		
		\par\vspace{0.3cm} Pelo modo que definirmos $\phi$, vemos que ela é naturalmente sobrejetora.
		
		\par\vspace{0.3cm} Por fim, sejam $(h_1\cdots h_n), (h_1'\cdots h_n')\in G$. Daí, temos
		%
		\begin{align*}
		    \phi[ (h_1\cdots h_n)(h_1'\cdots h_n') ] &= \phi (h_1h_1'\cdots h_nh_n') \\ 
		                                             &= (h_1h_1', \dots, h_nh_n') \\ 
		                                             &= (h_1, \dots, h_n)(h_1', \dots, h_n') \\ 
		                                             &= \phi(h_1\cdots h_n)\phi(h_1'\cdots h_n').
		\end{align*} 
		%
		\par\vspace{0.3cm} Portanto, $\phi$ é, de fato, isomorfismo.
	\end{proof}
	%
	\begin{theorem}
	\label{classificacao grupos de ordem p2}
		Todo grupo de ordem $p^2$, $p$ primo, é isomorfo a $\mathbb{Z}_{p^2}$ ou $\mathbb{Z}_p\oplus\mathbb{Z}_p$.
	\end{theorem}
	%
	\begin{proof}
		Seja $G$ um grupo de ordem $p^2$. Se $G$ tem um elemento de ordem $p^2$, então $G$ é cíclico e, 
		portanto, isomorfo a $\mathbb{Z}_{p^2}$. Então, suponha que todo elemento não identidade de $G$ 
		tem ordem $p$. Vamos, primeiro, mostrar que para todo $a$ em $G$, $\langle a \rangle\vartriangleleft G$.
		
		\par\vspace{0.3cm} Suponha o contrário. Então, existe $b\in G$ tal que $bab^{-1}\notin\langle a \rangle$.
		Logo, $\langle a \rangle$ e $\langle bab^{-1} \rangle$ são subgrupos distintos de ordem $p$. 
		Note que $\langle a \rangle\cap \langle bab^{-1} \rangle = \{e\}$. Daí, as classes laterais à 
		esquerda de $\langle bab^{-1} \rangle$ são $\langle bab^{-1} \rangle$, $a\langle bab^{-1} \rangle$,
		$\dots$, $a^{p-1}\langle bab^{-1} \rangle$. Como $b^{-1}$ deve estar em uma dessas classes, podemos
		escrever
		%
		\begin{align*}
		    b^{-1} = a^i(bab^{-1})^j
		\end{align*}
		%
		\par\vspace{0.3cm} para alguns $i,j\in\mathbb{Z}$. Daí,
		%
		\begin{align*}
		    b^{-1} = a^iba^jb^{-1} \iff e = a^iba^j \iff b = a^{-i-j}\in \langle a \rangle
		\end{align*}
		%
		\par\vspace{0.3cm} o que é absurdo. Portanto, para todo $a$ em $G$, o subgrupo 
		$\langle a \rangle$ é normal.
		
		\par\vspace{0.3cm} Por fim, seja $x$ um elemento não identidade de $G$ e $y$ um elemento de $G$ 
		fora de $\langle x \rangle$. Tanto $\langle x \rangle$ quanto $\langle y \rangle$ são normais 
		de ordem $p$, e sua interseção é trivial. Logo, 
		$G = \langle x \rangle\times\langle y \rangle\cong \langle x \rangle\oplus\langle y \rangle 
		\cong \mathbb{Z}_p\oplus\mathbb{Z}_p$.
	\end{proof}
	%
	\begin{corollary}
	\label{ordem p^2 implica abeliano}
		Se $G$ é um grupo de ordem $p^2$, $p$ primo, então $G$ é abeliano.
	\end{corollary}
	%
	\begin{proof}
		Do Teorema \ref{classificacao grupos de ordem p2}, segue que $G$ é isomorfo a 
		$\mathbb{Z}_{p^2}$ ou a $\mathbb{Z}_p\oplus\mathbb{Z}_p$. Como ambos são abelianos, segue que 
		$G$ também é abeliano, pois isomorfismos levam grupos abelianos em grupos abelianos.
	\end{proof}
    %
\section{Homomorfismos}
\label{sec-homomorfismos}
    %
    \begin{definition}
	\label{def homomorfismo}
		Um \textbf{homomorfismo} $\phi$ de um grupo $G$ para um grupo $\overline{G}$ é uma função 
		de $G$ em $\overline{G}$ que preserva a operação do grupo, isto é, 
		$\phi(ab) = \phi(a)\phi(b), \forall a,b\in G$.
	\end{definition}
	%
	\begin{definition}
	\label{def nucleo}
		O \textbf{núcleo} de um homomorfismo $\phi$ de um grupo $G$ para um grupo com identidade $e$ 
		é o conjunto $\{ x\in G \ | \ \phi(x) = e \}$, ou seja, o conjunto dos elementos que são 
		levados na identidade. O núcleo de $\phi$ é denotado por $\Ker\phi$.
	\end{definition}
	%
	\begin{remark}
		Note que, das Definições \ref{def isomorfismo} e \ref{def homomorfismo}, temos que um 
		isomorfismo nada mais é que um homomorfismo bijetor.
	\end{remark}
	%
	\par\vspace{0.3cm} Assim como foi para isomorfismos, os homomorfismos também têm propriedades, que
	naturalmente são parecidas com as dos isomorfismos.
	%
	\begin{theorem}
	\label{homomorfismos em elementos}
		Sejam $\phi$ um homomorfismo de $G$ em $\overline{G}$ e $g\in G$ um elemento qualquer. Então, 
		%
		\begin{enumerate}
			\item $\phi$ leva a identidade de $G$ na identidade de $\overline{G}$;
			\item $\phi(g^n) = (\phi(g))^n, \forall n\in \mathbb{Z}$;
			\item se $|g|$ é finita, então $|\phi(g)|$ divide $|g|$;
			\item $\Ker\phi$ é subgrupo de $G$;
			\item $\phi(a) = \phi(b) \iff a\Ker\phi = b\Ker\phi$;
			\item Se $\phi(g) = g'$, então $\phi^{-1}(g') = \{ x\in G \ | \ \phi(x) = g' \} = g\Ker\phi$.
		\end{enumerate}
		%
	\end{theorem}
	%
	\begin{proof}
		\textbf{1.} Note que 
		%
		\begin{align*}
		    e = ee \Rightarrow \phi(e) = \phi(e)\phi(e) \Rightarrow \phi(e) = \overline{e}
		    \text{ sendo $\overline{e}$ a identidade de $\overline{G}$}
		\end{align*}
		%
		\par\vspace{0.3cm}\hspace{17pt}\textbf{2.} Para $n = 0$ a propriedade é imediata. 
		Suponha, então, $n>0$. Daí, temos
		%
		\begin{align*}
		    \phi(g^n) = \underbrace{\phi(g)\phi(g)\cdots\phi(g)}_{n} = (\phi(g))^n.
		\end{align*} 
		%
		\par\vspace{0.3cm} Agora, suponha $n<0$. Daí, podemos escrever
		%
		\begin{align*}
		    \phi(e) = \phi(g^ng^{-n}) = \phi(g^n)(\phi(g))^{-n} = \overline{e} \Rightarrow \phi(g^n) = (\phi(g))^n.
		\end{align*}
		%
		\par\vspace{0.3cm}\hspace{17pt}\textbf{3.} Sejam $|g| = n$ e $|\phi(g)| = m$. 
		Sabemos, então, que $\phi(e) = \phi(g^n) = (\phi(g))^n = \overline{e}$. 
		Portanto, a ordem de $\phi(g)$ divide a ordem de $g$, ou seja, $m|n$. Note que a volta nem sempre é
		válida, pois $\phi$ não precisa ser injetora.
		
		\par\vspace{0.3cm}\hspace{17pt}\textbf{4.} Sabemos que $e\in \Ker\phi$. Sejam $a,b\in \Ker\phi$. 
		Então, sabemos que 
		%
		\begin{align*}
		    \phi(ab^{-1}) = \phi(a)(\phi(b))^{-1} = \overline{e}\overline{e} 
		                  = \overline{e} \therefore ab^{-1}\in\Ker\phi
		                  \text{ e, pelo teste do subgrupo, Ker $\phi$ é subgrupo de $G$.}
		\end{align*}
		%
		\par\vspace{0.3cm}\hspace{17pt}\textbf{5.} Note que
		%
		\begin{align*}
		    \phi(a) = \phi(b) \iff (\phi(b))^{-1}\phi(a) 
		            = \overline{e} \iff b^{-1}a\in \Ker\phi \overset{\text{Lema }\ref{propriedades}}{\iff} 
		            a\Ker\phi = b\Ker\phi.
		\end{align*}
		%
		\par\vspace{0.3cm}\hspace{17pt}\textbf{6.} Seja $x\in\phi^{-1}(g')$. Então, sabemos que
		%
		\begin{align*}
		    \phi(x) 
		    = \phi(g) \overset{5}{\iff} x\Ker\phi 
		    = g\Ker\phi \overset{\text{Lema }\ref{propriedades}}{\iff}
		    x\in g\Ker\phi \therefore \phi^{-1}(g') \subseteq g\Ker\phi,
		\end{align*}
		%
		\par\vspace{0.3cm} em que foi usada a propriedade anterior na primeira equivalência. 
		Agora, seja $k\in\Ker\phi$. Daí, sabemos que
		%
		\begin{align*}
		    %
		    \begin{cases} 
		        \phi(k) = \overline{e} \\ 
		        \phi(gk) = \phi(g) = g'
		    \end{cases} 
		    %
		    \text{, ou seja, $gk\in \phi^{-1}(g')$}\therefore g\Ker\phi \subseteq \phi^{-1}(g')
		\end{align*}
		%
		\par\vspace{0.3cm} Logo, $g\Ker\phi = \phi^{-1}(g')$.
	\end{proof}
	%
	\begin{theorem}
	\label{homomorfismos em subgrupos}
		Sejam $\phi$ um homomorfismo de $G$ em $\overline{G}$ e $H$ um subgrupo de $G$. Então,
		%
		\begin{enumerate}
			\item $\phi(H) = \{ \phi(h) \ | \ h\in H \}$ é subgrupo de $\overline{G}$
			\item Se $H$ é cíclico, então $\phi(H)$ é cíclico
			\item Se $H$ é abeliano, então $\phi(H)$ é abeliano
			\item Se $H$ é normal em $G$, então $\phi(H)$ é normal em $\phi(G)$
			\item Se $|\Ker\phi| = n$, então $\phi$ é um mapeamento $n$ para 1 de $G$ em $\phi (G)$
			\item Se $|H| = n$, então $|\phi(H)|$ divide $n$
			\item Se $\overline{K}$ é um subgrupo de $\overline{G}$, então 
			$\phi^{-1}(\overline{K}) = \{ k\in G \ | \ \phi(k)\in \overline{K} \}$ é subgrupo de $G$
			\item Se $\overline{K}$ é um subgrupo normal de $\overline{G}$, 
			então $\phi^{-1}(\overline{K}) = \{ k\in G \ | \ \phi(k)\in \overline{K} \}$ é subgrupo normal de $G$
			\item Se $\phi$ é sobrejetora e $\Ker\phi = \{e\}$, então $\phi$ é 
			um isomorfismo de $G$ em $\overline{G}$
		\end{enumerate}
		%
	\end{theorem}
	%
	\begin{proof}
		\textbf{1.} Sejam $\phi(h_1), \phi(h_2)\in \phi(H)$ quaisquer. Então, temos
		%
		\begin{align*}
		    \phi(h_1)(\phi(h_2))^{-1} = \phi(\underbrace{h_1h_2^{-1}}_{\in H}) \in\phi(H).
		\end{align*}
		%
		Portanto, pelo teste do subgrupo, $\phi(H)$ é subgrupo.
		
		\par\vspace{0.3cm}\hspace{17pt}\textbf{2.} Sejam $H = \langle h_1 \rangle$, $|\phi(h_1)| = k$ 
		e $|h_1| = n$. Por um lado, sabemos que
		%
		\begin{align*}
		    \phi(h_1^n) = (\phi(h_1))^n = \overline{e}\therefore k \mid n.
		\end{align*}
		%
		\par\vspace{0.3cm} Por outro lado, sabemos que
		%
		\begin{align*}
		    (\phi(h_1))^k = \phi(h_1^k) = \overline{e}\Rightarrow h_1^k = e\therefore n\mid k.
		\end{align*}
		%
		\par\vspace{0.3cm} Logo, $n = k$. Agora, basta notar que, como todo elemento de $H$ é uma 
		potência de $h_1$, então todo elemento de $\phi(H)$ é uma potência de $\phi(h_1)$, 
		logo $\phi(H) = \langle \phi(h_1) \rangle$. 
		
		\par\vspace{0.3cm}\hspace{17pt}\textbf{3.} Sejam $h_1, h_2\in H$. Se $H$ é abeliano, então
		%
		\begin{align*}
		    h_1h_2 = h_2h_1 \Rightarrow \phi(h_1)\phi(h_2) = \phi(h_2)\phi(h_1).
		\end{align*}
		%
		\par\vspace{0.3cm} Logo, $\phi(H)$ é abeliano. Note que a volta nem sempre é verdadeira, 
		pois $\phi$ não é injetora, necessariamente.
		
		\par\vspace{0.3cm}\hspace{17pt}\textbf{4.} Se $H$ é normal em $G$, então existem 
		$h_1, h_2\in H$ tais que, para todo $g\in G$, $gh_1 = h_2g$. Daí, sendo $\phi(g) = g'$, temos
		%
		\begin{align*}
		    \phi(g)\phi(h_1) 
		    = \phi(h_2)\phi(g) \Rightarrow g'\phi(h_1) 
		    = \phi(h_2)g' \Rightarrow g'\phi(H) = \phi(H)g' 
		    \therefore \phi(H)\vartriangleleft\phi(G).
		\end{align*}
		%
		\par\vspace{0.3cm}\hspace{17pt}\textbf{5.} Sejam $ e = x_1, x_2, \dots, x_n$ os elementos de $\Ker\phi$.
		Seja ainda $g\in G$ qualquer. Note que para todo $i = 1, 2, \dots, n$, $\phi(x_ig) = \phi(g)$, 
		ou seja, há $n$ elementos que são enviados para $\phi(g)$, a saber $g, x_2g, \dots, x_ng$.
		
		\par\vspace{0.3cm}\hspace{17pt}\textbf{6.} Seja $\phi_H$ a restrição de $\phi$ aos elementos de $H$.
		Então, $\phi_H$ é um homomorfismo de $H$ em $\phi(H)$. Suponha que $|\Ker\phi_H| = t$. Então, pela
		propriedade 5, $\phi_H$ é um mapeamento $t$ para 1, logo $|\phi(H)|t = |H| = n$, ou seja, 
		$|\phi(H)|$ divide $n$.
		
		\par\vspace{0.3cm}\hspace{17pt}\textbf{7.} Note que $e$ pertence a $\phi^{-1}(\overline{K})$. 
		Agora, sejam $k_1, k_2\in \phi^{-1}(\overline{K})$. Então, por definição, sabemos que 
		$\phi(k_1), \phi(k_2)\in\overline{K}$. Note que 
		$\phi(k_1k_2^{-1}) = \phi(k_1)(\phi(k_2))^{-1}\in\overline{K}$, pois $k_1k_2^{-1}\in G$ 
		e $\phi(k_1),(\phi(k_2))^{-1}\in\overline{K}$. Logo, por definição,
		$k_1k_2^{-1}\in\phi^{-1}(\overline{K})$. 	
		
		\par\vspace{0.3cm}\hspace{17pt}\textbf{8.} Vamos usar o Teorema \ref{condicao}. 
		Basta mostrar que $x\phi^{-1}(\overline{K})x^{-1}\subseteq\phi^{-1}(\overline{K})$, 
		para todo $x$ em $G$. Note que todo elemento em $x\phi^{-1}(\overline{K})x^{-1}$ tem a forma 
		$xkx^{-1}$, sendo $\phi(k)\in\overline{K}$. Como $\overline{K}\vartriangleleft\overline{G}$, 
		então $\phi(xkx^{-1}) = \phi(x)\phi(k)(\phi(x))^{-1}$ deve pertencer a $\overline{K}$. 
		Como $xkx^{-1}\in G$, então de fato $xkx^{-1}\in\phi^{-1}(\overline{K})$
		
		\par\vspace{0.3cm}\hspace{17pt}\textbf{9.} Como $|\Ker\phi| = 1$, sabemos, pela propriedade 5, 
		que $\phi$ é injetora. Como, por hipótese, $\phi$ é sobrejetora e, por definição $\phi$ preserva 
		a operação, concluímos que $\phi$ é isomorfismo.
	\end{proof}
	%
	A propriedade 8 do Teorema \ref{homomorfismos em subgrupos} tem um caso especial
	interessante, a saber quando $\overline{K} = \{\overline{e}\}$.
	%
	\begin{corollary}
	\label{nucleo normal}
		Seja $\phi$ um homomorfismo de $G$ em $\overline{G}$. Então, $\Ker\phi\vartriangleleft G$. 
	\end{corollary}
	%
	\begin{proof}
		Tomando $\overline{K}$ trivial na propriedade 8, obtemos o fato de que 
		$\phi^{-1}(\overline{e}) = \Ker\phi$ é normal em $G$.
	\end{proof}
	%
	\begin{example}
	Usando as propriedades dos homomorfismos, vamos determinar todos os homomorfismos 
	de $\mathbb{Z}_{12}$ em $\mathbb{Z}_{30}$. Pela propriedade 2 do 
	Teorema \ref{homomorfismos em elementos},
	sabemos que os homomorfismos são completamente definidos pela imagem de 1, ou seja, 
	se $1$ é mapeado em $a$, então $x$ é mapeado em $xa$. Pelo Teorema \ref{lagrange}, $|a|$ divide 30 e,
	pela propriedade 3 do Teorema \ref{homomorfismos em elementos}, $|a|$ divide 12. Daí, 
	$|a| = 1, 2, 3 \text{ ou } 6$. Agora, temos de ver quais elementos de $\mathbb{Z}_{30}$ têm tais ordens.
	Podemos ver que $a = 0, 15, 10, 20, 5 \text{ ou } 25$. Cada uma dessas seis correspondências nos dá uma
	aplicação bem definida e que preserva a operação.
	\end{example}
	%
	\begin{example}
	Outro exemplo é entre $S_n$ e $\mathbb{Z}_2$. A função de $S_n$ em $\mathbb{Z}_2$ 
	que leva uma permutação par para 0 e uma permutação ímpar para 1 é um homomorfismo. Para mostrar isso, 
	basta notar que se $a,b$ são duas permutações quaisquer de $S_n$, então $ab$ ou é par ou é ímpar. 
	Se $ab$ é par, então $a$ e $b$ ou são ambos pares ou ambos ímpares. Se ambos são pares, 
	$\phi(ab) = 0 = 0 + 0 = \phi(a) + \phi(b)$. Se ambos são ímpares, 
	$\phi(ab) = 0 = 1 + 1 = \phi(a) + \phi(b)$. Por fim, se $ab$ é ímpar, podemos assumir, 
	sem perda de generalidade, que $a$ é ímpar e $b$ é par. Daí, $\phi(ab) = 1 = 1 + 0 = \phi(a) + \phi(b)$.
	\end{example}
	%
	\begin{theorem}[1$^\circ$ Teorema de Isomorfismos]
	\label{primeiro teorema de isomorfismo}
		Seja $\phi$ um homomorfismo de $G$ em $\overline{G}$. Então, a correspondência de $G/\Ker\phi$ 
		para $\phi(G)$ dada por $g\Ker\phi \to \phi(g)$, é um isomorfismo. 
		Em símbolos, $G/\Ker\phi\cong\phi(G)$.
	\end{theorem}
	%
	\begin{proof}
		Seja $\psi$ tal correspondência. Pela propriedade 5 do Teorema \ref{homomorfismos em elementos}, 
		$\psi$ é bem definida e injetora. Para mostrar que $\psi$ preserva a operação, basta notar que 
		para quaisquer $x,y\in G$, temos
		%
		\begin{align*}
		    \psi(x\Ker\phi y\Ker\phi) 
		    = \psi(xy\Ker\phi) 
		    = \phi(xy) 
		    = \phi(x)\phi(y) 
		    = \psi(x\Ker\phi)\psi(y\Ker\phi).
		\end{align*}
		%
		Como todo $\phi(g)$ em $\phi(G)$ é atingido por algum elemento de $G/\Ker\phi$, 
		a saber $g\Ker\phi$, então $\psi$ é sobrejetora e, portanto, isomorfismo.
	\end{proof}
	%
	\par\vspace{0.3cm} Usando o Teorema \ref{lagrange}, a propriedade 1 do 
	Teorema \ref{homomorfismos em subgrupos} e o Teorema \ref{primeiro teorema de isomorfismo}, 
	podemos provar o seguinte corolário.
	%
	\begin{corollary}
		Se $\phi$ é um homomorfismo de um grupo finito $G$ em outro grupo finito $\overline{G}$, 
		então $|\phi(G)|$ divide $|G|$ e $|\overline{G}|$.
	\end{corollary}
	%
	\begin{proof}
		Do Teorema \ref{primeiro teorema de isomorfismo}, sabemos que $G/\Ker\phi\cong\phi(G)$, 
		ou seja, $|G/\Ker\phi| = |\phi(G)|$. Daí, sabemos que $|G| = |\phi(G)|\cdot|\Ker\phi|$, 
		portanto $|\phi(G)|$ divide $|G|$. Agora, usando a propriedade 1 do 
		Teorema \ref{homomorfismos em subgrupos}, sabemos que $\phi(G)$ é subgrupo de $\overline{G}$.
		Consequentemente, pelo Teorema \ref{lagrange}, $|\phi(G)|$ divide $|\overline{G}|$.
	\end{proof}
	%
	Com o Teorema \ref{primeiro teorema de isomorfismo} podemos mostrar que
	$\mathbb{Z}/\langle n \rangle\cong \mathbb{Z}_n$. Para isso, considere a função $\phi$ definida por
	$\phi(m) = m\pmod n$. Podemos ver que todos os múltiplos de $n$ são levados na identidade, 
	ou seja, $\Ker\phi = \langle n \rangle$. Então, do Teorema \ref{primeiro teorema de isomorfismo},
	$\mathbb{Z}/\langle n \rangle\cong \mathbb{Z}_n$, uma vez que 
	$\phi(\mathbb{Z}/\langle n \rangle) = \mathbb{Z}_n$.
	%
	\begin{theorem}
	\label{normalizador centralizador}
		$N(H)/C(H)$ é isomorfo a um subgrupo de $\aut(H)$. 
	\end{theorem}
	%
	\begin{proof}
		Seja $H$ um subgrupo de $G$ e lembre-se que o normalizador de $H$ em $G$ é 
		$N(H) = \{ x\in G \ | \ xHx^{-1} = H \}$ e que o centralizador de $H$ em $G$ é 
		$C(H) = \{ x\in G \ | \ xhx^{-1} = h, \forall h\in H \}$. Considere o mapeamento de $N(H)$ em 
		$\aut(H)$ dado por $g\mapsto \phi_g$, em que $\phi_g$ é o automorfismo interno de $H$ induzido 
		por $g$ (ou seja, $\phi_g(h) = ghg^{-1}, \forall h\in H$). Esse mapeamento é um homomorfismo 
		(pois se $\gamma$ é tal aplicação, então 
		$\gamma(g_1g_2) = \phi_{g_1g_2} = \phi_{g_1}\circ\phi_{g_2} = \gamma(g_1)\gamma(g_2)$) com núcleo 
		$C(H)$ (pois o núcleo é o conjunto dos $g$ tais que $ghg^{-1} = h, \forall h\in H$, que é $C(H)$, 
		por definição). Daí, do Teorema \ref{primeiro teorema de isomorfismo}, temos $N(H)/C(H)$ 
		isomorfo a um subgrupo de $\aut(H)$.
	\end{proof}
	%
	\begin{theorem}
		\label{subgrupos normais e nucleos}
		Todo subgrupo normal de um grupo $G$ é o núcleo de um homomorfismo de $G$. Em particular, um subgrupo normal $N$ é núcleo do homomorfismo $g\mapsto gN$ de $G\to G/N$. Tal homomorfismo é chamado \textbf{homomorfismo natural} de $G$ em $G/N$.
	\end{theorem}
	%
	\begin{proof}
		Seja $\displaystyle{ \underset{g\mapsto gN}{\gamma : G\to G/N }}$. Note que 
		$\gamma(xy) = (xy)N = xNyN = \gamma(x)\gamma(y)$, logo $\gamma$ é homomorfismo. 
		Além disso, note que se $g$ é um elemento qualquer de $\Ker\gamma$, então $gN = \gamma(g) = N$ que, 
		pela propriedade 2 do Lema \ref{propriedades}, é verdade se, e só se, $g\in N$. Logo, 
		todo elemento do núcleo é, na verdade, elemento de $N$ e vice-versa. Portanto, $N = \Ker\gamma$.
	\end{proof}
	%
	Assim como foi o caso com grupos quocientes, as imagens homomórficas de um grupo 
	nos dizem \textbf{algumas} propriedades do grupo original. Uma medida da similaridade entre um grupo 
	e sua imagem homomórfica é o tamanho do núcleo. Se o núcleo é a identidade, então a imagem de $G$ 
	nos diz tudo sobre $G$ (uma vez que ambos são isomorfos). Por outro lado, se o núcleo é o próprio grupo, 
	então a imagem não nos diz nada sobre $G$. Entre esses dois extremos, alguma parte da informação sobre 
	$G$ é preservada e outra é perdida. A utilidade de um homomorfismo particular é preservar propriedades 
	que queremos e descartar propriedades desnecessárias. Desse modo, substituímos $G$ por um grupo mais 
	simples de estudar, mas que possui as características essenciais de $G$ em que estamos interessados.
	%
	\begin{example}
	Por exemplo, se $G$ tem ordem 60 e uma imagem homomórfica cíclica de ordem 12, então
	sabemos (pelas propriedades 5, 7 e 8 do Teorema \ref{homomorfismos em subgrupos}) que $G$ tem subgrupos
	normais de ordens 5, 10, 15, 20, 30 e 60. Para ilustrar melhor, suponha que quiséssemos encontrar um grupo
	infinito que é dado pela união de três subgrupos próprios. Podemos tornar o problema mais simples encontrando,
	primeiro, um grupo finito que é a união de três subgrupos próprios. Notando que
	$\mathbb{Z}_2\oplus\mathbb{Z}_2$ é a união de $H_1 = \langle 1,0 \rangle$, $H_2 = \langle 0,1 \rangle$ e 
	$H_3 = \langle 1,1 \rangle$, encontramos nosso grupo finito. Agora, precisamos encontrar um grupo infinito
	cuja imagem homomórfica é $\mathbb{Z}_2\oplus\mathbb{Z}_2$ e pegar as imagens inversas de $H_1, H_2$ e $H_3$.
	Podemos ver que o mapeamento de $\mathbb{Z}_2\oplus\mathbb{Z}_2\oplus\mathbb{Z}$ em
	$\mathbb{Z}_2\oplus\mathbb{Z}_2$ dado por $\phi(a,b,c) = (a,b)$ satisfaz nossas condições e, portanto,
	$\mathbb{Z}_2\oplus\mathbb{Z}_2\oplus\mathbb{Z}$ é a união de 
	$\phi^{-1}(H_1) = \{ (a,0,c) \ | \ a\in\mathbb{Z}_2, c\in\mathbb{Z} \}$, $\phi^{-1}(H_2) 
	= \{ (0,b,c) \ | \ b\in\mathbb{Z}_2, c\in\mathbb{Z} \}$ e 
	$\phi^{-1}(H_3) = \{ (a,a,c) \ | \ a\in\mathbb{Z}_2, c\in\mathbb{Z} \}$.
	\end{example}
	%
	Apesar de serem parecidos, homomorfismos e isomorfismos desempenham papeis diferentes.
	Enquanto isomorfismos nos permitem olhar de um modo diferente para um grupo, homomorfismos agem como
	ferramentas investigativas. Em certas áreas de Teoria dos Grupos (especialmente nas aplicações
	à Física e à Química),
	geralmente queremos saber todas as imagens homomórficas de um grupo, mas sendo essas imagens grupos de
	matrizes sobre os complexos (chamadas representações de grupos). 
	
	\par\vspace{0.3cm} Antes de definir produto semidireto, seja $N$ um subgrupo normal de $G$. 
	Cada elemento $g$ de $G$ define um automorfismo $n\mapsto gng^{-1}$ de $N$ e isso define um homomorfismo
	%
	\begin{equation*}
	    \theta: G\to \aut(N), \text{ }g\mapsto i_g|_N.
	\end{equation*}
	%
	Se existe um subgrupo $Q$ de $G$ tal que $G\to G/N$ mapeia $Q$ isomorficamente 
	em $G/N$, então podemos reconstruir $G$ a partir de $N$, $Q$ e a restrição de $\theta$ a $Q$. 
	De fato, um elemento $g$ de $G$ pode ser escrito de forma única na forma
	%
	\begin{equation*}
	    g=nq, \text{ }n\in N, \text{ }q\in Q,
	\end{equation*}
	%
	onde $q$ deve ser o único elemento de $Q$ sendo mapeado em $gN\in G/N$, e 
	$n$ deve ser $gq^{-1}$. Logo, temos uma correspondência injetiva entre os conjuntos
	$G$ e $N\oplus Q$.
	%
	\par\vspace{0.3cm} Se $g=nq$ e $g'=n'q'$, então
	%
	\begin{equation*}
	    gg' = (nq)(n'q') = n(qn'q^{-1})qq' = n\cdot\theta_q(n')\cdot qq'.
	\end{equation*}
	%
	\begin{definition}
	\label{produto semidireto}
		Um grupo $G$ é o produto semidireto de seus subgrupos $N$ e $Q$ se $N$ é normal e 
		o homomorfismo $G\to G/N$ induz um isomorfismo $Q\to G/N$. Equivalentemente, 
		$G$ é o produto semidireto de $N$ e $Q$ se
		%
		\begin{equation*}
		    N\vartriangleleft Q, \text{ }NQ = G,\text{ }N\cap Q=\left\{1\right\}.
		\end{equation*}
		%
		Note que $Q$ não precisa ser normal em $G$. Quando $G$ é o produto semidireto 
	de $N$ e $Q$, escrevemos $G = N \rtimes Q$ ou ainda $G = N\rtimes_{\theta} Q$.
	\end{definition}
	%
	Se um grupo $G$ é produto semidireto de dois grupos cíclicos, então 
	$G$ é dito \textbf{metacíclico}. Em outras palavras, um grupo $G$ é \textbf{metacíclico} 
	se existe um subgrupo normal $N$ de $G$ tal que $N$ e $G/N$ são grupos cíclicos.
	%
	\begin{example}
	Por exemplo, em $D_n$, $n\geq2$, sejam $C_n = \langle r \rangle$ e 
	$C_2 = \langle s \rangle$, sendo $r$ uma rotação de $2\pi/n$ em torno do centro do polígono 
	e $s$ uma reflexão em torno da reta que liga o vértice 1 ao centro do polígono. Nesse caso, temos
	%
	\begin{equation*}
	    D_n = \langle r \rangle \rtimes_{\theta} \langle s \rangle = C_n\rtimes_{\theta}C_2,
	\end{equation*}
	%
	com $\theta_s(r^i) = r^{-i}$.
	\end{example}
	%
	\begin{example}
	O grupo alternado $A_n$ é normal em $S_n$ (pois tem índice 2) e 
	$C_2 = \left\{ (12) \right\}$ é isomorfo a $S_n/A_n$. Portanto, $S_n = A_n\rtimes C_2$.
	\end{example}
	%
	\begin{example}
	Um grupo cíclico de ordem $p^2$, $p$ primo, não pode ser escrito como 
	produto semidireto, uma vez que tem apenas um subgrupo de ordem $p$ e seriam necessários 2.
	\end{example}
	%
	\begin{example}
	Seja $G = GL(n,\F)$. Seja $B$ o subgrupo de matrizes triangulares superiores 
	em $G$, $T$ o subgrupo de matrizes diagonais em $G$ e $U$ o subgrupo de matrizes triangulares 
	superiores com coeficientes diagonais unitários. Então, para $n=2$,
	%
	\begin{equation*}
	    B = \left\{ \left(\begin{array}{cc}
        	\ast & \ast \\
        	0 & \ast 
        	\end{array} \right) \right\}, \text{ } 
        T = \left\{ \left(\begin{array}{cc}
        	\ast & 0 \\
        	0 & \ast 
        	\end{array} \right) \right\}, \text{ }
        U = \left\{ \left(\begin{array}{cc}
        	1 & \ast \\
        	0 & 1 
        	\end{array} \right) \right\}
	\end{equation*}
	%
	\par\vspace{0.3cm} Então, $U\vartriangleleft B$, $UT=B$ e $U\cap T = \left\{1\right\}$. Logo,
	%
	\begin{equation*}
	    B = U\rtimes T.
	\end{equation*}
	%
	\par\vspace{0.3cm} Note que para $n\geq 2$, $T$ não é normal em $B$ e, portanto
	%
	\begin{equation*}
	    B\neq T\rtimes U.
	\end{equation*}
	%
	\end{example}
	% 
	Vimos que, a partir de um produto semidireto $G = N \rtimes Q$, obtemos a terna
	%
	\begin{equation*}
	    (N, Q, \theta:Q\to\aut(N))
	\end{equation*}
	%
	e que essa terna determina $G$. Agora, vamos mostrar que toda terna 
	$(N,Q,\theta)$ consistindo de dois grupos $N$ e $Q$ e um homomorfismo $\theta:Q\to\aut(N)$ 
	surge de um produto semidireto. Então, seja $G = N\times Q$ e defina
	%
	\begin{equation*}
	    (n,q)(n',q') = (n\cdot\theta_q(n'), qq').
	\end{equation*} 
	%
	\begin{prop}
	\label{produto semidireto a partir de grupos}
		A lei de composição acima torna $G$ em um grupo, de fato, o produto semidireto de $N$ e $Q$.
	\end{prop}
	%
	\begin{proof}
		A partir da lei de composição, temos
		%
		\begin{equation*}
		    ( (n,q)(n',q') )(n'',q'') 
		    = ( n\theta_q(n'), qq' )(n'',q'') 
		    = ( n\cdot\theta_q(n')\cdot\theta_{qq'}(n''), qq'q'' ) 
		    = (n,q)( (n',q')(n'',q'') )
		\end{equation*}
		%
		\par\vspace{0.3cm} e, portanto, a associatividade vale. Como $\theta_1(1) = 1$ e $\theta_q(1) = 1$,
		%
		\begin{equation*}
		    (1,1)(n,q) = (n,q) = (n,q)(1,1)
		\end{equation*}
		%
		\par\vspace{0.3cm} e, portanto, $(1,1)$ é a identidade. Por fim, 
		%
		\begin{equation*}
		    (n,q)( \theta_{q^{-1}}(n^{-1}), q^{-1} ) = (1,1) = ( \theta_{q^{-1}}(n^{-1}), q^{-1} )(n,q)
		\end{equation*}
		%
		\par\vspace{0.3cm} e, portanto, $( \theta_{q^{-1}}(n^{-1}), q^{-1} )$ é o inverso de $(n,q)$. 
		Portanto, $G$ é um grupo e, como $N\vartriangleleft G$, $NQ=G$ e $N\cap Q = \left\{1\right\}$, 
		segue que $G = N \rtimes Q$. Além disso, quando $N$ e $Q$ são pensados como subgrupos de $G$, 
		a ação de $Q$ em $N$ é dada por $\theta$.
	\end{proof}
	%
	O produto direto pode ser pensado como um produto semidireto. De fato, a bijeção
	%
	\begin{equation*}
	    (n,q)\mapsto (n,q):N\times Q\to N \rtimes_{\theta} Q
	\end{equation*}
	%
	é um isomorfismo de grupos se, e só se, $\theta$ é o homomorfismo trivial 
	$Q\to \aut(N)$, i.e., $\theta_q(n) = n$ para todo $q\in Q, n\in N$.
	
	\par\vspace{0.3cm} Podemos usar o produto semidireto para construir grupos de ordem $p^3$. 
	De fato, seja $N = \langle a \rangle$ cíclico de ordem $p^2$ e seja $Q = \langle b \rangle$
	cíclico de ordem $p$, sendo $p$ um primo ímpar. Então $\aut(N)\cong C_{p-1}\times C_p$ e $C_p$ 
	é gerado por $\alpha:a\mapsto a^{1+p}$ (note que $\alpha^2(a) = a^{1+2p}$). 
	Defina $Q\to \aut(N)$ por $b\mapsto\alpha$. O grupo $G = N\rtimes_{\theta} Q$ 
	tem geradores $a,b$ e relações
	%
	\begin{equation}
	\label{com p^2}
	    a^{p^2} = 1, \text{ }b^p=1, \text{ }bab^{-1} = a^{1+p}.
	\end{equation}
	%
	$G$ é um grupo não comutativo de ordem $p^3$, e possui um elemento de ordem $p^2$.
	
	\par\vspace{0.3cm} Por outro lado, sejam $N = \langle a,b\rangle$ o produto de dois grupos cíclicos 
	$\langle a \rangle$ e $\langle b \rangle$ de ordem $p$ e $Q = \langle c \rangle$ um grupo cíclico 
	de ordem $p$. Defina $\theta: Q\to\aut(N)$ como o homomorfismo tal que
	%
	\begin{equation*}
	    \theta_{c^i}(a) = ab^i, \text{ }\theta_{c^i}(b) = b.
	\end{equation*}
	%
	O grupo $G = N\rtimes_{\theta} Q$ é um grupo de ordem $p^3$ com geradores $a,b,c$ 
	e relações
	%
	\begin{equation}
	\label{sem p^2}
	    a^p=b^p=c^p=1, \text{ }ab = cac^{-1}, \text{ }ab = ba, \text{ }bc = cb.
	\end{equation}
	%
	Como $b\neq 1$, a relação $ab = cac^{-1}$ mostra que $G$ não é comutativo. 
	Quando $p$ é ímpar, todo elemento não trivial tem ordem $p$. Quando $p=2$, $G\cong D_4$, que 
	tem um elemento de ordem $2^2$. Note que isso mostra que um grupo pode ter representações bem 
	diferentes como produto semidireto:
	%
	\begin{equation*}
	    D_4\cong C_4\rtimes C_2\cong (C_2\times C_2)\rtimes C_2.
	\end{equation*}
	%
	Para um primo ímpar $p$, um grupo não comutativo (ou não abeliano) de ordem $p^3$ 
	é isomorfo ao grupo em \eqref{com p^2} se possui elemento de ordem $p^2$ e isomorfo ao grupo em 
	\eqref{sem p^2} se não possui elemento de ordem $p^2$. Em particular, há exatamente dois grupos 
	não abelianos de ordem $p^3$, a menos de isomorfismo.