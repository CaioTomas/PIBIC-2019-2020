\chapter*{Introdução}
%\chaptermark{}
%
\hfill%
\begin{minipage}{10cm}
\begin{flushright}
\rightskip=0.5cm
\textit{``You know also that the beginning is the most important part of any work''}
\\[0.1cm]
\rightskip=0.5cm
--- Plato's Republic
\end{flushright}
\end{minipage}

\section*{O que é esse texto?}

    O presente texto é resultado de um projeto de PIBIC -- 
    Programa Institucional de Bolsas de Iniciação Científica --
    realizado no período de Agosto de 2019 a Setembro de 2020
    sob orientação da profª Sheila Campos Chagas da Universidade
    de Brasília.
    
    Nele, busquei registrar tudo que estudei e aprendi ao
    longo do projeto, assim como os {\it insights} e
    entendimentos que eu mesmo tive durante os estudos e 
    discussões com a profª.
    
\section*{O que você encontrará nele?}

    Os meus principais objetos de estudo ao longo do projeto
    foram as tranças e os nós. Como, na época, eu não havia cursado
    Álgebra 1 ainda, o primeiro passo foi estudar todos os 
    pré-requisitos de Teoria dos Grupos que eram necessários
    para entender a Teoria das Tranças e a Teoria dos Nós.
    Nessa primeira etapa, as principais referências foram
    \cite{Livro-do-Fraleigh, Gallian}. A primeira parte do
    livro se ocupa em tratar desses pré-requisitos, dividida
    em dois capítulos: o primeiro sobre a Teoria de Grupos
    ``genérica'', tratada em cursos iniciais de Álgebra Abstrata,
    e o segundo sobre grupos livres e alguns outros tópicos
    mais avançados da Teoria de Grupos.
    
    Estudados os pré-requisitos necessários, avançamos
    para o estudo das tranças e dos nós. Nesta etapa,
    as principais referências foram 
    \cite{knot-book, Livro-do-Margalit, Livro-do-Kunio}.
    A segunda parte do livro divide-se, então, em 4 capítulos:
    o primeiro introduz os tópicos que serão estudados nesta etapa;
    o segundo trata das principais propriedades (mais básicas)
    das tranças, de sua estrutura algébrica e das diferentes formas
    pelas quais podemos enxergá-las tranças;
    o terceiro foca nos nós, utilizando a teoria das tranças
    desenvolvida anteriormente para construir essa nova teoria;
    e o quarto e último capítulo traz, a título de curiosidade, 
    alguns tópicos interessantes e correlacionados com os temas
    tradados ao longo do livro.
    
    Por fim, é interessante comentar acerca das referências:
    todas listadas foram utilizadas de alguma maneira, seja
    como guia do estudo, seja apenas para uma breve consulta.
    Via de regra, as referências citadas ao longo do texto
    serviram de guia para alguma etapa do projeto, enquanto que
    as não citadas serviram para uma consulta breve.