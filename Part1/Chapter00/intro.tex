\chapter*{Introdução}
%\chaptermark{}
%
\hfill%
\begin{minipage}{10cm}
\begin{flushright}
\rightskip=0.5cm
\textit{``You know also that the beginning is the most important part of any work''}
\\[0.1cm]
\rightskip=0.5cm
--- Plato's Republic
\end{flushright}
\end{minipage}

\section*{O que é esse texto?}

    \par O presente texto é resultado de um projeto de PIBIC -- 
    Programa Institucional de Bolsas de Iniciação Científica --
    realizado no período de Agosto de 2019 a Setembro de 2020
    sob orientação da profª Sheila Campos Chagas da Universidade
    de Brasília.
    
    \par\vspace{0.3cm} Nele, busquei registrar tudo que estudei e aprendi ao
    no projeto, assim como os {\it insights} e
    entendimentos que eu mesmo tive durante os estudos e 
    discussões ao longo da pesquisa. Gostaria de deixar registrados
    os meus agradecimentos à professora Sheila, que me aceitou para
    participar do projeto ainda que eu não tivesse, a rigor, os pré-requisitos
    necessários. Agradeço também ao professor Igor dos Santos, que foi a ponte
    entre mim e a professora Sheila.
    
\section*{O que você encontrará nele?}

    \par Os meus principais objetos de estudo ao longo do projeto
    foram as tranças e os nós. Como, na época, eu não havia feito
    nenhum curso de Álgebra Abstrata ainda, o primeiro passo foi estudar todos os 
    pré-requisitos de Teoria dos Grupos que eram necessários
    para entender a Teoria das Tranças e a Teoria dos Nós.
    Nessa primeira etapa, as principais referências foram
    \cite{Livro-do-Fraleigh, Gallian}. A primeira parte do
    livro se ocupa em tratar desses pré-requisitos, dividida
    em dois capítulos: o primeiro sobre a Teoria de Grupos
    ``genérica'', tratada em cursos iniciais de Álgebra Abstrata,
    e o segundo sobre grupos livres e alguns outros tópicos
    mais avançados da Teoria de Grupos.
    
    Estudados os pré-requisitos necessários, avançamos
    para o estudo das tranças e dos nós. Nesta etapa,
    as principais referências foram 
    \cite{knot-book, Livro-do-Margalit, Livro-do-Kunio}.
    A segunda parte do livro divide-se, então, em 3 capítulos:
    o primeiro introduz os tópicos que serão estudados nesta etapa e
    trata das principais propriedades (mais básicas)
    das tranças, de sua estrutura algébrica e das diferentes formas
    pelas quais podemos enxergá-las;
    o segundo foca nos nós, utilizando a Teoria das Tranças
    desenvolvida anteriormente para construir essa nova teoria;
    e o terceiro e último capítulo traz, a título de curiosidade, 
    alguns tópicos interessantes e correlacionados com os temas
    tradados ao longo do livro.
    
    É interessante comentar acerca das referências:
    todas listadas foram utilizadas de alguma maneira, seja
    como guia do estudo, seja apenas para uma breve consulta.
    Via de regra, as referências citadas ao longo do texto
    serviram de guia para alguma etapa do projeto, enquanto que
    as não citadas serviram para uma consulta breve.
    
%\section{Conclusões}
    Além disso,
    os pontos principais que eu gostaria que você, leitor(a), levasse para a casa são:
    %
    \begin{enumerate}
        \item quando queremos formalizar um certo objeto, quanto mais ``liberdade'' ele tiver, mais difícil
        será o nosso trabalho;
        \item os invariantes são ferramentas poderosas para se lidar com certos objetos, como os nós;
    \end{enumerate}
    %
    \par Todos que têm um pouco mais experiência e vivência matemática já sentiram, 
    em algum momento, o primeiro item ``na pele''. Para aqueles que nunca pensaram nisso, as tranças 
    devem ser o exemplo principal:
    para formalizar esse objeto matematicamente, precisamos recorrer a uma teoria algébrica pesada e, 
    para analisar as propriedades deste objeto, precisamos de teorias mais rebuscadas ainda.
    
    \par\vspace{0.3cm} O segundo item merece um pouco mais de explicação. A discussão feita acerca
    dos polinômios de Alexander e Jones detectarem ou não o nó trivial é o germe dessa importância:
    os invariantes são uma das únicas ferramentas que nós temos para estudar e distinguir os nós
    (pelo menos por enquanto). A quantidade de invariantes é vasta\footnote{Essa lista foi obtida
    \href{http://katlas.org/wiki/Invariants}{deste site}}, e citamos alguns abaixo:
    %
    \begin{itemize}
        \item invariantes da teoria das tranças;
        \item invariantes tridimensionais;
        \item o polinômio de Alexander-Conway;
        \item o polinômio de Alexander multivariável;
        \item o determinante e a assinatura;
        \item o polinômio de Jones;
        \item os polinômios de Jones coloridos;
        \item o invariante A2;
        \item o polinômio HOMFLY-PT;
        \item o polinômio de Kauffman;
        \item os invariantes de Vassiliev;
        \item homologia de Khovanov;
        \item homologia Floer de nós de Heegaard;
        \item invariantes de R-Matriz.
    \end{itemize}
    %
    \par Alguns deles são algébricos, outros são geométricos e outros ainda são topológicos.
    Dependendo do problema que se tem em mãos, um deles pode ser mais adequado que os outros.
    
    \par\vspace{0.3cm} A noção de invariante também é relevante em outros problemas, como foi o caso
    da solução do 3º problema de Hilbert (não comentarei sobre a solução, mas quem se interessar pode
    ler mais sobre \href{https://en.wikipedia.org/wiki/Hilbert\%27s_third_problem}{aqui}.
    
    \par\vspace{0.3cm} Por isso, gostaria de finalizar esse texto incentivando o(a) leitor(a) a 
    reconsiderar a importância dos invariantes (ou a considerar, se nunca o havia feito antes), e
    tê-los em mente como ferramentas importantes e muito úteis em vários contextos.
    
    \par\vspace{0.3cm} Boa leitura!